\newacronym{avc}{AVC}{accident vasculaire cérébral}

\newacronym{chac}{CHAC}{Centre Hospitalier Alès - Cévennes}

\newacronym{imagina}{IMAGINA}{IMAges, Games and INtelligent Agents}

\newacronym{lirmm}{LIRMM}{Laboratoire d'Informatique, de Robotique et de Microélectronique de Montpellier}

\newacronym{m2h}{M2H}{Movement to Health Laboratory}

\newacronym{mojos}{MoJOS}{Moteur de Jeux Orienté Santé}

\newacronym{um2}{UM2}{Université Montpellier II}

\newglossaryentry{casual}{
	name={casual},
	description={Se dit d'une personne ne jouant à des jeux vidéo que de manière occasionnelle. Peut aussi se dire d'un jeu dont la cible sont des joueurs occasionnels}
}

\newglossaryentry{ergotherapie}{
	name={ergothérapie},
	description={L'ergothérapie est une profession de santé évaluant et traitant les personnes afin de préserver et développer leur indépendance et leur autonomie dans leur environnement quotidien et social}
}

\newglossaryentry{feedback}{
	name={feedback},
	description={retour d'information du jeu suite ou non à une action du joueur. 
Généralement : l’action en retour d’un effet sur le dispositif qui lui a donné naissance, et donc, ainsi, sur elle-même}
}

\newglossaryentry{fuglmeyer}{
	name={Fugl Meyer},
	description={Le test de Fugl Meyer est un ensemble d'exercices à réaliser permettant d'évaluer les capacités motrices d'une personne. Chaque exercice fournit un score dont l'évaluation totale permet d'estimer des progrès du patient}
}

\newglossaryentry{hemiplegie}{
	name={hémiplégie},
	description={Une hémiplégie est une paralysie d'un seul côté du corps. La paralysie peut affecter une ou plusieurs parties de l'hémicorps, jusqu'à être totale si la face, le tronc et les membres supérieurs et inférieurs sont paralysés}
}

\newglossaryentry{hypertonie}{
	name={hypertonie spastique},
	description={L’hypertonie spastique (musculaire) est une contraction réflexe du muscle qui s'oppose à l'étirement}
}

\newglossaryentry{lombalgie}{
	name={lombalgie},
	description={Une lombalgie est un état douloureux du rachis lombaire}
}

\newglossaryentry{paresie}{
	name={parésie},
	description={Perte partielle des capacités motrices d'une partie du corps (limitation de mouvement, diminution de la force musculaire), par opposition à la paralisie où le déficit moteur est total}
}

\newglossaryentry{spasticite}{
	name={spasticité},
	description={La spasticité consiste en un étirement rapide d'un muscle qui entraîne trop facilement sa contraction réflexe qui dure un certain temps}
}


%\newglossaryentry{}{
%	name={},
%	description={}
%}