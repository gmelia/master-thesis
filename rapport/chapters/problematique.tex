\subsection{Sujet et objectifs initiaux}
	Dans ce stage, nous allons nous intéresser à l’adaptation de la difficulté dans un jeu à but thérapeutique sous contrôle d’un médecin. Les variables permettant d’ajuster la difficulté d’un jeu sont nombreuses et variées. Elles peuvent être facilement modifiées et combinées pour créer des objectifs de jeu. Cependant, ces objectifs de jeu n’ont pas nécessairement un sens pour le médecin. Il s’agira dans ce stage de définir avec un thérapeute les variables ou paramètres adaptées à l’adaptation de la difficulté pour le soignant.
	\paragraph{}
	Dans le cadre de ce stage, le stagiaire aura pour mission de:
	\begin{itemize}
		\item {récupérer auprès d’un soignant une liste exhaustive d’objectifs de jeu thérapeutique}
		\item {traduire ces objectifs en paramètres dans le jeu Hammer \& Planks}
		\item {proposer une solution pour suivre et analyser visuellement les progrès du patient relativement aux objectifs fixés par le soignant}
	\end{itemize}
	Le stagiaire devra participer à des séances de coconception avec un thérapeute et sera amené à se déplacer pour suivre des séances de tests auprès de patients.
		
\subsection{Contexte et besoins}
Lors de mon arrivée au sein de NaturalPad, il existant déjà une version du jeu Hammer \& Planks, outil dans lequel devait s'insérer mon travail. Présenté lors du MIG 2012, le jeu était encore surtout orienté grand public. Ma première mission fut donc de m'approprier l'application et de l'adapter pour une utilisation paramétrable dans un contexte thérapeutique. En effet, une utilisation thérapeutique implique d'ajuster les différents paramètres en fonction des capacités et des besoins du patient.

	\paragraph{}Les échanges avec les professionnels de la santé ont été au coeur de mon stage. La compréhension des besoins et des contraintes médicales étant primordiales pour proposer un produit adapté, je me suis naturellement tourné vers les thérapeutes et soignants pour acquérir les connaissances et le background qu'il me manquait. Il m'est par ailleurs rapidement apparu qu'on ne pouvait répondre aux différents besoins thérapeutiques explicités par les soignants avec un seul jeu vidéo, même paramétrable. C'est pourquoi je me suis aussi concentré sur l'aspect de conception avec les thérapeutes. 

	\paragraph{Orientation du travail de stage \\}
Rappelons que la société NaturalPad propose comme outil une plateforme web permettant d'accéder à des jeux sérieux, et de les paramétrer directement depuis celle-ci. Hammer \& Planks constitue ainsi le premier jeu accessible depuis cette plateforme et, bien que servant d'exemple des possibilités d'un jeu vidéo pour la santé, il est amené à être rejoint par d'autres serious games. C'est dans cette optique que mon travail durant ce stage s'est progressivement orienté vers une méthode de conception de serious games pour la santé. Cela a pour objectif de proposer une solution appropriée aux différents besoins et contraintes de chaque situation. Ces derniers peuvent correspondre aux objectifs thérapeutiques, à la pathologie du patient, ses capacités, son âge, sa maitrise des nouvelles technologies ou son aisance avec les jeux vidéo par exemple.
Ce travail s'inscrit donc toujours dans le but d'adapter le jeu vidéo aux besoins thérapeutiques, et est complémentaire à une adaptation des paramètres de jeux, qu'elle soit manuelle ou automatique.
 
\paragraph{}Pour cela, je redéfinirais le sujet de ce stage comme suit :\\
\textcolor{marron}{\emph{ {\large Proposition d'une méthodologie de conception de jeux vidéo sérieux à but thérapeutique, et adaptation de la difficulté.}}}

\subsection{Outils et méthodologie}
	\subsubsection{Méthodologie}
Afin de mener à bien ses projets, l’équipe de NaturalPad emploie une méthode Agile de gestion de projet : SCRUM.

	\paragraph{\emph{Méthodes AGILES}\\}
Les méthodes Agiles sont des groupes de pratiques, réunies dans l'\emph{Agile Manifesto}\cite{Agil01}, s'appliquant dans la gestion de projets, généralement informatiques. Ces méthodes se veulent plus pragmatiques que les méthodes traditionnelles et diffèrent de celles-ci en se concentrant sur des valeurs humaines plutôt que sur les processus. Ce sont des méthodes itératives (ou plutôt semi-itératives), incrémentales et adaptatives.
\begin{figure}
	\centering
	\includegraphics[scale=0.5]{images/agile.jpg}
	\caption{Les méthodes agiles : un cycle semi-itératif}
	\label{agile}
\end{figure}
 
\paragraph{}
Les méthodes agiles prônent 4 valeurs fondamentales :
	\begin{enumerate}
		\item l'équipe : « Les individus et leurs interactions, plus que les processus et les outils. »
		\item L'application : « Des logiciels opérationnels, plus qu'une documentation exhaustive. »
		\item  La collaboration : « La collaboration avec les clients, plus que la négociation contractuelle. »
		\item L'acceptation du changement : « L'adaptation au changement, plus que le suivi d'un plan. »
	\end{enumerate}

		\paragraph{\emph{La méthode Scrum}\\}
Celle-ci définit 3 rôles :
	\begin{itemize}
		\item Le Product Owner
		\item Le Scrum Master
		\item Le Développeur
	\end {itemize}
Le Product Owner est le représentant des clients et des utilisateurs. Son objectif est de maximiser la valeur du produit développé. Il a pour rôle de rédiger des User Stories (comparables à des cas d'utilisation) et de valider le travail des développeurs. 
\\Le ScrumMaster est le responsable de la méthode. Il doit s’assurer qu’elle est correctement mise en application et comprise par les développeurs. Il organise le «Daily Scrum» (voir définition plus bas).
\\Enfin, le Développeur, représente en fait une équipe pluridisciplinaire et auto-organisée : toutes les décisions sont prises ensemble, sans hiérarchie externe ni interne.
 
		\paragraph{Daily Scrum :}
Il s’agit d’une réunion quotidienne ayant pour but de faire un point sur la coordination entre les tâches et les difficultés rencontrées.  Trois questions sont posées aux développeurs : 
	\begin{itemize}
		\item Qu’as-tu fait hier ?
		\item Qu’est-ce que tu vas faire aujourd’hui ?
		\item Est-ce que tu as rencontré des difficultés ?
	\end {itemize}
	
\paragraph{}Le travail est organisé sous forme de sprint. Il s’agit d’une courte période (au maximum un mois) au bout de laquelle l’équipe doit fournir une version améliorée du produit. Chaque sprint possède un but (ex : «on doit pouvoir envoyer des paramètres au jeu») et une liste de tâches (ex : «déterminer la méthode de communication, etc...»). Dès la fin d’un sprint, un nouveau est lancé.

\paragraph{}Enfin, une réunion a lieu en fin de sprint pour faire le point sur le travail accompli, les erreurs rencontrées et comment ne pas les éviter à l'avenir, ainsi que lancer le sprint suivant. Cette méthode est très intéressante car elle permet vraiment de garder une cohésion dans l’équipe de développement et d’avancer de manière visible. 

	\subsubsection{Outils}
		\paragraph{Gestion de projet\\}
Lors de mon arrivée dans l'entreprise, l’équipe utilisait Redmine, une application web de gestion de projets. Nous avons cependant changé deux fois d'outils de gestion de projet pendant la période de ce stage. Le premier est intervenu car les mises à jour des tâches dans Redmine étaient longues et l'outil finalement peu approprié à une méthodologie AGILE, ce qui freinait son utilisation. Nous avons donc mis en place une méthode Kanban qui consiste à écrire chaque tâche sur un post-it, et de déplacer ce post-it dans des colonnes «A faire», «En cours», «Terminé» ou «Validé» selon son avancement par exemple. De cette manière, l’avancement global était bien plus visible mais cette solution était finalement gourmande en post-it et en place. C'est pourquoi, nous utilisons désormais \href{www.trello.com}{Trello}, un outil de gestion de projet en ligne, se basant sur la méthode Kanban. Il s’agit d’un tableau virtuel dans lequel nous pouvons facilement déplacer les tâches, ajouter des commentaires ou des contraintes de temps notamment.

	\begin{figure}[!h]
		\centering
		\includegraphics[height=48px]{images/redmine.jpg}
		\includegraphics[height=48px]{images/trello.jpg}
		\caption{Logos de Redmine et Trello}
		\label{logos_redmine_trello}
	\end{figure}
 \newpage
		\paragraph{Développement}
		\subparagraph{} \emph{Unity3D\\}
La majeure partie technique de mon travail a été réalisée pour Hammer \& Planks, qui est développé avec le moteur de jeu Unity3D. Au fil de mon stage, nous sommes passés de la version 3.9 à la version 4.1. Unity permet de facilement intégrer les modèles 3D des objets réalisés dans les logiciels de modélisation 3D tels que Photoshop, Gimp ou Maya. Il propose aussi des options permettant d'utiliser un gestionnaire de versions pour les fichiers du projet.
	\begin{figure}[!h]
		\centering
		\includegraphics[height=48px]{images/unity.jpg}
		\caption{Logo d'Unity3d}
		\label{logo_unity}
	\end{figure}

		\subparagraph{} \emph{Play! Framework\\}
Play! Framework est un framework open source web qui permet d'écrire rapidement des applications web en Java ou en Scala. Il vise à apporter un outil simple et productif sur la machine virtuelle Java. Play Framework a pour particularité de ne pas être basé sur le moteur Java de Servlet. Il propose par ailleurs un moteur de template basé sur Scala.
	\begin{figure}[!h]
		\centering
		\includegraphics[height=48px]{images/play.png}
		\caption{Logo de Play! Framework}
		\label{logo_play}
	\end{figure}		

		\subparagraph{} \emph{Git\\}
Que ce soit pour Hammer \& Planks ou nos autres projets en cours, l'utilisation d'un gestionnaire de versions se révèle vite indispensable. Travaillant en équipe allant jusqu'à cinq développeurs et une graphiste, il est nécessaire de pouvoir mutualiser le travail. De plus, l'expérimentation et le développement de nouveaux éléments se prêtent très bien à l'utilisation de plusieurs branches de développement, chose que Git permet de gérer facilement.
	\begin{figure}[!h]
		\centering
		\includegraphics[height=48px]{images/git.png}
		\caption{Logo de Git}
		\label{logo_git}
	\end{figure}

		\subparagraph{}	\emph{BitBucket et GitHub}
Pour héberger ses projets, NaturalPad avait l'habitude d'utiliser GitHub. Avec l'arrivée de nouveaux stagiaires, il nous a fallu trouver une solution permettant un accès privé au dépôt pour un plus grand nombre de personnes, ce que permet BitBucket.
	\begin{figure}[!h]
		\centering
		\includegraphics[height=48px]{images/bitbucket.jpg}
		\includegraphics[height=48px]{images/github.jpg}
		\caption{Logos de BitBucket et Github}
		\label{logos_bitbucket_github}
	\end{figure}

	\subsubsection{Veille}
Le Jeu Vidéo et plus généralement l'Informatique est un domaine en constante évolution dans lequel il est nécessaire de se tenir à jour pour connaître les dernières technologies et actualités. Pour cela, j'ai observé durant l'intégralité de ma période de stage une veille technologique et stratégique. Nouveautés technologiques, logiques ou matérielles, communications d'entreprises ou de salons nationaux et internationaux ou bien encore annonces de sociétés dont le secteur d'activité est compatible avec NaturalPad ont donc été au coeur de mon étude quotidienne.
\paragraph{}Pour faciliter ce travail de veille, par ailleurs inclu dans mon planning, j'utilise un agrégateur de flux RSS, outil indispensable pour gérer aisément un contenu important sur un grand nombre de sources différentes. Il s'agit ensuite de mettre à jour et d'étendre régulièrement les sources en fonction de l'utilité observée de chacune d'entre elle ou des manques ressentis. 
\paragraph{Jeux Vidéo\\ \quad}
Étant étudiant en Informatique, option Image Game and Intelligent Agents, et ayant orienté ma formation vers une spécialité Jeux Vidéo, il m'a semblé important de me tenir à jour en terme d'actualité vidéoludique. J'ai pour cela étendu ma veille aux domaines des jeux vidéo, indépendants ou blockbusters, afin d'en étudier différents aspects tels le business model, le gameplay, les technologies employées ou les mécanismes de jeu innovants par exemple. J'ai ainsi pu testé des technologies récentes comme la console Ouya ou le système de contrôle Leap Motion, qui permet d'interagir en utilisant ses mains et ses doigts. Pour plus d'informations sur le Leap Motion, vous pouvez retrouver mon billet sur le blog de \href{naturalpad.fr/category/naturalblog}{NaturalPad}.
\begin{figure}
	\centering
	\includegraphics{images/leap_motion.jpg}
	\caption{Utilisation d'un Leap Motion}Une application retranscrit à l'écran la "vision" qu'elle a des mains de l'utilisateur.
	\label{leapmotion}
\end{figure}

	\subsubsection{État de l'art et recherche documentaire}
Étendue sur plusieurs semaines, la réalisation de l'état de l'art était pour moi quelque chose de nouveau qui a nécessité une certaine organisation. Deux moments ont été importants : la phase de démarrage et l'arrêt des recherches. Commencer ces recherches alors que les sujets que je devais ou voulais couvrir étaient vastes et non clairement définis fut à la fois plaisant et compliqué. Bien que beaucoup de choses me semblaient intéressantes, la question était de savoir par où commencer. De la même manière, au fil de mes lectures, je découvrais de nouveaux liens et références présentant de nouveaux aspects qui eux-même renvoyaient vers d'autres solutions et articles. La difficulté était donc de juger quand mes connaissances sur un thème donné étaient suffisantes pour éviter de poursuivre d'interminables recherches, aussi intéressantes puissent elles être, le temps étant limité dans le cadre d'un stage de Master.
	\paragraph{}Au niveau des lectures abordées, ma source principale fut des articles scientifiques, suivie par des articles de magazines spécialisés et articles web notamment. Pour les thèmes médicaux, un certain nombre d'articles m'a été directement conseillé par des professionnels de la santé, articles à partir desquels j'ai ensuite pu compléter mon état de l'art en suivant les références. Par ailleurs, il est fréquent que certains auteurs ressortent régulièrement lorsqu'on effectue des recherches sur un thème donné, ce qui permet, en plus du nombre de citations des articles, de rapidement cerner quels sont les articles et chercheurs de référence dans le domaine.
	\paragraph{}Bien entendu, afin de rendre mes lectures efficaces, je me constituais pour chacune d'elle une fiche de lecture où noter les points importants : 
\begin{itemize}
	\item Titre ou source
	\item Auteur(s)
	\item Mots clefs
	\item Synthèse
	\item Jugement personnel ou remarques
	\item Références importantes
\end{itemize}