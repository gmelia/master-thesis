\subsection*{Méthodologie de conception de serious games pour la rééducation motrice}
Nous proposons ici une méthodologie permettant de guider la création d’un jeu sérieux de ce type à partir des objectifs thérapeutiques et en se basant sur des types de jeux vidéo connus dont le gameplay a été largement testé et approuvé par la communauté dans l’histoire du jeu vidéo. Les divers paramètres de jugement de cette méthodologie pourront être : gain de temps de conception, quantité de jeux conçus, qualité des jeux conçus, impact sur les objectifs thérapeutiques, satisfaction des joueurs ou des thérapeutes.

\paragraph{}L’idée est de se baser à la fois sur les objectifs santé et sur les ressorts ludiques pour faire émerger un jeu respectant leurs contraintes respectives.
D’un côté, on regarde quels sont les objectifs thérapeutiques que l’on souhaite atteindre, puis à partir de la description des exercices nécessaires à leur réalisation, en induire le type de contrôle permettant ces exercices.\\
De l’autre côté, on part des types de jeux existants et connus (voir annexe \ref{types_jeux}), on regarde quel type de gameplay ceux-ci proposent et on en déduit un ou plusieurs types de contrôle adaptés.\\
L’intersection de ces résultats devrait ainsi nous permettre de faire le lien entre les deux parties pour créer un serious game adapté. 

\paragraph{}L’intérêt d’une telle approche est de proposer une méthodologie qui, à partir d’un besoin thérapeutique, permet de suggérer un jeu vidéo en adéquation avec ce besoin. De cette manière, on prend en considération les deux aspects sérieux et ludique d’un serious game dès sa conception, tout en profitant d’une certaine réemployabilité du résultat. Dans la pratique, elle pourrait être utilisée pour la conception de serious games venant enrichir la plateforme en ligne de NaturalPad. 

\paragraph{}Un complément possible serait pour chaque jeu, de proposer une série de paramètres permettant d’ajuster les différents éléments de difficulté identifiés comme tels par le thérapeute par rapport à l’objectif recherché. Cette personnalisation du jeu et donc de la thérapie est un point important dans l’efficacité de celle-ci. Par ailleurs, ajuster le jeu aux capacités du patient permet de maintenir son intérêt et sa motivation, et donc de garantir ou d’améliorer l’impact du jeu sérieux. En ajustant ces variables, il deviendrait possible pour les thérapeutes d’adapter plus précisément l’exercice en fonction du patient et de ses spécificités (pathologie, forme physique, morphologie, familiarité avec les jeux vidéo, etc.). Cela nécessite pour chaque exercice un travail d’association entre ses paramètres de jeu et les différents éléments de difficulté de cet exercice.

	\subsubsection*{Application théorique de la méthode}
Une fois défini le type d'exercice que l’on cherche à réaliser, cela devrait alors nous proposer un ou plusieurs types de gameplay permettant de répondre à ces besoins, voir un jeu particulier avec un ensemble de valeurs déterminées et un autre ensemble de valeurs à ajuster en fonction du patient. Par exemple, dans les paramètres déterminés, on aurait le type de contrôleur et les contrôles possibles ; dans les paramètres à ajuster, la fréquence et la localisation des obstacles.