	%%PART 4 : REHABILITATION			
	
	\subsection{Rééducation bimanuelle} \label{bilateral}
	Dans le cadre du développement du jeu Hammer \& Planks et de mon travail d'adaptation de contenu de serious games pour la rééducation, j'ai essayé d'envisager toutes les possibilités offertes par les interfaces naturelles ou \gls{nui}. Très vite, il m'est apparu que la richesse d'un gameplay par un contrôle naturel était liée au panel de possibilités de ce contrôle. Ainsi a priori, proposer de nombreux mouvements, notamment avec les deux membres supérieurs, pour interagir avec le jeu semble une bonne idée. Mais dans le cadre d'une réhabilitation post AVC, qui est donc la pathologie sur laquelle je me suis concentrée, il fallait m'assurer à la fois de la possibilité et de la pertinence d'une telle proposition. J'ai ainsi dirigé mes recherches vers un aspect beaucoup plus médical sur la réhabilitation bimanuelle post AVC. J'ai pour cela rencontré Julien Métrot, doctorant en médecine au laboratoire \emph{Movement 2 Health} et dont la spécialité est la récupération sensorimotrice des membres supérieurs. Il est aussi le co-auteur d'une étude sur le sujet.
	
		\subsubsection*{Évolution de la coordination bimanuelle} 
		
		\paragraph{\emph{Informations médicales}\\}
Une tâche nécessitant les deux mains entraîne une activation synchronisée des deux hémisphères, avec une balance égale du contrôle de l'inhibition. Cette activation est basée sur des concepts comportementaux et neurophysiologiques.\\
Le membre sain entraîne le membre membre parétique (voir \gls{paresie}) dans l'exécution de la tâche et améliore son résultat. A l'inverse, le membre affecté bride le membre non parétique à ses propres capacités		
		\paragraph{\emph{Analyse}\\}
Aujourd'hui, l'intérêt d'une thérapie bimanuelle fait débat car les études sur le sujet ne montrent pas de preuves assez probantes pour conclure à un apport bénéfique d'un \gls{bat}, ni qu'une thérapie BAT serait plus adaptée qu'une autre. Cependant, ces études ne prennent pas en compte toutes les phases de rétablissement : aiguë, subaiguë et chronique.\newline
[Metrot et al, 2013]\cite{Metr13} centrent leur analyse sur l'évolution de la coordination bimanuelle dans les six semaines qui suivent l'attaque ayant entrainée l'hémiplégie.

\paragraph{}
Les mesures cinématiques sont fortement suspectées de fournir une précision accrue et des mesures plus sensibles que les résultats cliniques [Rohrer et al, 2002 ; Alt Murphy et al, 2011] qui ne donnent qu'un score global et ne clarifient pas la question de savoir si les fluctuations proviennent d'un rétablissement moteur ou d'une compensation ni de comment les mouvements ont été réalisés. 

\paragraph{}
La fluidité du geste et le timing des mouvements semblent être les variables de mesure les plus sensibles de la récupération.

\paragraph{}
D'une manière générale, les actions bimanuelles sont plus longues et complexes à effectuer que les actions unimanuelles. A la surcharge cognitive de la coordination des deux mains, s'ajoute la plus grande dépendance de la main parétique. Les actions bimanuelles demandent aux patients un plus grand nombre de sous mouvements. 

		\paragraph{\emph{Constatations}\\}
On constate de forts progrès durant les trois premières semaines, avant d'atteindre une phase de plateau (progrès nuls) aux alentours de la 6ème semaine. Ce schéma concorde avec le fait que la plupart des progrès sont réalisés très tôt après l'attaque, puis diminuent progressivement. On peut considérer les progrès des 3 premières semaines comme un rétablissement spontané. La coordination des 2 mains durant les mouvements bimanuels est cependant considérée comme efficace autours de la 3ème semaine après l'inclusion (6ème semaine après l'attaque), indiquant une potentielle fenêtre pour un programme de réhabilitation. Ou à deux mois, après la phase de plateau.		

\paragraph{}
Whitall \cite{Whit04} indique que les patients légèrement affectés tireraient davantage de bénéfices d'une rééducation unilatérale, alors que les patients moyennement affectés auraient un meilleur retour d'une rééducation bilatérale. De plus, la coordination bimanuelle pourrait servir d'indicateur de rétablissement, et servir à informer de l'utilité ou non d'une rééducation bilatérale.

		\subsubsection*{Application dans Hammer \& Planks}
Hammer \& Planks est un jeu vidéo sérieux pour la santé dont la cible d'utilisateurs initiale sont des patients en période de réhabilitation post AVC. Comme nous venons de le voir, selon la phase de récupération et l'importance du handicap des patients, il peut être intéressant d'encourager une thérapie bimanuelle. On peut alors imaginer plusieurs types de contrôles et de gameplays allant dans ce sens~:
\begin{itemize}
	\item \emph{mouvement bilatéral symétrique} : pour réaliser une action, le joueur-patient doit réaliser un geste identique avec ces deux membres supérieurs. Il faut cependant penser que la personne risque de sur-utiliser son bras valide. On peut alors penser à un système permettant de ne prendre en compte que le geste le moins complet, correspondant a priori au membre parétique, pour asservir le contrôle. De cette manière, cela encouragerait le joueur à stimuler son hémicorps atteint.
	\item \emph{mouvement bilatéral asymétrique} : Très intéressant d'un point de vue gameplay, mais faire attention aux capacités cognitives du patient qui doivent être suffisamment bonnes pour comprendre les gestes à faire et réussir à les conceptualiser pour ensuite les réaliser. Cela est d'autant plus compliqué que les personnes hémiplégiques ont généralement des troubles cognitifs.
\end{itemize}