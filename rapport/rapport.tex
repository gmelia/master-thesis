\documentclass[french, 12pt]{article} % use larger type; default would be 10pt

\usepackage[utf8]{inputenc} % set input encoding (not needed with XeLaTeX)
\usepackage[T1]{fontenc}

%%% PAGE DIMENSIONS
\usepackage{geometry} % to change the page dimensions
\geometry{a4paper} % or letterpaper (US) or a5paper or....
% \geometry{margin=2in} % for example, change the margins to 2 inches all round

%PACKAGES
\usepackage{booktabs} % for much better looking tables
\usepackage{array} % for better arrays (eg matrices) in maths
\usepackage{paralist} % very flexible & customisable lists (eg. enumerate/itemize, etc.)
\usepackage{verbatim} % adds environment for commenting out blocks of text & for better verbatim
\usepackage{subfig} % make it possible to include more than one captioned figure/table in a single float
\usepackage{graphicx} % support the \includegraphics command and options
\usepackage[colorlinks=true]{hyperref} % permet l'utilisation d'hyperliens, et active leur coloration ( a definir)
\usepackage{pdfpages} %% permet d'importe des pages pdf
\usepackage[francais]{babel}

%GLOSSAIRE
%\usepackage{glossary}
\usepackage[toc]{glossaries}
\makeglossaries
\newacronym{AVC}{AVC}{accident vasculaire cérébral}

\newacronym{bat}{BAT}{bimanual arm training}

\newacronym{chac}{CHAC}{Centre Hospitalier Alès - Cévennes}

\newacronym{fps}{FPS}{First Person Shooter}

\newacronym{gui}{GUI}{graphical user interface}

\newacronym{ia}{IA}{intelligence artificielle}

\newacronym{imagina}{IMAGINA}{IMAges, Games and INtelligent Agents}

\newacronym{lirmm}{LIRMM}{Laboratoire d'Informatique, de Robotique et de Microélectronique de Montpellier}

\newacronym{m2h}{M2H}{Movement to Health Laboratory}

\newacronym{mig}{MIG}{Montpellier In Game}

\newacronym{mojos}{MoJOS}{Moteur de Jeux Orienté Santé}

\newacronym{mvc}{MVC}{Modèle-Vue-Contrôleur}

\newacronym{nui}{NUI}{natural user interface}

\newacronym{qte}{QTE}{Quick-Time Event}

\newacronym{rpg}{RPG}{Role Playing Game}

\newacronym{rts}{RTS}{Real Time Strategy}

\newacronym{sg}{SG}{Serious Game}

\newacronym{um2}{UM2}{Université Montpellier II}

\newglossaryentry{casual}{
	name={casual},
	description={Se dit d'une personne ne jouant à des jeux vidéo que de manière occasionnelle. Peut aussi se dire d'un jeu dont la cible sont des joueurs occasionnels}
}

\newglossaryentry{ergotherapie}{
	name={ergothérapie},
	description={L'ergothérapie est une profession de santé évaluant et traitant les personnes afin de préserver et développer leur indépendance et leur autonomie dans leur environnement quotidien et social}
}

\newglossaryentry{feedback}{
	name={feedback},
	description={retour d'information du jeu suite ou non à une action du joueur. 
Généralement : l’action en retour d’un effet sur le dispositif qui lui a donné naissance, et donc, ainsi, sur elle-même}
}

\newglossaryentry{fuglmeyer}{
	name={Fugl Meyer},
	description={Le test de Fugl Meyer est un ensemble d'exercices à réaliser permettant d'évaluer les capacités motrices d'une personne. Chaque exercice fournit un score dont l'évaluation totale permet d'estimer des progrès du patient}
}

\newglossaryentry{hardcore}{
	name={hardcore},
	description={Par opposition au jeu \gls{casual}, le jeu hardcore cible les très gros joueurs et possède un niveau de difficulté très élevé, demandant un fort investissement de la part du joueur. On appelle aussi ce type de joueurs des hardcore gamers}
}

\newglossaryentry{hemiplegie}{
	name={hémiplégie},
	description={Une hémiplégie est une paralysie d'un seul côté du corps. La paralysie peut affecter une ou plusieurs parties de l'\gls{hemicorps}, jusqu'à être totale si la face, le tronc et les membres supérieurs et inférieurs sont paralysés}
}

\newglossaryentry{hemicorps}{
	name={hémicorps},
	description={Moitié latérale du corps humain (gauche ou droite)}
}

\newglossaryentry{hypertonie}{
	name={hypertonie spastique},
	description={L’hypertonie spastique (musculaire) est une contraction réflexe du muscle qui s'oppose à l'étirement}
}

\newglossaryentry{lombalgie}{
	name={lombalgie},
	description={Une lombalgie est un état douloureux du rachis lombaire}
}

\newglossaryentry{paresie}{
	name={parésie},
	description={Perte partielle des capacités motrices d'une partie du corps (limitation de mouvement, diminution de la force musculaire), par opposition à la paralysie où le déficit moteur est total}
}

\newglossaryentry{serious gaming}{
	name={serious gaming},
	description={Dérivation de l'utilisation d'un jeu vidéo classique dans un but sérieux.}
}

\newglossaryentry{spasticite}{
	name={spasticité},
	description={La spasticité consiste en un étirement rapide d'un muscle qui entraîne trop facilement sa contraction réflexe qui dure un certain temps}
}

\newglossaryentry{skill}{
	name={skill},
	description={Technique ou compétence dans un jeu vidéo, peut représenter les actions possibles d'un personnage.}
}
%\newglossaryentry{}{
%	name={},
%	description={}
%}


%COULEURS
\usepackage{color} % permet de définir des couleurs pour les utiliser localement
\definecolor{orange}{rgb}{0.8, 0.4, 0.1}
\definecolor{vert}{rgb}{0.27, 0.57, 0.13}
\definecolor{marron}{rgb}{0.6, 0.2, 0.13}

% BIBLIOGRAPHY DEFINITIONS
\bibliographystyle{plain}


% HEADERS & FOOTERS
\usepackage{fancyhdr} % This should be set AFTER setting up the page geometry
\pagestyle{fancy} % options: empty , plain , fancy
\renewcommand{\headrulewidth}{0pt} % customise the layout...
\lhead{}\chead{}\rhead{}
\lfoot{}\cfoot{\thepage}\rfoot{}

% SECTION TITLE APPEARANCE
\usepackage{sectsty}
\allsectionsfont{\sffamily\mdseries\upshape} % (See the fntguide.pdf for font help)

% table of contents APPEARANCE
\usepackage[nottoc,notlof,notlot]{tocbibind} % Put the bibliography in the ToC
\usepackage[titles,subfigure]{tocloft} % Alter the style of the Table of Contents
\renewcommand{\cftsecfont}{\rmfamily\mdseries\upshape}
\renewcommand{\cftsecpagefont}{\rmfamily\mdseries\upshape} % No bold!


\title{Rapport de stage - M2 Informatique}
\author{Mélia Geoffrey}

\date{} % Activate to display a given date or no date (if empty), otherwise the current date is printed 

\ifpdf
	\pdfinfo
	{
		/Author (Geoffrey MELIA)
		/Title (Master thesis)
		/Subject (Difficulty adaptation in rehabilitation SG)
		/Keywords (serious games ; rehabilitation ; difficulty adaptation ; player behavior ; conception)
		/CreationDate (\today)
	}
\fi

\begin{document}
\maketitle

\begin{picture}(0,0)
	\put(-30,-550){\includegraphics[scale=0.8]{images/logo_um2.png}}
	\put(280,-570){\includegraphics[scale=0.6]{images/logo_naturalpad.png}}
	
	\put(-30,-120){\textsc{\LARGE{Vers une méthode de conception de jeux}}}
	\put(10,-150){\textsc{\LARGE{vidéo sérieux à but thérapeutique}}}
	\put(100,-180){\textsc{\large{et adaptation de la difficulté}}}
	
	\put(-30,-360){\textsc{\large{tuteurs universitaires : Vincent Boudet \& Nancy Rodriguez}}}
	\put(-30,-400){\textsc{\large{effectué chez la société NaturalPad}}}
	\put(-30,-420){\textsc{\large{sous la direction de Antoine Seilles}}}
\end{picture}

\newpage \newpage
\section*{Remerciements}
\addcontentsline{toc}{section}{Remerciements}
Je tiens tout d'abord à remercier Antoine Seilles pour avoir conçu et proposé un sujet de stage en accord à la fois avec les besoins de l'entreprise et mes affections, pour son aide avisée tout au long de ce stage ainsi que pour la confiance qu'il a mise en moi et en mon travail.

\paragraph{}Je remercie aussi Inès Di Loretto qui, malgré la distance, m'a apporté de manière régulière et amicale son aide, son expérience de chercheuse, ses idées et son regard critique sur mon travail, et a su me guider et me faire poser les bonnes questions.

\paragraph{}Merci à Sebastien Andary et Anthony Barreau pour avoir partagé leurs connaissances techniques et technologiques et contribué à améliorer mon travail. Merci aussi pour m'avoir soutenu et contribué au bon déroulement de mon stage.

\paragraph{}Je remercie Nancy Rodriguez pour son suivi et conseils sur mon travail, dont l'oeil critique était le bienvenu.

\paragraph{}Un grand merci à Didier Costeau\footnote{Kinésithérapeute, en libéral à Montpellier}, Karima Bahkti\footnote{Kinésithérapeute au centre hospitalier de Lapeyronie, Montpellier}, Julien Métrot\footnote{PhD student au laboratoire d'EUROMOV - Movement 2 Health}, Arnaud Dupeyron\footnote{Docteur en médecine spécialisé dans la lombalgie, EUROMOV, Movement 2 Health} et Anaïs Ivorra\footnote{Ergothérapeute} pour m'avoir accordé leur temps et partagé leur expérience et connaissances médicales, sans lesquelles aucun travail n'aurait été possible, et de manière générale à tous les professionnels de la santé impliqués dans ce projet.

\paragraph{}Merci enfin à toute l'équipe de NaturalPad et à Fabien Dutartre pour l'expérience humaine à laquelle ils m'ont fait participer durant ces six mois, quand travail et plaisir ne font plus qu'un.


\definecolor{red}{rgb}{0,0,0}
\newpage
\tableofcontents

\newpage
\listoffigures
\definecolor{red}{rgb}{1,0,0}

\newpage 

\part{Rapport de stage}
	\section{Introduction}
	\subsection{Généralités}
Ce stage réalisé dans le cadre de ma 2$^{eme}$ année de Master Informatique fut l'occasion de poursuivre mon travail dans le domaine de la santé débuté durant mon TER. Succintement, ce travail consistait en l'expérimentation de l'utilisation de technologies informatiques pour l'évaluation des capacités motrices de patients hémiplégiques. Ce projet de TER a été réalisé avec William DYCE, par ailleurs stagiaire avec moi chez NaturalPad. Nous avons conçu un prototype d'application permettant, grâce à la Kinect, de juger de la réussite ou non de plusieurs des exercices d'évaluation du test de \gls{fuglmeyer} par un patient en réhabilitation suite à un AVC. J'ai par ailleurs déjà réalisé un stage de deux mois chez NaturalPad durant l'été 2012, à la fin de ma première année de Master.


\paragraph{}
Fort de cette expérience et vivement motivé par le fait de travailler dans un contexte médical, ce stage constituait une excellente opportunité d'orienter ma formation vers le développement d'applications et de jeux pour la santé.\\
L'objectif de ce stage était de proposer un moyen ou des outils pour adapter la difficulté de jeux sérieux pour la santé dans le domaine de la rééducation motrice. Nous verrons que différentes approches sont possibles et quelle fut démarche lors de ce stage. Par ailleurs, ce fut pour moi l'occasion de m'insérer dans le projet de grande envergure, Hammer \& Planks, et de participer à mes deux premières GameJam. Vous pouvez retrouver un résumé de ces expériences et de mon ressenti sur le \href{http://naturalpad.fr/la-game-jam-chez-naturalpad/}{blog}\footnote{http://naturalpad.fr/la-game-jam-chez-naturalpad} de NaturalPad.

\subsection{L'entreprise}
	\subsubsection{Naturalpad}
NaturalPad est une jeune startup innovante basée dans la région de Montpellier. Elle est spécialisée dans les technologies appliquées à la santé et développe notamment des jeux vidéos à but thérapeutique pour la rééducation fonctionnelle. Ces jeux utilisent des technologies de capture du mouvement grand public comme le Kinect de Microsoft ou la Wii Board de Nintendo. L’idée de Naturalpad est née lors du projet \gls{mojos} , pionnier en matière de projet de recherche serious game, sur lequel plusieurs membres de l’équipe ont travaillé. En effet, prenant conscience du marché porteur des serious games et plus largement de celui de la santé, Antoine Seilles et les 4 autres fondateurs ont créé le projet NaturalPad en se démarquant des concurrents par une volonté de faire avant tout du jeu, bien qu’il soit sérieux, et d’y apporter une dimension sociale. L’idée au delà du développement de jeux est de proposer un système sous forme de plateforme intégrant plusieurs jeux thérapeutiques. Ceux-ci permettant au patient de suivre sa thérapie à domicile tout en jouant avec ses proches (considérés dans la conception du gameplay) et donnant la possibilité au thérapeute de suivre la progression du patient. NaturalPad a par ailleurs signé un partenariat avec le \gls{chac} dans le but de développer des services web innovants pour ses équipes de pédiatrie, néonatalogie et gériatrie. Le \gls{chac} peut être considéré comme le site pilote des produits NaturalPad. Un partenariat de recherche a été mis en place avec deux laboratoires de recherche de Montpellier que sont le \gls{lirmm} et le \gls{m2h} et l'entreprise de NaturalPad.

	\subsubsection{L'équipe}
	\begin{figure}[!h] 
		\centering
		\includegraphics[width=375px]{images/naturalpad_groupe.jpg}
		\caption{Une partie de l'équipe de NaturalPad à Prades le Lez}
		\label{naturalpad_groupe}
	\end{figure}
	
		\subsubsection*{Antoine SEILLES - Fondateur, CEO}
\begin{minipage}[t!]{0.2\linewidth}
\centering
\includegraphics[width=0.8\textwidth]{images/tetocarre/antoine}
\end{minipage}
\begin{minipage}[t!]{0.79\linewidth}
Antoine Seilles est docteur en informatique. Sa thèse, soutenue en avril 2012 porte sur les usages du Web 3.0 (ou web socio-sémantique) dans le contexte de la démocratie électronique. Les domaines de recherche d’Antoine portent essentiellement sur l’aspect social du Web, thème sur lequel il a participé à plusieurs conférences grand public, et sur le Web sémantique, thème sur lequel il a publié à plusieurs reprises et traduit le livre de Tom Heath et Christian Bizer «Linked Data».
		\begin{quotation} \emph{Au sein de NaturalPad} \end{quotation}
Chez NaturalPad, la recherche d’Antoine porte sur les solutions Web 2.0 autour du Dossier Médical Personnel et sur les formats de données utilisant les technologies du Web sémantique (RDF notamment) pour la télémédecine.
\end{minipage}

		\subsubsection*{Sébastien ANDARY - Fondateur}
\begin{minipage}[t!]{0.2\linewidth}
\centering
\includegraphics[width=0.8\textwidth]{images/tetocarre/seb}
\end{minipage}
\begin{minipage}[t!]{0.79\linewidth}
Il a reçu le diplôme de Master d’Informatique à l’Université de Montpellier et termine actuellement un Doctorat de Robotique au \gls{lirmm}. Sa thèse porte sur la commande des systèmes mécaniques sous-actionnés dans le cadre de la robotique humanoïde. 
		\begin{quotation} \emph{Au sein de NaturalPad} \end{quotation}
Sébastien travaille au développement de technologies et outils innovants pour la captation de mouvements.
\end{minipage}

		\subsubsection*{Inès DI LORETO - Fondatrice}
\begin{minipage}[t!]{0.2\linewidth}
\centering
\includegraphics[width=0.8\textwidth]{images/tetocarre/ines}
\end{minipage}
\begin{minipage}[t!]{0.79\linewidth}
Elle est diplômée en philosophie et a obtenu un doctorat en informatique à l’Università degli Studi di Milano (Italie). Fin 2009, elle rejoint en post-doc le projet \gls{mojos}. Dans ce projet de recherche sur les jeux thérapeutiques elle a mené des activités scientifiques pour relever les défis de l’acceptation des patients et des thérapeutes du jeu comme moyen de rééducation. 
		\begin{quotation} \emph{Au sein de NaturalPad} \end{quotation}
Inès travaille sur la transformation des objectifs thérapeutiques en objectifs de jeu en collaboration directe avec les professionnels de la santé.
\end{minipage}

		\subsubsection*{Tristan LE GRANCHE - Fondateur}
\begin{minipage}[t!]{0.2\linewidth}
\centering
\includegraphics[width=0.8\textwidth]{images/tetocarre/tristan}
\end{minipage}
\begin{minipage}[t!]{0.79\linewidth}
Travaillant dans le monde du cinéma d’animation depuis plus de 4 ans, il a travaillé sur plus d’une dizaine de projets de séries télévisées, et de longs et courts métrages. Aujourd’hui il s’est spécialisé dans les Effets Spéciaux Numériques et travaille à l’international en tant que Directeur Technique.
		\begin{quotation} \emph{Au sein de NaturalPad} \end{quotation}
Tristan occupe le poste de Directeur Artistique.
\end{minipage}
	
		\subsubsection*{Benoit LANGE - Fondateur}
\begin{minipage}[t!]{0.2\linewidth}
\centering
\includegraphics[width=0.8\textwidth]{images/tetocarre/ben}
\end{minipage}
\begin{minipage}[t!]{0.79\linewidth}
Il est diplômé d’un master informatique spécialisé en Web et Intelligence Artificielle. Il a ensuite obtenu son diplôme de docteur en informatique à l’\gls{um2} en Novembre 2012. Ce doctorat avait pour sujet de réaliser une méthode de visualisation de données afin de permettre l’optimisation énergétique pour le bâtiment. 
		\begin{quotation} \emph{Au sein de NaturalPad} \end{quotation}
Benoit propose, étudie et met en œuvre des prototypes d’interactions adaptés à des thérapies.
\end{minipage}

		\subsubsection*{Anthony BARREAU - Employé}
\begin{minipage}[t!]{0.2\linewidth}
\centering
\includegraphics[width=0.8\textwidth]{images/tetocarre/anthony}
\end{minipage}
\begin{minipage}[t!]{0.79\linewidth}
Il est diplômé d’une licence professionnelle en informatique spécialité web et gestion. Il a ensuite débuté son expérience professionnelle au sein de l’équipe SMILE du \gls{lirmm}. Il a rejoint l’équipe pour participer au développement du projet \gls{mojos}. Dans ce projet de recherche sur les jeux thérapeutiques, il a participé au développement de prototypes de jeux et d’agents intelligents servant à adapter la difficulté. 
		\begin{quotation} \emph{Au sein de NaturalPad} \end{quotation}
Anthony participe au prototypage et au développement des outils innovants de l’entreprise.
\end{minipage}

		\subsubsection*{Marion FLORIS - Employée}
\begin{minipage}[t!]{0.2\linewidth}
\centering
\includegraphics[width=0.8\textwidth]{images/tetocarre/marion}
\end{minipage}
\begin{minipage}[t!]{0.79\linewidth}
Titulaire d’un D.U.T. Information-Communication et d’une Licence Professionnelle Management des Ressources Numériques, Marion est issue d ’une formation littéraire. Après avoir été libraire spécialisée en bande dessinée pendant deux ans, elle a développé son expertise en Community Management par une formation de 6 mois chez Objectif3D.
		\begin{quotation} \emph{Au sein de NaturalPad} \end{quotation}
Marion est Community Manager et est chargée de la communication.
\end{minipage}

		\subsubsection*{William DYCE - Stagiaire}
\begin{minipage}[t!]{0.2\linewidth}
\centering
\includegraphics[width=0.8\textwidth]{images/tetocarre/william}
\end{minipage}
\begin{minipage}[t!]{0.79\linewidth}
William effectue son stage de Master 2 \gls{imagina} en même temps que moi chez NaturalPad. L'objectif de son stage était de permettre une reconnaissance des joueurs et de leurs mouvements plus fine avec la Kinect. L'utilisation du Kinect dans un contexte thérapeutique exige en effet une reconnaissance avec des contraintes plus poussées que pour une utilisation purement ludique. Grâce à cela, il sera possible d’imaginer de nouveaux gameplay de jeux basés sur le mouvement.
\end{minipage}

		\subsubsection*{Andy CAMICCI - Stagiaire}
\begin{minipage}[t!]{0.2\linewidth}
\centering
\includegraphics[width=0.8\textwidth]{images/tetocarre/andy}
\end{minipage}
\begin{minipage}[t!]{0.79\linewidth}
Étudiant en licence professionnelle Activité et Techniques de Communication à Arles, Andy a effectué un stage de trois mois de Avril à Juin durant lesquels il a participé au développement de l'interface web thérapeutique et aux modules de visualisation des données de jeux. Nous avons travaillé conjointement afin de fournir un outil ergonomique et efficace pour la paramétrisation des parties des jeux thérapeutiques présents sur la plateforme.
\end{minipage}
		
		\subsubsection*{Kevin BRADSHAW - Stagiaire}
\begin{minipage}[t!]{0.2\linewidth}
\centering
\includegraphics[width=0.8\textwidth]{images/tetocarre/kevin}
\end{minipage}
\begin{minipage}[t!]{0.79\linewidth}
Actuellement en stage, Kevin est en licence Informatique et souhaite devenir developpeur de jeux vidéo. Durant son stage, il conçoit et développe son propre jeu de manière indépendante : Zether. Pour cela, il crée les assets graphiques, le son et les diverses fonctionnalités du jeu. \\
Ce jeu a pour dessein de venir se greffer sur la plateforme web de NaturalPad. Kevin a en effet imaginé un gameplay jouable à la fois avec souris+clavier et par les mouvements du corps avec une caméra Kinect. Utilisant une approche différente de la mienne, l'objectif thérapeutique n'est pas clairement défini bien que pris en compte. La plateforme de NaturalPad veut aussi pouvoir accueillir des jeux qui ne sont initialement pas développés dans une optique de rééducation mais dont une telle utilisation est possible.
\end{minipage}
		
		\subsubsection*{Célia GIRONNET - Stagiaire}
\begin{minipage}[t!]{0.2\linewidth}
\centering
\includegraphics[width=0.8\textwidth]{images/tetocarre/celia}
\end{minipage}
\begin{minipage}[t!]{0.79\linewidth}
Stagiaire infographiste, cette étudiante de SupInfoGame contribue à enrichir le monde de Hammer \& Planks en réalisant les modèles 2D et 3D des personnages, monstres et autres décors de cet univers de pirates !
\end{minipage}

	\subsection{Hammer \& Planks}
Hammer \& Planks est le premier jeu sérieux pour la santé de NaturalPad. Il a été conçu en collaboration avec une ergothérapeute, Anaïs Ivorra, dans le but de permettre aux personnes hémiplégiques de retrouver leur faculté d’équilibre. Il a été présenté à diverses occasions lors de salons, qu’ils soient grand public comme le MIG ou spécialisés (\href{http://www.ted.com/tedx/events/5188}{TEDx Montpellier}\footnote{http://www.ted.com/tedx/events/5188}/ \href{http://www.e-virtuoses.net/}{e-virtuoses}\footnote{http://www.e-virtuoses.net/}). Il a par ailleurs été primé aux e-virtuoses 2013 de Valenciennes dans la catégorie "Serious Game Healthcare" et est largement cité dans un article composé d'une vidéo sur les Serious Games thérapeutiques sur le \href{http://videos.doctissimo.fr/sante/recherche/serious-game-therapeutique.html}{site de doctissimo}\footnote{http://videos.doctissimo.fr/sante/recherche/serious-game-therapeutique.html}.
		
\begin{figure}[htbp]
	\begin{minipage}[c]{.45\linewidth}
		\begin{center}
			\includegraphics[width=210px, height=157px]{images/remise_awards_e-virtuoses_groupe.jpg}
			\caption{Gagnants des e-virtuoses 2013 dans les différentes catégories.}
			\label{Gagnants}
		\end{center}
	\end{minipage}
	\hfill
	\begin{minipage}[c]{.45\linewidth}
		\begin{center}
			\includegraphics[height=157px]{images/doctissimo.png}
			\caption{Article de doctissimo sur les Serious game thérapeutiques.}
			\label{naturalpad_groupe}
		\end{center}
	\end{minipage}
\end{figure}		

Hammer \& Planks est constitué d’un environnement 3D vu de dessus. Le joueur contrôle un bateau qu'il peut déplacer de gauche à droite et de haut en bas, et peut éliminer les ennemis grâce à ses canons. Enfin, il doit éviter des obstacles et peut récupérer des bonus. Pour contrôler le bateau, il existe plusieurs solutions : on peut utiliser la Kinect, la Wii Board, une manette ou bien le clavier et la souris. Le jeu a initialement été conçu pour une utilisation avec la Wii Board afin de travailler les facultés d'équilibre, que ce soit assis ou debout. Le jeu possède par ailleurs un ensemble de paramètres qui peuvent être modifiés en cours de partie, rendant ainsi le jeu ajustable selon les besoins thérapeutiques. On notera qu'il existe une version grand public du jeu, où la paramétrisation avancée est remplacée par un enrichissement du gameplay et du scénario notamment.
	\begin{figure}[!t]
		\centering
		\includegraphics[width=400px]{images/hammer_and_planks.png}
		\caption{Impression d'écran du jeu Hammer \& Planks}
		A droite, le bateau contrôlé par le joueur.
		\label{hammer_and_planks}
	\end{figure}
 
	
	\newpage
	\section{Problématique}
	\subsection{Sujet et objectifs initiaux}
	Dans ce stage, nous allons nous intéresser à l’adaptation de la difficulté dans un jeu à but thérapeutique sous contrôle d’un médecin. Les variables permettant d’ajuster la difficulté d’un jeu sont nombreuses et variées. Elles peuvent être facilement modifiées et combinées pour créer des objectifs de jeu. Cependant, ces objectifs de jeu n’ont pas nécessairement un sens pour le médecin. Il s’agira dans ce stage de définir avec un thérapeute les variables ou paramètres adaptées à l’adaptation de la difficulté pour le soignant.
	\paragraph{}
	Dans le cadre de ce stage, le stagiaire aura pour mission de:
	\begin{itemize}
		\item {récupérer auprès d’un soignant une liste exhaustive d’objectifs de jeu thérapeutique}
		\item {traduire ces objectifs en paramètres dans le jeu Hammer \& Planks}
		\item {proposer une solution pour suivre et analyser visuellement les progrès du patient relativement aux objectifs fixés par le soignant}
	\end{itemize}
	Le stagiaire devra participer à des séances de coconception avec un thérapeute et sera amené à se déplacer pour suivre des séances de tests auprès de patients.
		
\subsection{Contexte et besoins}
Lors de mon arrivée au sein de NaturalPad, il existant déjà une version du jeu Hammer \& Planks, outil dans lequel devait s'insérer mon travail. Présenté lors du MIG 2012, le jeu était encore surtout orienté grand public. Ma première mission fut donc de m'approprier l'application et de l'adapter pour une utilisation paramétrable dans un contexte thérapeutique. En effet, une utilisation thérapeutique implique d'ajuster les différents paramètres en fonction des capacités et des besoins du patient.

	\paragraph{}Les échanges avec les professionnels de la santé ont été au coeur de mon stage. La compréhension des besoins et des contraintes médicales étant primordiales pour proposer un produit adapté, je me suis naturellement tourné vers les thérapeutes et soignants pour acquérir les connaissances et le background qu'il me manquait. Il m'est par ailleurs rapidement apparu qu'on ne pouvait répondre aux différents besoins thérapeutiques explicités par les soignants avec un seul jeu vidéo, même paramétrable. C'est pourquoi je me suis aussi concentré sur l'aspect de conception avec les thérapeutes. 

	\paragraph{Orientation du travail de stage \\}
Rappelons que la société NaturalPad propose comme outil une plateforme web permettant d'accéder à des jeux sérieux, et de les paramétrer directement depuis celle-ci. Hammer \& Planks constitue ainsi le premier jeu accessible depuis cette plateforme et, bien que servant d'exemple des possibilités d'un jeu vidéo pour la santé, il est amené à être rejoint par d'autres serious games. C'est dans cette optique que mon travail durant ce stage s'est progressivement orienté vers une méthode de conception de serious games pour la santé. Cela a pour objectif de proposer une solution appropriée aux différents besoins et contraintes de chaque situation. Ces derniers peuvent correspondre aux objectifs thérapeutiques, à la pathologie du patient, ses capacités, son âge, sa maitrise des nouvelles technologies ou son aisance avec les jeux vidéo par exemple.
Ce travail s'inscrit donc toujours dans le but d'adapter le jeu vidéo aux besoins thérapeutiques, et est complémentaire à une adaptation des paramètres de jeux, qu'elle soit manuelle ou automatique.
 
\paragraph{}Pour cela, je redéfinirais le sujet de ce stage comme suit :\\
\textcolor{marron}{\emph{ {\large Proposition d'une méthodologie de conception de jeux vidéo sérieux à but thérapeutique, et adaptation de la difficulté.}}}

\subsection{Background}
	%%Définition d'un Jeu 
	\subsubsection{Définition d'un Jeu}
Le jeu vidéo est un média participatif en pleine extension et de plus en plus largement accepté et plébiscité par la population. \\Dans sa définition, un jeu vidéo est une extension du jeu au monde numérique en utilisant les technologies informatiques. Il s'agit donc en amont de bien comprendre ce qu'est un jeu. \\Dans son travail d’analyse, Juul\cite{Juul05} donne une synthèse qui regroupe les points partagés par toutes les définitions existantes~:
\begin{quotation}
A game is a rule-based formal system with a variable and quantifiable outcome, where different outcomes are assigned different values, the player exerts effort in order to influence the outcome, the player feels attached to the outcome, and the consequences of the activity are optional and negotiable. 
\end{quotation}
\paragraph{Points clefs \\}
Les 6 points clefs identifiés par Juul sont :
\begin{itemize}
	\item les règles
	\item le résultat quantifiable variable
	\item la valorisation du résultat
	\item l’effort du joueur
	\item l’attachement du joueur au résultat (identifié par Gendler comme Alief, 2009)
	\item des conséquences négociables.
\end{itemize}

		\subsubsection*{Pratique et consommation des jeux vidéo}
Le jeu vidéo est un média récent qui trouve ses origines dans le début de la seconde moitié du XX$^{eme}$ siècle. D'abord majoritairement accessibles sur des bornes d'arcade, les jeux vidéo se sont progressivement répandus avec la commercialisation de consoles de salon. Selon une étude commandée par le Centre National du Cinéma et de l'image animée\href{http://www.cnc.fr}{(CNC)}, c'est près de 60\% de la population française qui jouaient à des jeux vidéo au cours de l'année 2011\cite{Cnc11}. L'âge moyen des joueurs était alors de 34,7 ans et cette population constituée de 54,1\% d'hommes fin 2011\cite{Cnc11}. L'arrivée de la console Wii et de jeux orientés plus casuals\footnote{Se dit d'une personne ne jouant à des jeux vidéo que de manière occasionnelle. Peut aussi se dire d'un jeu dont la cible sont des joueurs occasionnels.} a permi en France une meilleure diffusion du média au sein de la population, qui accepte de plus en plus de se livrer à cette activité. Contrairement à ce que l'on pourrait croire, notamment à cause de leur méconnaissance des nouvelles technologies et du jeu vidéo plus spécifiquement, les seniors se prêtent aussi volontier à la pratique de ce loisir numérique. C'est une constation que nous avons pu vérifier par nous-même lors de nos différentes sessions de tests au centre hospitalier de Lapeyronie et dans une maison de retraite à Lille, mais aussi dans différentes lectures qui seront détaillées ci-après.
	
		\subsubsection*{Propriétés des jeux vidéo}
En 1984, [Driskell and Dwyer]\cite{Dris84}réalisent un premier état de l'art et trouvent que plusieurs caractéristiques des jeux vidéo peuvent influer sur les propriétés d'apprentissage et de motivation : des objectifs spécifiques, un challenge, de la fantaisie et du mystère. Ils théorisent qu'une augmentation de la motivation produit une augmentation de l'attention et mène à une meilleure mémorisation des acquis (connaissance déclarative) et une focalisation de l'attention (stratégie cognitive).\newline
[Malone et Lepper, 1987] mentionnent le challenge, la curiosité, le contrôle et la fantaisie comme caractéristiques intégrantes des jeux vidéo. Selon [de Felix et Johnson , 1993], les jeux sont composés d'éléments visuels dynamiques, d'interactions, de règles et d'objectifs. Puis [Thiagarajan, 1999] affirme que le conflit, le contrôle, la terminalité et l'artifice sont les quatre éléments nécessaires d'un jeu. En 2001, [Garis and Ahlers, 01] donnent 39 descripteurs qui seront réduits à 12 pour ne garder que les paramètres statistiquement les plus significatifs pour renforcer la sensation de "game-like". \\ Finalement en 2002, [Garris et al] proposent un sous-ensemble de tous ces attributs qui seront considérés commes les paramètres de jeu clefs de l'apprentissage~:
\begin{itemize}
	\item la fantaisie
	\item les règles
	\item la stimulation sensorielle
	\item le challenge
	\item le mystère
	\item le contrôle
\end{itemize}

\paragraph{}En 2009, [Wilson et al]\cite{Wils09} partent de l'ensemble de paramètres de [Garris et al] pour enrichir le modèle avec de nouveaux paramètres qui leur semblent avoir un impact sur l'apprentissage.
Les tableaux \ref{game_attributes_one} et \ref{game_attributes_two} indiquent et décrivent ces douze attributs.

\begin{figure}[htbp]
	\centering
	\includegraphics[width=\linewidth, height=\textheight ]{images/game_attributes_one}
	\caption{Propriétés des jeux vidéo et leur définition. (1/2) \cite{Wils09}}
	\label{game_attributes_one}
\end{figure}
\begin{figure}[htbp]
	\centering
	\includegraphics[width=\linewidth, height=\textheight ]{images/game_attributes_two}
	\caption{Propriétés des jeux vidéo et leur définition. (2/2) \cite{Wils09}}
	\label{game_attributes_two}
\end{figure}	
	
	\subsubsection{Jeux sérieux et Théories Comportementales}
Les jeux vidéo se révèlent être un outil dont l’impact peut dépasser la simple portée ludique. Les serious games se proposent de profiter du ressort ludique du jeu vidéo pour servir volontairement un objectif sérieux distinct. Jeux éducationnels, commerciaux, idéologiques ou d’entraînement font partie de cette famille des jeux sérieux. D’un point de vue thérapeutique, il est possible de les utiliser afin de rendre le travail de réhabilitation ou de remise en forme plus motivant pour le patient en combinant les aspects ludique et thérapeutique.

\paragraph{}
Les jeux vidéo sérieux se présentent comme un potentiel médiateur dans la modification des habitudes comportementales en permettant d’inclure des connaissances pratiques dans un modèle ludique apprécié. Il est possible d’y mettre en place des procédures de changement comme l’établissement d’objectifs ou la modélisation et le développement de compétences dans un environnement attrayant, significatif et immersif [Baranowski et al, 2008]\cite{Bara08}. \\
Les jeux vidéo promeuvent les interactions sociales et d’apprentissage [Wideman et all, 2008], créent un environnement où les actions du joueur ont un effet [Gee, 2004]\cite{Gee04} , encouragent la résolution de problèmes [Gee, 2004]\cite{Gee04} et renforcent la compréhension en créant des situations de réflexion ou en aidant le joueur dans ses objectifs [Gee, 2004]\cite{Gee04}. Enfin, les jeux sérieux pour la santé sont fait pour distraire le joueur tout en l’éduquant, en l’entrainant ou en changeant ses comportements [Stokes, 2005]\cite{Stok05}.

		\subsubsection*{Théories comportementales}
Le comportement est la résultante d’influences multiples, rendant ainsi souvent les personnes réfractaires au changement [Baranowski, Lin \& al, 1997]\cite{Bara97}. Le comportement doit alors être considéré comme un mécanisme complexe découlant de l’enchaînement de plusieurs étapes. Ainsi, plutôt que de chercher à impacter directement le comportement, les experts comportementaux valorisent une action sur ces facteurs intermédiaires, appelés médiateurs. Changer ces médiateurs permet de changer le comportement [Baranowski, Lin \& al 1997]\cite{Bara97}.
\paragraph{}Plusieurs grandes théories comportementales existent~:
\begin{itemize}
	\item la théorie d’inoculation comportementale [McGuire, 1961]
	\item la théorie socio-cognitive [Bandura, 1986]
	\item la théorie de l’auto-détermination [Ryan \& Deci, 2000]
	\item la théorie de l’immersion [Green \& Brock, 2000]
\end{itemize}

\paragraph{}De ces théories, l’on peut alors identifier un certains nombre de ces facteurs médiateurs tels que : l’immersion, l’attention, l’auto-régulation, le développement de compétences, la motivation interne et externe, l’autonomie ou encore le sentiment de compétence. La science du comportement fournit aussi des techniques qui facilitent le changement comportemental, et propose d’utiliser ces facteurs dans les média de divertissement tel que le jeu vidéo.\\
Le modèle : "Elaboration Likelihood Model" [Petty \& Cacioppo, 1986] soutient ainsi que des personnages crédibles, attrayants et sympathiques sont plus susceptibles d’être persuasifs que les autres et peuvent donc servir d’intermédiaires pour véhiculer un message. \\
La théorie d’inoculation comportementale [McGuire, 1961] met en garde contre une possible contre-productivité en identifiant et en réfutant les menaces potentielles à l’accomplissement des objectifs du changement désiré.\\
La théorie socio-cognitive [Bandura, 1986] préconise l’établissement d’objectifs et le développement de compétences comme paramètres importants dans le changement comportemental.\\
Enfin, les théories socio-cognitive [Bandura, 1986] et d’auto-détermination [Ryan \& Deci, 2000] mettent toutes deux l’accent sur l’importance du feedback pour guider et mettre en forme le comportement durant le processus de changement.	

	\subsubsection{Méthodologies de conception}
On peut naïvement imaginer deux approches de conception s’opposant dans la conception des jeux sérieux. La première consisterait, de partir des objectifs sérieux et de proposer une “gamification” de ceux-ci en ajoutant des éléments de jeu. La seconde, à l’inverse, serait de partir de la composante ludique du jeu pour y intégrer ensuite le contenu sérieux. Bien qu’ayant l’avantage d’être simples à concevoir, ces deux démarches ont pour limite d’avantager l’une ou l’autre des composantes. Un jeu sérieux conçu à partir d’une base ludique aurait un impact sérieux limité, alors que l’ajout d’une composante ludique à une finalité sérieuse serait peu convaincant.
Une troisième approche est donc de prendre en considération à la fois la composante ludique et l’intention sérieuse dès le début du processus de conception pour les fusionner au mieux. Si elle est correctement mise en place, cette approche promet une forte utilisation du jeu et un impact sérieux efficace. Elle présente néanmoins le défaut de devoir imaginer une nouvelle solution pour tout nouveau couple (jeu, objectif sérieux). C’est ce type d’approche que nous proposons ici, et nous verrons comment les résultats peuvent être réemployables.

	\subsubsection*{Participative design : conception participative}
La conception participative est une méthode de travail utilisée principalement en conception de logiciel interactif. Sa principale caractéristique est la participation active des utilisateurs au travail de conception. Il s'agit donc d'une méthode de conception centrée sur l'utilisateur où l'accent est mis sur le rôle actif des utilisateurs \cite{wiki:cp}.

\paragraph{}
Il existe dans la littérature de nombreuses variantes de la méthode et de nombreuses techniques utilisées pour impliquer efficacement les utilisateurs. On peut noter particulièrement~:
\begin{itemize}
    \item l'observation et entretiens
    \item la production de scénarios
    \item le brainstorming
    \item le prototypage papier
    \item le prototypage vidéo
\end{itemize}

\paragraph{}
Une première séance de conception a lieu en début de projet : celle-ci regroupe les chercheurs et les développeurs de l'application, mais aussi les utilisateurs futurs ou potentiels. Intégrer les utilisateurs au processus de conception permet d'entrevoir au mieux leurs besoins et d'éviter un maximum d'erreurs d'interprétation ou d'oublis (voir illustration figure \ref{projet_info}). Les utilisateurs aident aussi à définir les problèmes éventuels et leurs possibles solutions.
\begin{figure}
	\centering
	\includegraphics[width=16cm]{images/projet_info.jpg}
	\caption{Allégorie d'un projet informatique}
	\label{projet_info}
\end{figure}

\paragraph{}Par ailleurs, plusieurs séances peuvent avoir lieues tout au long du projet, afin de vérifier si les besoins ont évolués et si le développement actuel est effectivement en accord avec ceux-ci. Les utilisateurs aideront aussi l'équipe de recherche et développement à juger de la pertinence des solutions apportées.
\paragraph{}
D'un point de vue du développement, on notera que ce type de conception s'accorde parfaitement avec une méthodologie AGILE, et notamment la méthode SCRUM qui consiste en de courtes périodes de développement entre lesquelles on met en relief l'avancement par rapport au projet global.

	\subsubsection{Jeux sérieux thérapeutiques}
Dans notre contexte, l’aspect sérieux recherché du jeu est la réhabilitation motrice ou la rééducation physique du joueur. Ces objectifs sont indiqués par des thérapeutes, médecins ou kinés, notamment dans le cadre de réhabilitation de personnes hémiplégiques ou souffrant de douleurs lombaires. L’intérêt du jeu vidéo est alors de proposer un environnement de réhabilitation plus agréable et de faciliter l’acceptation des travaux de rééducation par le patient grâce aux éléments de gameplay. Le jeu peut permettre un plus grand volume de travail de la part du patient car celui-ci sera plus enclin à les réaliser dans le cadre du jeu sérieux. Surtout, le jeu sérieux augmente la répétition de mouvements, ce qui améliore l'état physique du joueur-patient. L’objectif à terme est donc d’améliorer les résultats de l’objectif thérapeutique.

	\subsubsection{Contexte socio-médical}
		\paragraph{L'AVC et Hammer \& Planks\\}
Dans les pays occidentaux, près d'un individu sur 600 est victime d'un AVC chaque année, soit près de 120 000 accidents par an en France. L'AVC est une cause fréquente d'hémiplégie chez les victimes et reste une des principales causes d'invalidité. \\
La récupération des fonctions motrices, de la parole ou de la compréhension dépendent pour beaucoup de l'âge du patient et de son atteinte au niveau du cerveau.
\paragraph{}Hammer \& Planks est né d'un projet d'une étudiante en ergothérapie dont le but était de proposer un jeu servant à travailler l'équilibre chez des personnes hémiplégiques ayant été victimes d'un AVC. Aujourd'hui, H\&P peut être utilisé aussi bien pour travailler son équilibre, ses membres supérieurs son tronc ou sa capacité d'attention.

	\paragraph{Récupération de lombalgie\\}
Si la réhabilitation post AVC a été au cœur de mon travail et de celui de NaturalPad, son objectif est de pouvoir proposer ou accéder à des solutions pour divers types de pathologies. Ainsi, un projet de NaturalPad pour lequel j'ai participé à la phase de conception a pour objectif de créer un serious game pour la rééducation de personnes lombalgiques.
\paragraph{}
La lombalgie est un état douloureux du rachis lombaire qui peut être aiguë ou chronique. Les lombalgies affectent une forte proportion de personnes puisque entre 40 et 70\% de la population est touchée à un moment ou un autre. Sous l'effet de la douleur, une majorité des patients va cesser toute activité physique voir même professionnelle. Une de ses conséquences est aussi une démotivation de la personne pouvant aller jusqu'à un état de dépression, notamment dû à l'inactivité et la douleur. Comme préconisé dans le Guide du Dos\cite{backbook}, la reprise et le maintien d'une activité physique sont primordiaux dans le processus de récupération. \\
Le projet de NaturalPad est une application de coaching sportif adaptée à ce besoin et proposant un certain nombre d'exercices physiques gamifiés afin d'encourager la reprise d'activité des utilisateurs.	


\input{chapters/background_rehabilitation}

	\subsubsection{Système de recommandation }
	
Les systèmes de recommandation représentent les préférences de l'utilisateur dans le but de proposer des articles à acheter ou à examiner notamment. Dans notre problématique, un système de recommandation pourrait servir à sélectionner les paramètres de jeux, voir le jeu lui même, qui correspondraient le mieux aux besoins du joueur. Rappelons que ces besoins peuvent être soit explicites, notamment à travers les recommandations et exigences du thérapeute, soit plus inconscients. Ces besoins inconscients représentent pas exemple les préférences du joueur-patient en terme de gameplay. Un jeu plus distrayant et motivant pour le patient renforcera son implication dans le programme de réhabilitation, et donc son rétablissement. Pour cela il faut donc à la fois connaître les préférences du patient, explicites ou `découvertes'  grâce à un système d'apprentissage par exemple, mais aussi s'appuyer sur un certain nombre de théories et connaissances que l'on sait efficaces pour renforcer cette immersion. 	
	 
 \paragraph{}
 La proposition est ici de s'inspirer du monde la musique (ou des livres, des films, ou encore des ventes en ligne) et de son système de recommandation.\\
 On pense rapidement à deux types de recommandations. La recommandation sociale, qui consiste par exemple à conseiller à un utilisateur des musiques qu'apprécient des personnes de son réseau, surtout si elles écoutent généralement des musiques identiques. Un autre exemple sur les sites de vente en ligne, où l'on propose à un utilisateur venant d'acheter un objet, une liste de produits ayant été achetés en même temps par d'autres utilisateurs. \\
Le second type de recommandation se base pas non pas sur l'environnement social de l'utilisateur, mais sur le contenu même des objets recommandés. L'idée est alors de chercher à décrire un objet selon certaines caractéristiques, et à faire de même pour les préférences de l'utilisateur. On va ensuite lui conseiller les objets qui semblent être le plus proche des attentes de l'utilisateur en se basant sur ces critères de préférences. 
 
\paragraph{}
Pour Vincent Castaignet, fondateur et directeur de la publication de Musicovery, au-delà de la recommandation éditoriale, il y a différentes manières de proposer des artistes/titres par similarité~:
\begin{enumerate}
	\item d’après les formes musicales sur lesquelles le goût des auditeurs est fondé.
	\item d’après les repères mentaux utilisés par les auditeurs (genres, sous-genres, style).
	\item social  : si tu es membre de cette tribu et aimes cet artiste, alors tu vas aimer ces artistes.
	\item contextuel  : ceux qui écoutent ce titre dans ce contexte écoutent aussi ces titres.
\end{enumerate}
Les passionnés de musique qui cherchent activement préféreront de la similarité type (2), le grand public plus passif de la similarité type (4). Un moteur de similarité intelligent devrait pouvoir combiner ces différentes formes et s’adapter en fonction du profil de chacun.
Pandora est principalement construit sur (1), last.fm est (2) et (3), Musicovery (4). Deezer avec ces 30 millions de playlists a un actif considérable à exploiter en (3) et (4).

 		\subsubsection*{Les différents types de recommandation}
Plusieurs techniques de recommandation ont été proposées : basées sur le contenu, sur des connaissances ou encore des techniques dîtes collaboratives ou sociales. Pour de meilleurs résultats, certaines de ces techniques peuvent être utilisées conjointement dans des systèmes de recommandation hybrides.

	\paragraph{\emph{Propriétés des systèmes de recommandation} \\ \quad}
Les systèmes de recommandation possèdent :
\begin{itemize}
	\item des données de base : données que le système possède avant même de commencer la recommandation
	\item des données d’entrée : données que l’utilisateur fournit au système dans le but que ce dernier lui fournissent des recommandations.
	\item un algorithme qui utilise ces données de base et d’entrée pour générer les résultats.
\end{itemize}

	\paragraph{}
\begin{figure}[hbtp]
	\centering
	On peut distinguer 5 types de systèmes de recommandation
	\includegraphics[width=1\linewidth]{images/types_recommandation.png}
	\caption{Techniques de recommandations [R. Burke, 2002] \cite{Burk02} }
	\label{types_recommandation}
\end{figure}    
avec :
\begin{itemize}
	\item I : ensemble d’objets sur lequel sont faites les recommandations
	\item U : l’ensemble d’utilisateurs dont les préférences sont connues
	\item u : l’utilisateur pour lequel les recommandations doivent être générées
	\item i : objets pour lesquels on souhaiterait prédire une préférence de la part de u
\end{itemize} 

		\paragraph{\emph{Collaborative} \\ \quad}
Méthode la plus mature et répandue. Le système agrège les notes ou recommandations des objets, relève les similarités entre les appréciations des utilisateurs et en déduit de nouvelles recommandations pour les utilisateurs. Certains systèmes prennent le temps en paramètre dans leur évaluation afin de prendre en compte l’évolution de l’intérêt des utilisateurs au fil du temps (effet de mode, etc). L’évaluation peut être simplement binaire ou plus complexe en utilisant une échelle de graduation. Les systèmes peuvent être soit basés sur une mémoire, comparant les utilisateurs par corrélation ou autre, soit basés sur un modèle :  celui-ci est dérivé à partir de l'historique des données d'évaluation et utilisé pour faire les prédictions.
La plus grande force de ces techniques est qu’elles sont complètement indépendantes de la représentation informatique des objets recommandés.

		\paragraph{\emph{Démographique} \\ \quad}
Ces systèmes de recommandation ont pour but de catégoriser l’utilisateur à partir de ses caractéristiques propres et de faire des recommandations en fonction de son appartenance à l’une des classes démographiques prédéfinies. L’avantage d’une approche démographique est qu’elle ne requiert pas un historique des évaluations des utilisateurs à l’inverse des méthodes collaboratives et basées sur le contenu.

		\paragraph{\emph{Basée sur le contenu} \\ \quad}
 La recommandation basée sur le contenu est une excroissance et la poursuite de la recherche d'information de filtrage. Dans ces systèmes, les objets sont définis en fonction de leurs caractéristiques associées. Le système apprend à connaître le profil de l’utilisateur en se basant sur les caractéristiques des objets évalués par l’utilisateur. C’est une corrélation objet-à-objet. Le profil dérivé dépend évidemment du type d’apprentissage employé : arbre de décisions, réseaux de neurones et représentations par vecteurs sont utilisés.

		\paragraph{\emph{Fondée sur l’utilité} \\ \quad}
Ces systèmes font des suggestions en se basant sur une estimation de l’utilité de chaque objet pour l’utilisateur. Le problème central étant comment créer cette fonction d’utilité pour chaque utilisateur. D’abord évaluer les objets (différentes méthodes), puis le profil utilisateur, avant de calculer la correspondance entre les deux. L’avantage de la technique est qu’elle peut prendre en compte des attributs non directement propres aux objets évalués (fiabilité du vendeur, disponibilité du produit, etc.) pour proposer des recommandations plus pertinentes (besoin immédiat ou meilleur prix par ex.).

		\paragraph{\emph{Basée sur le savoir/connaissance} \\ \quad}
Tenter de recommander des objets en inférant les besoins et préférences de l’utilisateur. Ces systèmes se distinguent en ce qu’ils ont un savoir fonctionnel : ils ont connaissance que tel objet répond à tel besoin et peuvent alors abstraire la relation entre le besoin et une possible recommandation.

	\paragraph{}
Les systèmes de recommandations basés sur l’utilité et sur le savoir n’essaient pas de construire des généralisations à long terme à propos de leurs utilisateurs, mais préfèrent baser leurs conseils sur une évaluation de la correspondance entre les besoins d’un utilisateur et un ensemble d’options disponibles.

\subsection{Outils et méthodologie}
	\subsubsection{Méthodologie}
Afin de mener à bien ses projets, l’équipe de NaturalPad emploie une méthode Agile de gestion de projet : SCRUM.
Celle-ci définit 3 rôles :
	\begin{itemize}
		\item Le Product Owner
		\item Le Scrum Master
		\item Le Développeur
	\end {itemize}
Le Product Owner est le représentant des clients et des utilisateurs. Son objectif est de maximiser la valeur du produit développé. Il a pour rôle de rédiger des User Stories (comparables à des cas d'utilisation) et de valider le travail des développeurs. 
\\Le ScrumMaster est le responsable de la méthode. Il doit s’assurer qu’elle est correctement mise en application et comprise par les développeurs. Il organise le «Daily Scrum» (voir définition plus bas).
\\Enfin, le Développeur, représente en fait une équipe pluridisciplinaire et auto-organisée : toutes les décisions sont prises ensemble, sans hiérarchie externe ni interne.
 
		\paragraph{Daily Scrum :}
Il s’agit d’une réunion quotidienne ayant pour but de faire un point sur la coordination entre les tâches et les difficultés rencontrées.  Trois questions sont posées aux développeurs : 
	\begin{itemize}
		\item Qu’as-tu fait hier ?
		\item Qu’est-ce que tu vas faire aujourd’hui ?
		\item Est-ce que tu as rencontré des difficultés ?
	\end {itemize}
	
\paragraph{}Le travail est organisé sous forme de sprint. Il s’agit d’une courte période (au maximum un mois) au bout de laquelle l’équipe doit fournir une version améliorée du produit. Chaque sprint possède un but (ex : «on doit pouvoir envoyer des paramètres au jeu») et une liste de tâches (ex : «déterminer la méthode de communication, etc...»). Dès la fin d’un sprint, un nouveau est lancé.

\paragraph{}Enfin, une réunion a lieu en fin de sprint pour faire le point sur le travail accompli, les erreurs rencontrées et comment ne pas les éviter à l'avenir, ainsi que lancer le sprint suivant. Cette méthode est très intéressante car elle permet vraiment de garder une cohésion dans l’équipe de développement et d’avancer de manière visible. 

	\subsubsection{Outils}
		\paragraph{Gestion de projet\\}
Lors de mon arrivée dans l'entreprise, l’équipe utilisait Redmine, une application web de gestion de projets. Nous avons cependant changé deux fois d'outils de gestion de projet pendant la période de ce stage. Le premier est intervenu car les mises à jour des tâches dans Redmine étaient longues et l'outil finalement peu approprié à une méthodologie AGILE, ce qui freinait son utilisation. Nous avons donc mis en place une méthode Kanban qui consiste à écrire chaque tâche sur un post-it, et de déplacer ce post-it dans des colonnes «A faire», «En cours», «Terminé» ou «Validé» selon son avancement par exemple. De cette manière, l’avancement global était bien plus visible mais cette solution était finalement gourmande en post-it et en place. C'est pourquoi, nous utilisons désormais \href{www.trello.com}{Trello}, un outil de gestion de projet en ligne, se basant sur la méthode Kanban. Il s’agit d’un tableau virtuel dans lequel nous pouvons facilement déplacer les tâches, ajouter des commentaires ou des contraintes de temps notamment.

	\begin{figure}[!h]
		\centering
		\includegraphics[height=48px]{images/redmine.jpg}
		\includegraphics[height=48px]{images/trello.jpg}
		\caption{Logos de Redmine et Trello}
		\label{Logos de Redmine et Trello}
	\end{figure}

		\paragraph{Développement}
		\subparagraph{} \emph{Unity3D\\}
La majeure partie technique de mon travail a été réalisée pour Hammer \& Planks, qui est développé avec le moteur de jeu Unity3D. Au fil de mon stage, nous sommes passés de la version 3.9 à la version 4.1. Unity permet de facilement intégrer les modèles 3D des objets réalisés dans les logiciels de modélisation 3D tels que Photoshop, Gimp ou Maya. Il propose aussi des options permettant d'utiliser un gestionnaire de versions pour les fichiers du projet.
	\begin{figure}[!h]
		\centering
		\includegraphics[height=48px]{images/unity.jpg}
		\caption{Logo d'Unity3d}
		\label{Logo d'Unity3d}
	\end{figure}

		\subparagraph{} \emph{Git\\}
Que ce soit pour Hammer \& Planks ou nos autres projets en cours, l'utilisation d'un gestionnaire de versions se révèle vite indispensable. Travaillant en équipe allant jusqu'à cinq développeurs et une graphiste, il est nécessaire de pouvoir mutualiser le travail. De plus, l'expérimentation et le développement de nouveaux éléments se prêtent très bien à l'utilisation de plusieurs branches de développement, chose que Git permet de gérer facilement.
	\begin{figure}[!h]
		\centering
		\includegraphics[height=48px]{images/git.png}
		\caption{Logo de Git}
		\label{Logo de Git}
	\end{figure}

		\subparagraph{}	\emph{BitBucket et GitHub}
Pour héberger ses projets, NaturalPad avait l'habitude d'utiliser GitHub. Avec l'arrivée de nouveaux stagiaires, il nous a fallu trouver une solution permettant un accès privé au dépôt pour un plus grand nombre de personnes, ce que permet BitBucket.
	\begin{figure}[!h]
		\centering
		\includegraphics[height=48px]{images/bitbucket.jpg}
		\includegraphics[height=48px]{images/github.jpg}
		\caption{Logos de BitBucket et Github}
		\label{Logos de BitBucket et Github}
	\end{figure}

	\subsubsection{Veille}
Le Jeu Vidéo et plus généralement l'Informatique est un domaine en constante évolution dans lequel il est nécessaire de se tenir à jour pour connaître les dernières technologies et actualités. Pour cela, j'ai observé durant l'intégralité de ma période de stage une veille technologique et stratégique. Nouveautés technologiques, logiques ou matérielles, communications d'entreprises ou de salons nationaux et internationaux ou bien encore annonces de sociétés dont le secteur d'activité est compatible avec NaturalPad ont donc été au coeur de mon étude quotidienne.
\paragraph{}Pour faciliter ce travail de veille, par ailleurs inclu dans mon planning, j'utilise un agrégateur de flux RSS, outil indispensable pour gérer aisément un contenu important sur un grand nombre de sources différentes. Il s'agit ensuite de mettre à jour et d'étendre régulièrement les sources en fonction de l'utilité observée de chacune d'entre elle ou des manques ressentis.
\paragraph{Jeux Vidéo\\ \quad}
Étant étudiant en Informatique, option Image Game and Intelligent Agents, et ayant orienté ma formation vers une spécialité Jeux Vidéo, il m'a semblé important de me tenir à jour en terme d'actualité vidéoludique. J'ai pour cela étendu ma veille aux domaines des jeux vidéo, indépendants ou blockbusters, afin d'en étudier différents aspects tels le business model, le gameplay, les technologies employées ou les mécanismes de jeu innovants par exemple. J'ai ainsi pu testé des technologies récentes comme la console Ouya ou le système de contrôle Leap Motion, qui permet d'interagir en utilisant ses mains et ses doigts. Pour plus d'informations sur le Leap Motion, vous pouvez retrouver mon billet sur le blog de \href{naturalpad.fr/category/naturalblog}{NaturalPad}.
\begin{figure}
	\centering
	\includegraphics{images/leap_motion.jpg}
	\caption{Utilisation d'un Leap Motion}Une application retranscrit à l'écran la "vision" qu'elle a des mains de l'utilisateur.
	\label{leapmotion}
\end{figure}
	
	\newpage
	\section{Background}
	Si les jeux vidéo sérieux commencent à être connus du grand public et leur impact reconnu, ceux pour la santé ne sont pas encore assez nombreux ni assez largement acceptés. Mon travail de stage s'inscrit en réponse à cette constatation : il s'agit de proposer une méthode de conception de jeux vidéo pour la santé qui permettrait une simplification ou une amélioration de la conception de tels jeux. Proposer plus de jeux  thérapeutiques et/ou des jeux avec un impact santé de plus grande qualité contribuerait à améliorer leur diffusion et leur reconnaissance. Pour ces raisons, il était ainsi primordial de parfaire ma connaissance des différents domaines concernés.

Cette partie présente le résultat de mes recherches, des techniques et outils existants et comporte des notions importantes à connaître pour la conception de serious games pour la santé.

%****************** Subsection 1 : jeux vidéo
NaturalPad est une société innovante dont le domaine d'activité est encore à ses débuts. Si les jeux vidéo sérieux commencent à être connus du grand public et leur impact reconnu, ceux pour la santé ne sont pas encore assez nombreux ni assez largement acceptés. Mon travail de stage s'inscrit en réponse à cette constatation : il s'agit de proposer une méthode de conception de jeux vidéo pour la santé qui permettrait une simplification ou une amélioration de la conception de tels jeux. Proposer plus de jeux  thérapeutiques et/ou des jeux avec un impact santé de plus grande qualité contribuerait à améliorer leur diffusion et leur reconnaissance. Pour ces raisons, il était ainsi primordial de parfaire ma connaissance des différents domaines concernés.

Cette partie présente le résultat de mes recherches, des techniques et outils existants et comporte des notions importantes à connaître pour la conception de serious games pour la santé.

\subsection*{Méthode de travail}
Étendue sur plusieurs semaines, la réalisation de l'état de l'art était pour moi quelque chose de nouveau qui a nécessité une certaine organisation. Deux moments ont été importants : la phase de démarrage et l'arrêt des recherches. Commencer ces recherches alors que les sujets que je devais ou voulais couvrir étaient vastes et non clairement définis fut à la fois plaisant et compliqué. Bien que beaucoup de choses me semblaient intéressantes, la question était de savoir par où commencer. De la même manière, au fil de mes lectures, je découvrais de nouveaux liens et références présentant de nouveaux aspects qui eux-même renvoyaient vers d'autres solutions et articles. La difficulté était donc de juger quand mes connaissances sur un thème donné étaient suffisantes pour éviter de poursuivre d'interminables recherches, aussi intéressantes puissent elles être, le temps étant limité dans le cadre d'un stage de Master.
	\paragraph{}Au niveau des lectures abordées, ma source principale fut des articles scientifiques, suivie par des articles de magazines spécialisés et articles web notamment. Pour les thèmes médicaux, un certain nombre d'articles m'a été directement conseillé par des professionnels de la santé, articles à partir desquels j'ai ensuite pu compléter mon état de l'art en suivant les références. Par ailleurs, il est fréquent que certains auteurs ressortent régulièrement lorsqu'on effectue des recherches sur un thème donné, ce qui permet, en plus du nombre de citations des articles, de rapidement cerner quels sont les articles et chercheurs de référence dans le domaine.
	\paragraph{}Bien entendu, afin de rendre mes lectures efficaces, je me constituais pour chacune d'elle une fiche de lecture où noter les points importants : 
\begin{itemize}
	\item Titre ou source
	\item Auteur(s)
	\item Mots clefs
	\item Synthèse
	\item Jugement personnel ou remarques
	\item Références importantes
\end{itemize}

%% PART 1 : JEUX VIDEO
\subsection{Jeux Vidéo}
Durant mon stage, j'ai ainsi voulu comprendre pourquoi et comment un jeu vidéo est bon, quels en sont les mécanismes ou bien encore ce qu'est la difficulté, pourquoi et comment l'adapter.
	%%Définition d'un Jeu 
	\subsubsection{Définition d'un Jeu}
Le jeu vidéo est un média participatif en pleine extension et de plus en plus largement accepté et plébiscité par la population. \\Dans sa définition, un jeu vidéo est une extension du jeu au monde numérique en utilisant les technologies informatiques. Il s'agit donc en amont de bien comprendre ce qu'est un jeu. \\Dans son travail d’analyse, Juul\cite{Juul05} donne une synthèse qui regroupe les points partagés par toutes les définitions existantes~:
\begin{quotation}
A game is a rule-based formal system with a variable and quantifiable outcome, where different outcomes are assigned different values, the player exerts effort in order to influence the outcome, the player feels attached to the outcome, and the consequences of the activity are optional and negotiable. 
\end{quotation}
\paragraph{Points clefs \\}
Les 6 points clefs identifiés par Juul sont :
\begin{itemize}
	\item les règles
	\item le résultat quantifiable variable
	\item la valorisation du résultat
	\item l’effort du joueur
	\item l’attachement du joueur au résultat (identifié par Gendler comme Alief, 2009)
	\item des conséquences négociables.
\end{itemize}

	\subsubsection*{Population de joueurs}
Le jeu vidéo est un média récent qui trouve ses origines dans le début de la seconde moitié du XX$^{eme}$ siècle. A l'origine uniquement accessibles sur des bornes d'arcade, les jeux vidéo se sont progressivement répandus avec la sortie de consoles de salon. Selon une étude commandée par le Centre National du Cinéma et de l'image animée\href{http://www.cnc.fr}{(CNC)}, c'est près de 60\% de la population française qui jouaient à des jeux vidéo au cours de l'année 2011\cite{Cnc11}. L'âge moyen des joueurs était alors de 34,7 ans et cette population constituée de 54,1\% d'hommes fin 2011\cite{Cnc11}. L'arrivée de la console Wii et de jeux orientés plus casuals\footnote{Se dit d'une personne ne jouant à des jeux vidéo que de manière occasionnelle. Peut aussi se dire d'un jeu dont la cible sont des joueurs occasionnels.} a permi en France une meilleure diffusion du média au sein de la population, qui accepte de plus en plus de se livrer à cette activité. Contrairement à ce que l'on pourrait croire, notamment à cause de leur méconnaissance des nouvelles technologies et du jeu vidéo plus spécifiquement, les seniors se prêtent aussi volontier à la pratique du jeu vidéo. C'est une constation que nous avons pu vérifier par nous-même lors de nos différentes sessions de tests au centre hospitalier d'Alès et dans une maison de retraite à Lille, mais aussi dans différentes lectures qui seront détaillées ci-après.
	
	%% Jeux vidéo et schéma comportementaux
	\subsubsection{Jeux vidéo et schémas comportementaux}
Pour expliquer l'attrait croissant de la population envers les jeux vidéo, il peut être intéressant de se tourner du coté des théories comportementales. Dans son article sur le lien entre le conditionnement comportemental et les jeux vidéo, Carl Rocray \cite{Rocr09} explique qu'en dépit de ses origines technologiques récentes, le jeu vidéo entretient d’anciens schémas de comportements, et cette influence est subtile. C’est le propre du jeu de nous divertir, non seulement de nos tracas réels, mais aussi de son influence sur nous pendant que nous jouons. Cette influence est évidente à travers le lien entre plaisir et apprentissage : l’effort diminue si on a du plaisir à le faire, et on peut même en oublier que nous apprenons.

\paragraph{}On se rend compte que la plupart des jeux vidéo peuvent être classés selon leur gameplay en ce que l'on appelle des types de jeux. Parmi les grands types connus, on retrouve par exemple les jeux d'action, les jeux de stratégie ou encore les jeux de simulation sportive. On trouvera en \href{types_jeux}{annexe} une liste plus complète de ces types de jeux vidéo classés en fonction de leur gameplay. \\

Il est intéressant de constater le lien entre les différents schémas comportementaux basiques et les types de jeux mettant en place des mécaniques stimulant nos instincts primaires.

\paragraph{}
		\paragraph{Survie \\ \quad}
«Éliminer ou être éliminé». C’est la base de la majorité des jeux vidéo et probablement aussi le comportement le plus simple. Même les jeux de sports ou de cartes entretiennent cet instinct primaire, bien qu’il soit surtout manifeste dans les «First Person Shooter» (Doom, Bioshock), les jeux de combats (Tekken, Street Fighter) et les «platformers» (Mario Bros, Assassin’s Creed).
		\paragraph{Gestion \\ \quad}
Il s’agit surtout de gestion de ressources, mais aussi d’équipement quand celui-ci est limité. Une bonne gestion permet habituellement de mieux survivre (meilleures ressources, équipement adéquat), ce qui nécessite une planification et une organisation stratégiques. On retrouve ce comportement dans les jeux de simulation (SimCity, Civilization), de «Real Time Strategy» (Warcraft, Spacecraft) et ceux qui visent un certain réalisme (équipement limité dans Resident Evil).
		\paragraph{Responsabilisation d’autrui \\ \quad}
Ce comportement fait appel à une sorte d’instinct parental où nous devons assurer la survie d’un autre joueur ou le bien-être d’un avatar virtuel qui nous ramène en fait à nous-même. Il est d’abord présent dans les jeux de simulation (The Sims, Tamagotchi) et ceux de coopération (Army of Two, Left 4 Dead).
		\paragraph{Résolution de puzzles \\ \quad}
Mystères à résoudre (Myst), problèmes de logique faisant appel à nos capacités cognitives (Brain Challenge), voire comment empiler des formes (Tetris) ou aligner des couleurs (Bejeweled), il s’agit toujours de trouver la solution la plus efficace à un problème donné.
		\paragraph{Réflexes et dextérité \\ \quad}
La plupart des jeux font appel à cet instinct également lié à la survie.
Surtout lié à la coordination «oeil-main» et à la précision d’actions souvent rapides (voir entre autres les «Quick-Time Events»), il est aussi lié à la maîtrise d’un outil, c’est-à-dire la manette de jeu. On le retrouve dans les jeux de rythme (Guitar Hero, Rock Band), mais aussi dans les jeux de Survie cités plus haut et tous les jeux de sports.
		\paragraph{Système de récompenses \\ \quad}
La grande majorité des jeux vidéo utilise ce système de la carotte au bout du bâton pour diriger les actions du joueur. Il est donc étroitement lié à la motivation et à la gratification de comportements donnés. Voir entre autres Little Big Planet et Diablo II, ainsi que la mode assez récente des Trophées et des «Achievements».
		\paragraph{Système d’améliorations \\ \quad}
Une autre mécanique liée à la Survie et à la courbe de difficulté des jeux. En bref, il s’agit de devenir plus fort pour défaire des adversaires toujours plus forts. C’est la base des «Role-Playing Games» (World of Warcraft, Oblivion) et de certains jeux de «Shooter» (pour améliorer ses armes).

		\paragraph{}
Il est probable que d’autres comportements soient ainsi entretenus par le médium vidéo-ludique.
Les sept types de mécaniques décrites ci-haut démontrent néanmoins comment les jeux vidéo nous
font répéter d’anciens schémas de comportement.

	%%State of Flow
	\subsubsection{L'état de flux ou expérience optimale}
Le state of flow ou état de flux, est l'état dans lequel une personne peut se trouver, proche de l’extase (dans le sens “se trouver à coté de”) complètement immergée dans ce qu'elle fait, dans un état maximal de concentration. Cette personne éprouve alors un sentiment d'engagement total et de réussite : on est alors hyper compétent, naturellement, inconsciemment dans ce qu’on est en train de faire : musique, sport, travail, jeux, lecture, etc. Cet état est commun chez les personnes exerçant une activité répétitive et/ou demandant de gros efforts physiques ou psychiques On est comme détaché de son corps, que l’on voit agir tout seul. On est alors capable d’accomplir des choses qu’on serait incapable de réaliser consciemment de manière contrôlée. Cet état (s’il n’est pas interrompu) peut durer une dizaine ou une vingtaine de minutes.
	
\begin{figure}[h!]
	\centering
	\includegraphics[width=9cm]{images/state_of_flow.png}
	\caption{Positionnement entre les humeurs et l'état de flux}
	\label{state_of_flow}
\end{figure}

\paragraph{}
L’état de flux est l’une des raisons pour lesquelles on joue aux vidéo, dont le but est de divertir en jouant sur la motivation du joueur. Le jeu, au moyen d’une balance entre compétences et challenge, maintient la vivacité d’esprit du joueur, avec une motivation importante et une attention forte. Être dans cet état de flux permet donc au joueur une meilleure expérience de jeu augmentant son ressenti et son souhait de continuer à jouer. Pour rentrer dans cet état de grâce, les deux états d’esprit les plus proches sont l’excitation et le contrôle, le flow state se trouvant à l’intersection de ces deux noeuds. On peut donc essayer de mettre le joueur dans ces états là si on cherche à stimuler ce flow state.

\paragraph{}
Csikszentmihalyi identifie les caractéristiques de l’état de flux~:
\begin{enumerate}
   \item prédispositions (caractéristiques propices) pour atteindre cet état~:
         \begin{itemize}
            \item des objectifs clairs et précis
            \item équilibre entre difficulté de la tâche et compétences de l’acteur (joueur)
            \item l’activité est en soi une source de satisfaction (amusante ou pour laquelle le joueur est impliquée)
         \end{itemize}
   \item conséquences et caractéristiques~:
   	\begin{itemize}
            \item hyperfocus, concentration exacerbée sur une action précise
            \item perte de la conscience de soi
            \item perception du temps modifiée
            \item rétroaction immédiate : prise de conscience de l’action effectuée, pour ajuster les suivantes
            \item sentiment de contrôle de la situation
	\end{itemize}
\end{enumerate}


%****************** Subsection 2 : Serious games et Theories comportementales
	\subsection{Jeux sérieux}
Les jeux vidéo se révèlent être un outil dont l’impact peut dépasser la simple portée ludique. 
		\subsubsection*{Serious games, serious gaming et apprentissage tangentiel}
		\label{sggt}
On notera bien la différence entre les serious games conçus spécifiquement avec un objectif sérieux de deux autres formes d'impact sérieux que peuvent revêtir les jeux vidéo. 
\paragraph{•}\emph{Les serious games} se proposent de profiter du ressort ludique du jeu vidéo pour servir volontairement un objectif sérieux distinct. Jeux éducationnels, commerciaux, idéologiques ou d’entraînement font partie de cette famille des jeux sérieux. D’un point de vue thérapeutique, il est possible de les utiliser afin de rendre le travail de réhabilitation ou de remise en forme plus motivant pour le patient en combinant les aspects ludique et thérapeutique.

\paragraph{•}\emph{Le serious gaming} est la dérive de l'utilisation d'un jeu vidéo classique ludique dans un objectif sérieux distinct et non prévu dans la conception du jeu. Il est par exemple possible de jouer à un jeu vidéo permettant de conduire des véhicules dans une ville (les séries \emph{GTA} ou \emph{Driver} par exemple) dans le but de réviser son code de la route. C'est une forme de gameplay émergeant, c'est-à-dire une situation complexe rendue possible par les mécanismes de base mis en place dans le jeu, issu de l'appropriation de celui-ci par les joueurs.

\paragraph{•}\emph{L'apprentissage tangentiel} représente les connaissances ou compétences acquises par le joueur lors de ces sessions de jeu de manière non volontaire et non directement prévues par les concepteurs du jeu. Cela peut être le fait de développer ses connaissances linguistiques en jouant à un jeu non traduit, de développer son vocabulaire ou d'acquérir des connaissances dans un jeu ayant un thème particulier (vocabulaire militaire ou d'arsenal dans un jeu de guerre, connaissances théologiques dans un jeu sur l'Égypte ou la Grèce antique par exemple).

		\subsubsection{Serious games et théories comportementales}
Les jeux vidéo sérieux se présentent comme un potentiel médiateur dans la modification des habitudes comportementales en permettant d’inclure des connaissances pratiques dans un modèle ludique apprécié. Il est possible d’y mettre en place des procédures de changement comme l’établissement d’objectifs ou la modélisation et le développement de compétences dans un environnement attrayant, significatif et immersif [Baranowski et al, 2008]\cite{Bara08}. \\
Les jeux vidéo promeuvent les interactions sociales et d’apprentissage [Wideman et all, 2008], créent un environnement où les actions du joueur ont un effet [Gee, 2004]\cite{Gee04} , encouragent la résolution de problèmes [Gee, 2004]\cite{Gee04} et renforcent la compréhension en créant des situations de réflexion ou en aidant le joueur dans ses objectifs [Gee, 2004]\cite{Gee04}. Enfin, les jeux sérieux pour la santé sont fait pour distraire le joueur tout en l’éduquant, en l’entrainant ou en changeant ses comportements [Stokes, 2005]\cite{Stok05}.

		\subsubsection*{Théories comportementales}
Le comportement est la résultante d’influences multiples, rendant ainsi souvent les personnes réfractaires au changement [Baranowski, Lin \& al, 1997]\cite{Bara97}. Le comportement doit alors être considéré comme un mécanisme complexe découlant de l’enchaînement de plusieurs étapes. Ainsi, plutôt que de chercher à impacter directement le comportement, les experts comportementaux valorisent une action sur ces facteurs intermédiaires, appelés médiateurs. Changer ces médiateurs permet de changer le comportement [Baranowski, Lin \& al 1997]\cite{Bara97}.
\paragraph{}Plusieurs grandes théories comportementales existent~:
\begin{itemize}
	\item la théorie d’inoculation comportementale [McGuire, 1961]
	\item la théorie socio-cognitive [Bandura, 1986]
	\item la théorie de l’auto-détermination [Ryan \& Deci, 2000]
	\item la théorie de l’immersion [Green \& Brock, 2000]
\end{itemize}

\paragraph{}De ces théories, l’on peut alors identifier un certain nombre de ces facteurs médiateurs tels que : l’immersion, l’attention, l’auto-régulation, le développement de compétences, la motivation interne et externe, l’autonomie ou encore le sentiment de compétence. La science du comportement fournit aussi des techniques qui facilitent le changement comportemental, et propose d’utiliser ces facteurs dans les média de divertissement tel que le jeu vidéo.\\
Le modèle : "Elaboration Likelihood Model" [Petty \& Cacioppo, 1986] soutient ainsi que des personnages crédibles, attrayants et sympathiques sont plus susceptibles d’être persuasifs que les autres et peuvent donc servir d’intermédiaires pour véhiculer un message. \\
La théorie d’inoculation comportementale [McGuire, 1961] met en garde contre une possible contre-productivité en identifiant et en réfutant les menaces potentielles à l’accomplissement des objectifs du changement désiré.\\
La théorie socio-cognitive [Bandura, 1986] préconise l’établissement d’objectifs et le développement de compétences comme paramètres importants dans le changement comportemental.\\
Enfin, les théories socio-cognitive [Bandura, 1986] et d’auto-détermination [Ryan \& Deci, 2000] mettent toutes deux l’accent sur l’importance du feedback pour guider et mettre en forme le comportement durant le processus de changement.	

	\subsubsection{Jeux sérieux thérapeutiques}
Dans notre contexte, l’aspect sérieux recherché du jeu est la réhabilitation motrice ou la rééducation physique du joueur. Ces objectifs sont indiqués par des thérapeutes, médecins ou kinés, notamment dans le cadre de réhabilitation de personnes hémiplégiques ou souffrant de douleurs lombaires. L’intérêt du jeu vidéo est alors de proposer un environnement de réhabilitation plus agréable et de faciliter l’acceptation des travaux de rééducation par le patient grâce aux éléments de gameplay. Le jeu peut permettre un plus grand volume de travail de la part du patient car celui-ci sera plus enclin à les réaliser dans le cadre du jeu sérieux. Surtout, le jeu sérieux augmente la répétition de mouvements, ce qui améliore l'état physique du joueur-patient. L’objectif à terme est donc d’améliorer les résultats de l’objectif thérapeutique.


%****************** Subsection 3 : Difficulté
	%%PART 3 : LA DIFFICULTE
\subsection{Difficulté}
Dans son livre \emph{La cigale : jeux, vie et utopie}, le philosophe Bernard SUITS indiquait : \begin{quote}{“Jouer consiste à tenter volontairement de surmonter des obstacles inutiles”}.  \end{quote}
		
	\subsubsection{Définition et propriétés}
La difficulté s’inscrit comme l’un des principes de base dans la création d’un jeu vidéo, et l’un des mécanismes pourvoyeurs de plaisir principaux de celui-ci. Il n’y en en effet pour un joueur rien de plus frustrant qu’un jeu à la difficulté inexistante ou à l’inverse tout bonnement injouable de part sa difficulté excessive. \paragraph{}

Dans sa thèse, Guillaume Levieux [Levieux, 2011]\cite{Levi11} propose de définir la difficulté d’un jeu vidéo comme l’effort fourni par le joueur pour atteindre ses objectifs. La difficulté d’un jeu n’est pas une donnée stable et suit un processus qui doit être en constante évolution. Le niveau du joueur varie en effet au fil du jeu, du fait de son expérience et de son apprentissage, et la difficulté doit donc s’adapter. 
La difficulté n’est donc en fait pas une propriété du jeu mais la valeur de la relation entre le jeu et le joueur. Or en jouant, le joueur progresse, découvre l’univers du jeu et parfait sa connaissance de la mécanique du jeu, devient capable d’heuristiques pour prévoir les conséquences de ses actions, augmente ses capacités de coordination oculo-manuelle et sa vitesse de réalisation des actions. La difficulté est donc variable, et tend à diminuer au cours du temps. La figure \ref{evolution_difficulte} illustre l'évolution de la difficulté d'un jeu au cours du temps.\\
Notons que le niveau du joueur peut aussi varier à la baisse, si il ne joue pas au jeu pendant un certain temps par exemple. \\

G. Levieux précise donc dans sa définition de la difficulté, qu’il est important de prendre en compte son aspect relationnel avec le joueur et introduit alors les notions de difficulté absolue et relative~:
	\begin{itemize}
		\item la difficulté absolue d’un jeu décrit l’effort que doit fournir un joueur type, aux capacités statiques, pour atteindre les objectifs que son gameplay propose. 
		\item la difficulté relative d’un jeu décrit l’effort que doit fournir le joueur, dont les capacités évoluent tout au long du jeu, pour atteindre les objectifs que son gameplay propose.
	\end{itemize}
\paragraph{}Pour maintenir la difficulté relative du jeu, il est donc nécessaire d’augmenter la difficulté absolue du jeu en fonction de l’évolution des capacités du joueur.

\begin{figure}[!htbp]
	\centering
	\includegraphics[width=11cm]{images/evolution_difficulte.png}
	\caption{Evolution de la difficulté d'un jeu au cours du temps}
	\label{evolution_difficulte}
\end{figure}

\paragraph{}La difficulté augmente si l’on resserre les contraintes et délais d’exécutions des actions, si on ajoute de nouveaux éléments qui augmentent la complexité du système ou si l’on découvre une nouvelle partie de l’univers demandant ainsi une appréhension du système plus étend. A l’inverse, la difficulté tend à baisser pour le joueur qui travaille son habileté, ou enregistre le mécanisme de nouveaux éléments ou parties du jeu.

		\subsubsection{Types de difficulté}
Lorsqu'on pense aux jeux vidéo, on envisage naïvement deux types de difficultés : la difficulté de compréhension, et la difficulté d’exécution. Autrement dit, des jeux où il est difficile de savoir ce qu’il faut faire, et d’autres où il est difficile de réussir à le faire. En fait, tout jeu relève à la fois des deux types de difficultés, du moins dans une certaine mesure. C’est d’ailleurs un moyen de différencier un jeu casual (un peu des deux difficultés) d’un jeu hardcore (une des deux ou les deux, mais bien plus conséquentes). Cependant, ces deux types de difficultés ne s’opposent pas de manière binaire. Les jeux à haute difficulté d’exécution vont souvent être des jeux basés sur un gameplay classique mais en une version très difficile et poussée, alors que les jeux à haute difficulté de compréhension vont relever soit de leur propre genre, soit d’un genre nouveau unique. \\

Durant sa thèse, Guillaume Levieux\cite{Levi11} a tenté de mesurer le niveau de difficulté de plusieurs jeux, comme \emph{PacMan}(qui dépend directement de la vitesse de déplacement du joueur et des fantômes). En s’inspirant d’un modèle de traitement de l’information, il a identifié trois niveaux de difficulté~:
	\begin{itemize}
		\item la difficulté sensorielle qui correspond à la perception de l’univers,
		\item la difficulté logique se référant à la compréhension de l’univers,
		\item et la difficulté motrice, en rapport avec l'exécution physique de l’action à effectuer.
\end{itemize}
\paragraph{}L’effort du joueur n’est pas directement mesurable à partir de l’historique de jeu, mais ses résultats le sont. Le problème c’est que l’effort n’est pas normalisé et dépend de chaque style de jeu. La difficulté réside donc dans une relation entre un joueur et le défi qu’il doit relever. La difficulté est en effet relative aux capacités des joueurs : nous n’éprouvons pas tous les mêmes difficultés pour les mêmes jeux ni aux mêmes endroits. Ce qui signifie qu’il faut définir le niveau de capacité des joueurs pour évaluer le niveau de difficulté du jeu.\\
La difficulté d’un problème n’a rien à voir avec la complexité : c’est un point de vue humain sur un problème. Une solution est de mesurer les échecs et leur évolution dans un jeu, le taux d’échec étant le résultat visible du niveau de difficulté pour une personne.

\begin{figure}[!hbtp]
	\centering
	\includegraphics[width=\linewidth]{images/dimensions_difficulte.png}
	\caption{Dimensions de difficulté}
	\label{dimensions_difficulte}
\end{figure}

\paragraph{}Guillaume Levieux [réf] définit donc trois types de difficultés dans le jeu vidéo :
	\begin{itemize}
		\item la difficulté sensorielle : décrit l’effort que doit fournir le joueur pour obtenir de nouvelles informations sur l’état de l’univers du jeu. Ces informations nouvelles correspondent à toute information que le joueur ne peut pas déduire des faits et règles logiques qu’il connaît déjà.
		\item la difficulté logique : décrit l’effort que doit fournir le joueur pour exploiter les informations dont il dispose, c’est à dire comprendre le fonctionnement de l’univers par induction, et choisir la prochaine action à réaliser par déduction.
		\item la difficulté motrice : décrit le niveau de précision spatiale et temporelle dont le joueur doit faire preuve lorsqu’il exécute une action.
\end{itemize}
A titre d’exemple, on peut associer un type de jeu par type de difficulté. Les jeux d’aventure se basent essentiellement sur la difficulté sensorielle, les jeux de stratégie sur la difficulté logique, et les jeux d’actions sur la difficulté motrice. Bien sur, chaque jeu est composé de chacune des trois dimensions, mais exploitées dans des proportions différentes.
		
		\paragraph{\emph{Punitivité}\\ \quad}
Il s’agit de différencier la difficulté du jeu de la punition en cas d’échec. Ces punitions peuvent être dans l’ordre de sévérité : le respawn instantané, celui avec délai, la sauvegarde libre, le checkpoint, les vies limitées et la permadeath (mort immédiate et définitive). Ainsi, un jeu peut être très difficile mais peu punitif (\emph{Super Meat Boy}) ou plus facile mais très punitif (\emph{Binding of Isaac}, \emph{Diablo} en mode hardcore). Lorsqu’il est à la fois difficile et punitif, le jeu entre alors dans la catégorie des jeux Hardcore.

		\paragraph{\emph{Le casual et le hardcore}\\ \quad}
Difficulté et punitivité contribuent donc, parmi d’autres facteurs, à créer une relation entre le jeu et le joueur. Plus celles-ci vont être élevées, plus on va s’éloigner du jeu casual pour se rapprocher du jeu hardcore, où un véritable investissement devient nécessaire pour accomplir le jeu. Il nécessite alors un temps d’investissement important ou une concentration soutenue (difficulté), chaque action va peser (punitivité) et demander au joueur de s’investir, à l’inverse du jeu casual.		
		
		\subsubsection{Pourquoi aime-t'on la difficulté?}
			\paragraph{\emph{Chimiquement} \\ \quad}
Le jeu vidéo est capable de fournir aux joueurs des sensations permettant de délivrer au cerveau dopamine ou adrénaline. L’expérimentation du flow state permet aussi au joueur un ressenti qu’il va chercher à renouveler.

			\paragraph{\emph{L’engagement} \\ \quad}
Le jeu vidéo a par ailleurs cette particularité de faire que le joueur va avoir la volonté de recommencer un niveau ou une partie après un échec. Et cette volonté aura tendance à augmenter tant que le joueur n’aura pas atteint son objectif. Cette constatation peut être expliquée par la théorie psychosociale de l’engagement, et plus particulièrement du concept de dépense gâchée. Selon cette théorie, plus on a passé de temps dans une activité, à apprendre quelque chose ou dans une réalisation, moins on est enclin à y renoncer, sous prétexte du temps inutilement passé à s’y consacrer. Dans le jeu “je ne vais pas abandonner après être arrivé aussi loin !”. L’engagement (et l’attachement aux valeurs) est d’autant plus important que l’investissement a été important, que ce soit en terme de temps, d’efforts, de sacrifices, de souffrance, etc.

			\paragraph{\emph{Dissonance cognitive} \\ \quad}
Par ailleurs, l’humain (entre autre) est mal à l’aise et ressent une tension désagréable lorsqu’il est en état de dissonance cognitive. Cette dissonance est ressentie lorsque l’individu est en présence de cognitions (connaissances, croyances ou perceptions de soi ou son environnement) contradictoires ou incompatibles entre elles.
Cet état entraîne un inconfort psychologique, parfois une réaction émotionnelle, qui pousse la personne à penser ou agir. pour rétablir son équilibre cognitif à l’aide de stratégies inconscientes de rationalisation. L’éveil peut prendre bien des formes, la soumission, la rationalisation, la fuite, un comportement ou une action délibérée, la modification de ses croyances, attitudes ou connaissances pour les accorder avec la nouvelle cognition. Dans le jeu vidéo, cela se traduit par une auto justification de la persévérance du joueur, ou un rejet radical de l’activité. On va se trouver des excuses, etc.

			\paragraph{\emph{Découverte et apprentissage}  \\ \quad}
Ces principes sont primordiaux pour un certain nombre de joueurs.  Que ce soit la découverte d’un monde immense, des capacités de son personnage, des mécanismes du jeu ou encore d’un univers particulier, le plaisir réside dans le fait que rien n’est acquis et se découvre à force d’expérimentations et d’échecs. Au fur et à mesure de ses expériences et observations, le joueur va alors suivre une courbe de progression généralement logarithmique très gratifiante qui va l’inciter à poursuivre son apprentissage pour parfaire sa maîtrise du jeu. Cet intérêt est d’ailleurs suffisamment fort pour qu’une certaine communauté de joueurs complète ses connaissances à l’aide de forums, wiki ou vidéos qu’elle aura elle même mis en ligne.

			\paragraph{\emph{Auto-détermination} \\ \quad}
R. Ryan et al propose d’expliquer la motivation du joueur à travers l’auto-détermination. Ils considèrent que les jeux vidéo satisfont des besoins psychologiques et permettent le développement d’un sentiment d’autonomie, de compétence et de connexion. L’autonomie décrit à la fois le fait que l’investissement du joueur est volontaire et que le joueur possède une autonomie au sein du jeu.

			\paragraph{\emph{Auto satisfaction et dépassement de soi} \\ \quad}
Le plus grand plaisir qu’un joueur peut ressentir en jouant à un jeu difficile ou hardcore, est le sentiment d’auto satisfaction lorsqu’il réussit enfin à accomplir son action. A force d’efforts ou d’entraînement, il réussit à réaliser ce qu’il croyait impossible au premier abord, parce qu’il ne comprenait pas comment y arriver ou n’était simplement pas capable de le faire, par manque de techniques, d’imagination ou d’entraînement. Ce dépassement de soi (technique, intellectuel ou physique) est déjà gratifiant en soi, et récompense le long apprentissage auquel s’est adonné le joueur.

C'est la \textcolor{orange}{réussite} d'un challenge \textcolor{orange}{difficile} qui est satisfaisante, et non directement son accomplissement. À l'inverse, une majorité va choisir une difficulté normale plutôt que facile ou difficile, car les gens aiment \textcolor{vert}{faire} quelque chose qui représente un \textcolor{vert}{challenge modéré} : pas le choisir ou le réussir, moins glorieux.

			\paragraph{\emph{L’enjeu} \\ \quad}
C’est aussi un grand pourvoyeur de plaisir. Un fort enjeux va inciter le joueur à ne pas jouer à la légère, à s’impliquer et donc à s’appliquer dans sa partie. On notera que l’enjeu est d’autant plus fort lors de partie multijoueur : les actions d’un joueur peuvent potentiellement influencer l’expérience de jeu de chacun des autres joueurs. En confrontation, il faut arriver à surpasser l’autre joueur, qui va faire de son mieux pour vous en empêcher. En collaboration, où l’erreur de l’un peut alors aussi coûter aux autres. L’enjeu crée alors un sentiment de tension, qui va lui même renforcer l’immersion du joueur.

\paragraph{}Enfin l’intérêt des joueurs pour les jeux difficiles ou réputés comme tels, peut aussi s’expliquer par une certaine \emph{nostalgie}, une forme d’\emph{élitisme} voir de snobisme envers les jeux/joueurs dits casuals, mais surtout aussi par le plaisir ressenti par la réussite d’un défi qui leur est posé. Une forme de frustration idéalement dosée et que l’on a surmontée.

\newpage
		
\subsubsection{\emph{Encart proposition \\} Proposition d'une nouvelle composante : la difficulté émotionnelle}
Nous avons vu dans notre recherche documentaire que sont définies trois types de difficultés dans les jeux vidéo. [Levieux, 2011]\cite{Levi11} définit ainsi la difficulté sensorielle, la difficulté logique et la difficulté motrice.

\paragraph{}Il est aussi possible d’envisager une dimension émotionnelle dans la difficulté. Cette difficulté peut se manifester lors de la réalisation d’une action donc la réussite ou non est importante pour le joueur, lors d’une confrontation avec une situation, un problème ou un objet dont le joueur a peur ou le rend particulièrement mal à l’aise par exemple : mise en situation d’une phobie, d’une scène en désaccord avec ses moeurs ou convictions, lui rappelant des évènements difficiles ou traumatisants, etc.
 \paragraph{}
On peut citer l’exemple du jeu \emph{Paper Please}, dans lequel on incarne un employé travaillant à un poste de frontière et qui contrôle l’accès au pays. Dans ce jeu, le joueur sera partagé entre respecter les consignes strictes d’immigration et le caractère émotionnel et personnel des personnes souhaitant entrer dans le territoire avec des histoires et des motivations personnelles, personnes pour lesquelles il nous faudra décider si on autorise ou on restreint l’accès. Cette décision pourra être particulièrement difficile car elle se fera au risque de perdre son emploi et ne plus pouvoir faire survivre sa famille ou d’être en profond désaccord, voir en situation de dégoût, avec soi-même...\paragraph{}
Le joueur peut aussi s’imposer lui-même un certain nombre de contraintes, pour être en accord avec ses principes. Ces contraintes peuvent être d’ordre moral ou éthique (refus de tuer un personnage dans le jeu ou d’effectuer une mauvaise action), ou plus artificiel comme vouloir jouer de manière “Role Play” et donc s’interdire certaines actions ou au contraire s’en imposer d’autres. Ainsi, même si le joueur sait qu’il gagnerait à réaliser une action particulière et qu’il est en mesure d’y parvenir, il ne passera pas nécessairement à l’œuvre. 
\paragraph{}On pourra ainsi citer l’exemple du jeu \emph{Valkyrie Profile}, dans lequel le joueur peut contrôler un personnage principal, ainsi qu’un groupe de personnages secondaires. Durant les phases de combats tactiques, le joueur a la possibilité de sacrifier un personnage secondaire afin d’obtenir une puissance phénoménale, qui se révélera souvent nécessaire d’acquérir tant la difficulté du jeu est élevée. Mais ces sacrifices sont permanents et l’avatar, ainsi que le joueur, devront les assumer et vivre avec la conscience d’avoir tuer ces personnages, ce qui fera évoluer différemment l’histoire.
\paragraph{}
Un autre aspect émotionnel se trouve dans l’acceptation du déroulement du jeu. Dans un jeu multijoueur compétitif ou opposant une IA, si l’adversaire emploie une stratégie ou une technique particulièrement frustrante pour le joueur, celui-ci peut s’en trouver affecté (colère, énervement, mauvaise estime de soi, mauvaise foi). Si cette situation continue ou est répétée, ou bien que malgré une difficulté modérée notre joueur continue de perdre ou de se faire mener en bateau pour une raison ou une autre, la difficulté émotionnelle deviendra telle qu’il pourra préférer abandonner. Cette situation particulièrement fréquente dans les jeux multijoueur en ligne peut mener à ce que l’on appelle couramment un rage quit, qui désigne familièrement le fait pour un joueur de quitter une partie en cours sous l'effet de la colère.

\paragraph{}On peut noter que cette difficulté peut en fait impacter directement les trois autres aspects de la difficulté précédemment cités. Un joueur qui aura réellement peur perdra de ses capacités sensorielles, logiques ou motrices par exemple. Cet impact n’est cependant pas systématique : une situation obligeant le joueur à réaliser une action allant à l’encontre de ses principes n’affectera pas ses capacités, mais le joueur hésitera cependant à réaliser l’action qu’il sait nécessaire ; il aura compris la situation, trouvé la solution à sa réalisation et est physiquement capable de la réaliser, mais ne souhaitant pas la faire, différera son exécution, voir l’évitera si possible.

\paragraph{}Dans le cas particulier d’un Serious Game pour la réhabilitation motrice, la difficulté émotionnelle peut se situer dans l’intérêt particulier qu’a le joueur patient dans l’évolution de sa pathologie. Il est nécessaire en phase de rééducation que le patient soit capable de sentir qu’il progresse afin de garder sa motivation et poursuive son travail. Une confrontation trop fréquente à des gestes qu’il n’est pas encore/toujours pas capable de réaliser parce que trop difficile, aura pour conséquence de lui rappeler sa déficience et pourra lui faire perdre toute ambition thérapeutique.
		
	\subsubsection*{Relation entre les différents paramètres d'un jeu vidéo}
Afin de mieux comprendre pourquoi le jeu vidéo est un média apprécié et comment ses différents paramètres peuvent être utilisés pour des objectifs sérieux, j'ai cherché à trouver le lien entre ces composantes. Dans un contexte de rééducation, on va aussi chercher à connaître quelles théories sont intéressantes pour les thérapeutes, pour qu'ils puissent réaliser un classement par importance pour la thérapie. Comprendre les relations qu'il existe entre le jeu vidéo et le joueur pourrait aussi permettre de mieux comprendre les mécanismes en jeu et mieux orienter les exercices d'éducation ou de réhabilitation.
\begin{figure}[htbp]
Schéma inspiré des théories comportementales et psychologiques et de concepts mis en place dans les jeux vidéo.
	\centering
	\includegraphics[height=19.6cm]{images/lien_theories}
	\caption{Relation entre les principaux ressorts psychologiques d'un jeu vidéo}
	\label{lien_theories}
\end{figure}			

	\subsubsection{Difficulté dans les Serious Games}
Les SG ont la particularité de conjuguer les mécanismes classiques du jeu vidéo à des objectifs sérieux de nature différente. Ces objectifs peuvent être la transmission de connaissances ou de valeurs si la visée est intellectuelle, ou bien un travail sur la forme ou les capacités physiques du joueur. Dès lors, la difficulté du jeu se dote d’une nouvelle composante relative à cet objectif thérapeutique.\\
Dans le cas de SG physiques, qui utilisent des périphériques comme la wii board, la kinect ou le PSmove par exemple, on pourra assimiler cette nouvelle composante à la difficulté motrice déjà définie. A la difficulté de synchronisation oculo-motrice de la main sur le contrôleur, s’ajoute des difficultés physique telles que la précision, l’endurance, l’équilibre ou la souplesse.
Dans les jeux dont l’aspect sérieux est intellectuel, l’objectif sérieux peut venir enrichir la difficulté logique du jeu (difficulté de compréhension, de raisonnement, de mémoire).
Dans les jeux sérieux dont le but est une rééducation psychomotrice, il est aussi important d’envisager un nouvel aspect de difficulté de type émotionnel. Il faut en effet prendre en compte l’enjeu médical et la possible fragilité du joueur, dont la progression ou non peut avoir un impact important sur son mental.
		
	\subsubsection{Ajustement de la difficulté}
Il est nécessaire d'adapter la difficulté pour chaque joueur pour garantir une expérience de jeu optimale.			
\begin{quote}Malone : “Pour être stimulant, un jeu doit proposer un but que le joueur n’est pas certain d’atteindre”.
\end{quote}

L’objectif de l’ajustement de la difficulté est de pouvoir faire correspondre la difficulté du jeu aux capacités et niveau de jeu du joueur, de manière à ce que quel que soit son niveau, le feedback difficulté puisse être identique. Dans leur ouvrage \emph{On Game Design} \cite{Andr03} Andrew Rollings et Ernest Adams précisent cependant que l’ajustement ne doit pas être trop évident ni visible, afin d’éviter toute forme d’exploit, évidemment non voulu. Ils précisent aussi que cet ajustement ne doit pas se faire au détriment de l’impact décisif de l’action du joueur ; celui-ci doit rester le facteur décisif de sa réussite ou non, indépendemment de l’ajustement réalisé par le système. On pourra noter qu’un niveau de difficulté idéal mènerait le joueur à un taux d’échecs/réussites de 50/50.

		\paragraph{\emph{Évaluation de la difficulté}\\ \quad}
Modifier la difficulté d'un jeu vidéo ne semble pertinent qu'après en avoir évaluer le niveau lors de phases de tests. Deux types de tests peuvent être mis en place. 
\begin{itemize}
	\item des tests de jouabilité, réalisés par des joueurs testeurs : fidèles mais couteux et complexes à mettre en place, définis dans le temps et subjectifs.
	\item des tests par joueur synthétique : rapides, peu chers et répétables mais basiques, sans perception subjective, ignorent les aspects perceptifs.
\end {itemize}

	\paragraph{\emph{Jeux de progression VS jeux d’émergence} \\ \quad}
Les jeux de progression sont scénarisés et le level design est défini et fixé à l’avance. La difficulté y est donc statique et prédéfinie selon les choix des designers. L’ensemble des situations du jeu a été pensé et calibré. Le parcours est relativement contraint et organisé. \\
Dans les jeux émergents, on ne peut prévoir l’évolution de la partie. Seul un certain nombre de règles permet de définir et de faire évoluer le contenu du jeu par application de ces règles et de l’interaction du joueur avec les objets du jeu. \\
Progression et émergence constituent en fait les deux formes opposées d’une catégorie décrivant le contrôle que possède le level designer sur l’expérience du joueur. On remplacera alors le contrôle manuel des designer par des algorithmes de génération appropriés. Un jeu pourra ainsi se situer le long de l’axe décrivant ce contrôle selon sa conception du level design (figure \ref{axe_scenaristique}). 
\begin{figure}[hbtp]
\centering
\includegraphics[width=5cm]{images/axe_scenaristique.png}
\caption{Axe scénaristique d'un jeu vidéo}
\label{axe_scenaristique}
\end{figure}

\paragraph{}Il existe un sous domaine de l’intelligence artificielle qui cherche à lier contenus émergents et scénarisés afin de tirer parti des avantages respectifs de l’un et l’autre domaine tout en limitant leurs contraintes : la narration interactive. On cherche alors à créer un univers virtuel où l’on va pouvoir “aller n’importe où et faire n’importe quoi, quand on le souhaite” de manière cohérente à la fois d’un point de vu gameplay et scénario. Cela en adaptant par exemple le scénario en fonction du comportement du joueur (utilisation par exemple d’un Drama Manager).

			\paragraph{\emph{Scénarisation VS adaptation dynamique} \\ \quad}
Le problème de l’adaptation de la difficulté est abordable à partir de deux méthodes de game design : la scénarisation et l’adaptation dynamique de la difficulté.  Ces deux méthodes pourraient bénéficier d’une méthode générale de la mesure de la difficulté, pour s’adapter au mieux.

\paragraph{}La scénarisation est une manière d’encadrer l’apprentissage du joueur à l’aide d’un découpage du gameplay en phases successives. 

\paragraph{}L’adaptation dynamique permet de maintenir le niveau de difficulté du jeu cohérent avec les capacités du joueur. Ces méthodes d’auto-adaptation peuvent être très efficaces, notamment pour ajuster un certain nombre de paramètres. Des algorithmes peuvent ainsi restreindre la lisibilité du jeu ou relâcher les contraintes de réalisation d’une action en fonction des résultats du joueur. On peut alors soit jouer sur la difficulté de la tâche ou de l’action à réaliser, soit sur les moyens employables pour y arriver (vie, temps, capacités, temps de réaction, propriétés du joueur / de l’environnement, etc.). Attention tout de même à ne pas faire un “système parfait” qui ferait que le joueur ne serait alors plus le facteur déterminant de sa réussite.

		\paragraph{\emph{Équilibrage dynamique} \\ \quad}
L’équilibrage dynamique, ou ajustement dynamique, consiste à modifier un certain nombre de paramètres du gameplay afin de s’adapter au comportement du joueur [Levieux, 11]. Il faut cependant pour cela d’abord être capable d’évaluer l’équilibre du jeu, sans quoi toute modification serait infondée . Dans cet objectif, l’apprentissage dynamique est particulièrement exploré. Ces techniques permettent en effet de calculer automatiquement les meilleurs paramètres pour atteindre un but donné, la difficulté du gameplay en l'occurrence. Idéalement, le jeu serait ainsi capable de détecter quand et comment le joueur parvient à surpasser les obstacles qui lui sont proposés, puis d’y répondre en proposant des modifications visant à obliger le joueur à reconsidérer sa stratégie de jeu. Bien sur, il est possible que le système ne parvienne pas à détecter la stratégie (voir l’exploit) du joueur, ou soit incapable de fournir une réponse appropriée ou cohérente avec l’univers du jeu, d’un point de vue autre que le gameplay pur. En effet, de tels systèmes ne prennent pas en compte l’intégralité de la pensée du game designer, qui reste complexe et non entièrement formalisable (visée artistique, morale, intentions, proposition d’une ‘expérience’ de jeu, etc.).

\paragraph{}Un jeu est donc une activité qui demande un effort au joueur. Cet effort librement consenti par le joueur  doit être utile pour la réussite ou non à ce jeu, puisqu’il pourrait aisément être rendu vain ou superflu en modifiant les règles du jeu. La difficulté, définie comme l’effort réalisé par le joueur, est donc une composante essentielle du jeu de part sa relation avec le joueur.

		\subsubsection{Techniques d'adaptation dans les jeux ludiques et sérieux}
L'adaptation de la difficulté dans les jeux vidéo est une fonctionnalité importante qui permet d’individualiser et de contextualiser l'expérience de jeu. Dans le cas de serious games, elle permet également de gérer la frustration des joueurs-apprenants tout en augmentant leurs motivations [Hocine et al 11]. L'individualisation et la contextualisation du jeu pour chaque joueur-apprenant, notions déjà définies par [Gee, 05] comme importantes pour l'apprentissage, ont pour conséquence d'augmenter sa satisfaction tout en améliorant l’efficacité de la formation.
				
		\paragraph{}
La génération dynamique d’IA permet de modifier le niveau de difficulté en créant de nouvelles entités avec un niveau et des règles donnés, ou en modifiant des paramètres du gameplay en cours de jeu.\\
Par exemple, Andrade et al utilisent l’apprentissage dynamique pour ajuster la difficulté. L’algorithme consiste à utiliser une base de couples (action, état de jeu) associés chacun à une valeur d’efficacité, afin que le jeu choisisse pour chaque situation un comportement de l’efficacité souhaitée.			
				
		\paragraph{\emph{Systèmes adaptables VS systèmes auto-adaptatifs} \\ \quad} 
L’adaptation peut être définie comme une caractéristique exprimée au niveau d’un système, dans notre cas un système informatique, qui reflète sa capacité à se modifier structurellement en réaction à certains évènements bien identifiés (Andresen K. et al, 2005). Nous parlerons de système adaptable lorsque l’intervention humaine est nécessaire pour enclencher le processus de modification et de système auto-adaptatif si aucune intervention extérieure n'est nécessaire (Moisuc B., 2001)

\paragraph{}
Levieux précise que la génération dynamique d’IA permet de modifier le niveau de difficulté en créant de nouvelles entités avec un niveau et des règles donnés, ou en modifiant des paramètres du gameplay en cours de jeu.\\
Par exemple, Andrade et al utilisent l’apprentissage dynamique pour ajuster la difficulté. L’algorithme consiste à utiliser une base de couples (action, état de jeu) associés chacun à une valeur d’efficacité, afin que le jeu choisisse pour chaque situation un comportement de l’efficacité souhaitée.

		\paragraph{\emph{Adaptation de la difficulté dans les jeux sérieux} \\ \quad}
Contribuer à l'acceptation et à l'utilisation des jeux sérieux constitue un enjeu majeur pour la réussite et l'efficacité de ceux-ci. En effet, ces systèmes sont destinés à satisfaire les joueurs-apprenants et à répondre à leurs besoins en termes d'acquisition de compétences et/ou de divertissement. L’adaptation a pour but d’améliorer l’utilisabilité d’un jeu sérieux ou ludique en restructurant certaines de ses propriétés.

\paragraph{}
[Hocine et al, 11] se proposent d'étudier les différents systèmes d'adaptation dans les jeux ludiques et sérieux. Pour évaluer ces systèmes, ils définissent trois critères majeurs :
\begin{itemize}
	\item L’efficacité d’un système évalue le degré de succès avec lequel les utilisateurs réalisent leurs objectifs dans le système.
	\item L’efficience évalue les moyens mis en œuvre par les utilisateurs pour accomplir leurs objectifs.
	\item La satisfaction évalue le niveau d’acceptation par les utilisateurs.
\end{itemize}

\paragraph{}
Afin d'évaluer et de comparer les différents systèmes, ils utilisent un système d'évaluation des techniques d'adaptation intéressant et très parlant du fait qu'il repose sur un modèle MVC (voir figure \ref{criteres_adaptation}).

\begin{figure}[!hbtp]
	\centering
	\includegraphics[width=1\linewidth]{images/criteres_adaptation.png}
	\caption{Critères d’analyse des techniques d’adaptation [Hocine et al, 2011]\cite{Hoci11}}
	\label{criteres_adaptation}
\end{figure}

	\paragraph{Périmètre d'adaptation\\}
Le périmètre de l'adaptation identifie le périmètre dans lequel l'adaptation est appliquée, selon le modèle MVC. 
\begin{itemize}
	\item \emph{Adaptation de la présentation (la Vue)}. L'adaptation peut donc avoir lieu au niveau de l'interface, du son ou de feedbacks envers l'utilisateur. Dans Hammer \& Planks, nous avons ainsi rendu possible la modification de ces paramètres : ajustement du volume de la musique ou des bruitages, contraste, taille des objets, vitesse de jeu et possibilité de désactiver les éléments cosmétiques facultatifs au jeu (animation de la mer, effets visuels etc.).
		\item  \emph{Adaptation du contrôle} : ce niveau englobe les règles du jeu et les règles métier qui spécifient la dynamique du jeu (ou le gameplay) en réaction aux actions des joueurs. C'est dans ce niveau que l'on adaptera le niveau de difficulté du jeu. Suite à mon travail sur le jeu Hammer \& Planks, il est possible de modifier le nombre et le type d'ennemis que doit affronter le joueur, la vitesse du jeu, les propriétés des personnages (joueur ou ennemis) ou encore la fréquence des obstacles par exemple.
		\item  \emph{Adaptation du contenu (le Modèle)} : l'adaptation à ce niveau modifie dynamiquement soit les schémas de données utilisés ou bien le contenu. L'adaptation s'efforce donc de produire un contenu lié au contexte de jeu et aux compétences des joueurs. Nous pouvons citer à titre d'exemple la génération automatique des dialogues et textes narratifs (Barry G., 2007) ou d'ambiance sonore (Chen Y et al, 2006)
\end{itemize} 

	\paragraph{Paramètres d'adaptation\\}
Ce sont les éléments sur lesquels repose la prise de décision du processus d'adaptation. Ces éléments vont être utilisés soit comme déclencheurs de l'adaptation soit comme sources de données. Hocine et al distinguent deux types de paramètres en fonction de l'utilisateur~:
\begin{itemize}
	\item \emph{Modèle utilisateur} : ensemble de variables et métriques décrivant les caractéristiques de l’utilisateur dans le système. Ces caractéristiques peuvent être des données représentant les préférences de l’utilisateur, son état attentionnel, ses émotions et/ou ses compétences. Ces données sont stockées dans le profil de l’utilisateur qui sera utilisé comme paramètre de processus d’adaptation.
	\item \emph{Paramètre non-utilisateur ou variable système }: ce paramètre représente les variables propres au système et qui ne dépendent pas du modèle utilisateur. A titre d'exemple, nous pouvons citer les paramètres liés à la configuration matérielle et logicielle du système hôte.
\end{itemize}

	\paragraph{Modèle d'adaptation\\}
	L’adaptation peut être implémentée dans le système sous forme d’un module qui interagit avec le système pour modifier son comportement et sa structure. Ce module peut être~:
\begin{itemize}
	\item \emph{Implicite} : dans ce cas les procédures d'adaptation se retrouvent éparpillées et étroitement liées aux différents composants du système. Il serait dans ce cas difficile de séparer dans les instructions (ou le code source) les éléments qui incombent à l'adaptation des autres aspects.
	\item \emph{Explicite }: la technique d'adaptation utilise des modèles explicites comme un moteur de règles logiques, une matrice de décision ou des algorithmes d’IA.
\end{itemize}
	
	\paragraph{Adaptation Mono ou Multi-joueurs\\}
Contrairement aux jeux mono-joueur, l'adaptation dans un système multi-joueurs doit prendre en compte l'aspect collaboratif et l'hétérogénéité entre les joueurs tout  en maintenant une cohérence globale du jeu.
	
	\paragraph{\emph{Tour d'horizon de systèmes d'ajustement dynamiques dans les jeux vidéo}	 \\ \quad}
Durant son travail sur la difficulté dans les jeux vidéo, Levieux se propose d'étudier les différentes solutions d’équilibrage dynamique qui ont été mis en place dans le jeu vidéo. Bien que de tels systèmes, de part leur aspect automatique, ne nous seront pas directement utiles, il peut être intéressant d'en connaître les principales mises en œuvre. On distingue plusieurs types de solutions~: 
\begin{enumerate}
	\item l’apprentissage par renforcement [Sutton 98]~:
	\begin{itemize}
		\item jeux de combat : [Andrade 05], [Graepel 04]
		\item RTS : [Madeira 04], [Madeira 06], [Ula 05]
		\item FPS : [Lee-Urban 08]
	\end{itemize}
	\item le scripting dynamique, qui calcule des préférences à partir de règles écrites par le designer~:
	\begin{itemize}
		\item jeux d’aventure : Neverwinter Nights - Bioware
		\item RTS : Wargus [Spronck 05], [Spronck 06], [Spronck 08], [Timuri 07], [Ludwig 07]
	\end{itemize}
		\item évolution génétique et réseau de neurones (l’un et/ou l’autre)~:
	\begin{itemize}
		\item jeux d’actions [Demasi 05], [Spronck 02]
		\item RTS : [Ponsen 05], [Agogino 00]
		\item FPS : [Cole 04], [Thurau 03]
		\item jeux de sport : Fifa 99 -EA Games [Chan 04]
		\item puzzle : Tetris - Nintendo [Bohm 05]
		\item réalité virtuelle : [Yannakakis 09], [Yannakakis 07]
	\end{itemize}
		\item algorithmes de champs potentiels pour comportements stratégiques dans les FPS [Thurau 04]
		\item raisonnement au cas par cas dans les RTS : [Aha 05]
\end{enumerate}

	\subsubsection*{Comparaison avec les besoins de NaturalPad}
Dans leur état de l'art des techniques d'adaptation dans les jeux vidéo, [Hocine et al] définissent la différence entre les systèmes adaptables et les systèmes auto-adaptatif. Leur étude, comme celle de Levieux, se concentrent cependant sur les systèmes dont l'ajustement est automatique. Or, comme nous l'avons déjà dit, nos besoins et propositions sont plutôt de proposer des jeux mettant en place un système d'ajustement manuel. Un tel système permet à chaque thérapeute amené à utiliser un jeu sérieux pour la santé de notre environnement, d'adapter le jeu aux besoins thérapeutiques dont il aura précisément besoin selon le contexte : préférences thérapeutiques, pathologies et spécificités du patient ou encore état de santé ou de forme de celui-ci pour ne citer que quelques exemples.


%****************** Subsection 4 : Réhabilitation
		%%PART 4 : REHABILITATION			
	
	\subsection{Rééducation bimanuelle} \label{bilateral}
	Dans le cadre du développement du jeu Hammer \& Planks et de mon travail d'adaptation de contenu de serious games pour la rééducation, j'ai essayé d'envisager toutes les possibilités offertes par les interfaces naturelles ou \gls{nui}. Très vite, il m'est apparu que la richesse d'un gameplay par un contrôle naturel était liée au panel de possibilités de ce contrôle. Ainsi a priori, proposer de nombreux mouvements, notamment avec les deux membres supérieurs, pour interagir avec le jeu semble une bonne idée. Mais dans le cadre d'une réhabilitation post AVC, qui est donc la pathologie sur laquelle je me suis concentrée, il fallait m'assurer à la fois de la possibilité et de la pertinence d'une telle proposition. J'ai ainsi dirigé mes recherches vers un aspect beaucoup plus médical sur la réhabilitation bimanuelle post AVC. J'ai pour cela rencontré Julien Métrot, doctorant en médecine au laboratoire \emph{Movement 2 Health} et dont la spécialité est la récupération sensorimotrice des membres supérieurs. Il est aussi le co-auteur d'une étude sur le sujet.
	
		\subsubsection*{Évolution de la coordination bimanuelle} 
		
		\paragraph{\emph{Informations médicales}\\}
Une tâche nécessitant les deux mains entraîne une activation synchronisée des deux hémisphères, avec une balance égale du contrôle de l'inhibition. Cette activation est basée sur des concepts comportementaux et neurophysiologiques.\\
Le membre sain entraîne le membre membre parétique (voir \gls{paresie}) dans l'exécution de la tâche et améliore son résultat. A l'inverse, le membre affecté bride le membre non parétique à ses propres capacités		
		\paragraph{\emph{Analyse}\\}
Aujourd'hui, l'intérêt d'une thérapie bimanuelle fait débat car les études sur le sujet ne montrent pas de preuves assez probantes pour conclure à un apport bénéfique d'un \gls{bat}, ni qu'une thérapie BAT serait plus adaptée qu'une autre. Cependant, ces études ne prennent pas en compte toutes les phases de rétablissement : aiguë, subaiguë et chronique.\newline
[Metrot et al, 2013]\cite{Metr13} centrent leur analyse sur l'évolution de la coordination bimanuelle dans les six semaines qui suivent l'attaque ayant entrainée l'hémiplégie.

\paragraph{}
Les mesures cinématiques sont fortement suspectées de fournir une précision accrue et des mesures plus sensibles que les résultats cliniques [Rohrer et al, 2002 ; Alt Murphy et al, 2011] qui ne donnent qu'un score global et ne clarifient pas la question de savoir si les fluctuations proviennent d'un rétablissement moteur ou d'une compensation ni de comment les mouvements ont été réalisés. 

\paragraph{}
La fluidité du geste et le timing des mouvements semblent être les variables de mesure les plus sensibles de la récupération.

\paragraph{}
D'une manière générale, les actions bimanuelles sont plus longues et complexes à effectuer que les actions unimanuelles. A la surcharge cognitive de la coordination des deux mains, s'ajoute la plus grande dépendance de la main parétique. Les actions bimanuelles demandent aux patients un plus grand nombre de sous mouvements. 

		\paragraph{\emph{Constatations}\\}
On constate de forts progrès durant les trois premières semaines, avant d'atteindre une phase de plateau (progrès nuls) aux alentours de la 6ème semaine. Ce schéma concorde avec le fait que la plupart des progrès sont réalisés très tôt après l'attaque, puis diminuent progressivement. On peut considérer les progrès des 3 premières semaines comme un rétablissement spontané. La coordination des 2 mains durant les mouvements bimanuels est cependant considérée comme efficace autours de la 3ème semaine après l'inclusion (6ème semaine après l'attaque), indiquant une potentielle fenêtre pour un programme de réhabilitation. Ou à deux mois, après la phase de plateau.		

\paragraph{}
Whitall \cite{Whit04} indique que les patients légèrement affectés tireraient davantage de bénéfices d'une rééducation unilatérale, alors que les patients moyennement affectés auraient un meilleur retour d'une rééducation bilatérale. De plus, la coordination bimanuelle pourrait servir d'indicateur de rétablissement, et servir à informer de l'utilité ou non d'une rééducation bilatérale.

		\subsubsection*{Application dans Hammer \& Planks}
Hammer \& Planks est un jeu vidéo sérieux pour la santé dont la cible d'utilisateurs initiale sont des patients en période de réhabilitation post AVC. Comme nous venons de le voir, selon la phase de récupération et l'importance du handicap des patients, il peut être intéressant d'encourager une thérapie bimanuelle. On peut alors imaginer plusieurs types de contrôles et de gameplays allant dans ce sens~:
\begin{itemize}
	\item \emph{mouvement bilatéral symétrique} : pour réaliser une action, le joueur-patient doit réaliser un geste identique avec ces deux membres supérieurs. Il faut cependant penser que la personne risque de sur-utiliser son bras valide. On peut alors penser à un système permettant de ne prendre en compte que le geste le moins complet, correspondant a priori au membre parétique, pour asservir le contrôle. De cette manière, cela encouragerait le joueur à stimuler son hémicorps atteint.
	\item \emph{mouvement bilatéral asymétrique} : Très intéressant d'un point de vue gameplay, mais faire attention aux capacités cognitives du patient qui doivent être suffisamment bonnes pour comprendre les gestes à faire et réussir à les conceptualiser pour ensuite les réaliser. Cela est d'autant plus compliqué que les personnes hémiplégiques ont généralement des troubles cognitifs.
\end{itemize}
	
	\subsubsection*{Récupération de lombalgie}
Si la réhabilitation post AVC a été au cœur de mon travail et de celui de NaturalPad, son objectif est de pouvoir proposer ou accéder à des solutions pour divers types de pathologies. Ainsi, un projet de NaturalPad pour lequel j'ai participé à la phase de conception a pour objectif de créer un serious game pour la rééducation de personnes lombalgiques.
\paragraph{}
La lombalgie est un état douloureux du rachis lombaire qui peut être aiguë ou chronique. Les lombalgies affectent une forte proportion de personnes puisque entre 40 et 70\% de la population est touchée à un moment ou un autre. Sous l'effet de la douleur, une majorité des patients va cesser toute activité physique voir même professionnelle. Une de ses conséquences est aussi une démotivation de la personne pouvant aller jusqu'à un état de dépression, notamment dû à l'inactivité et la douleur. Comme préconisé dans le Guide du Dos\cite{backbook}, la reprise et le maintien d'une activité physique sont primordiaux dans le processus de récupération. \\
Le projet de NaturalPad est une application de coaching sportif adaptée à ce besoin et proposant un certain nombre d'exercices physiques gamifiés afin d'encourager la reprise d'activité des utilisateurs.

%****************** Subsection 5 : Système de recommandation
		\subsection{Système de recommandation }
(faire de l'adaptation via ce genre de système)
La proposition est ici de s'inspirer du monde la musique (ou libres, films, ventes en ligne) et de son système de recommandation. Il existe en fait deux types de recommandations. La recommandation sociale consiste par exemple à conseiller à un utilisateur des musiques qu'apprécient des personnes de son réseau, notamment si elles écoutent généralement des musiques identiques. Un autre exemple est sur un site de vente en ligne, de proposer à un utilisateur venant d'acheter un objet, une liste d'objets ayant aussi été achetés par d'autres utilisateurs en même temps que le premier objet.
Le second type de recommandation se base pas non pas sur l'environnement social de l'utilisateur, mais sur le contenu même des objets recommandés. L'idée est alors de chercher à décrire un objet selon certaines caractéristiques, et à faire de même pour les préférences de l'utilisateur. On va ensuite lui conseiller les objets qui semblent être le plus proche des attentes de l'utilisateur en se basant sur ces critères de préférences. 
 \paragraph{}
 Dans notre problématique, ce système de recommandation pourrait servir à sélectionner les paramètres de jeux, voir dans de futurs travaux le jeu lui même, qui correspondraient le mieux aux besoins du joueur. Rappelons que ces besoins peuvent être soit explicites, notamment à travers les recommandations et exigences du thérapeute, soit plus inconscients. Ces besoins inconscients représentent pas exemple les préférences du joueur-patient en terme de gameplay. Un jeu plus distrayant et motivant pour le patient renforcera son implication dans le programme de réhabilitation, et donc son rétablissement. Pour cela il faut donc à la fois connaître les préférences du patient, explicites ou `découvertes'  grâce à un système d'apprentissage par exemple, mais aussi s'appuyer sur un certain nombre de théories et connaissances que l'on sait efficaces pour renforcer cette immersion. 		
 
 
 	\subsubsection{State art}

Les systèmes de recommandation représentent les préférences de l'utilisateur dans le but de proposer des articles à acheter ou à examiner. Plusieurs techniques de recommandation ont été proposées : basée sur le contenu, sur des connaissances ou encore des techniques dîtes collaboratives ou sociales. Pour de meilleurs résultats, certaines de ces techniques peuvent être utilisées conjointement dans des systèmes de recommandation hybrides. Ce papier dresse d’abord un panorama des différents systèmes existants (en 2002) puis en propose un nouveau nommé EntreeC. EntreeC est un système de recommandation hybride combinant des connaissances à un réseau de collaboration pour recommander des restaurants.

Note : d’après Vincent Castaignet, fondateur et directeur de la publication de Musicovery.
Au-delà de la recommandation éditoriale, il y a différentes manières de proposer des artistes/titres par similarité : 
d’après les formes musicales sur lesquelles le goût des auditeurs est fondé
d’après les repères mentaux utilisés par les auditeurs (genres, sous-genres, style)
social (si tu es membre de cette tribu et aime cet artiste, alors tu vas aimer ces artistes)
contextuel (ceux qui écoutent ce titre dans ce contexte écoutent aussi ces titres).
Les passionnés de musique qui cherchent activement préféreront de la similarité type (2), le grand public plus passif de la similarité type (4). Un moteur de similarité intelligent devrait pouvoir combiner ces différentes formes et s’adapter en fonction du profil de chacun.
Pandora est principalement construit sur (1), last.fm est (2) et (3), Musicovery (4). Deezer avec ces 30 millions de playlists a un actif considérable à exploiter en (3) et (4).

Les systèmes de recommandation possèdent :
des données de base : données que le système possède avant même de commencer la recommandation
des données d’entrée : données que l’utilisateur fournit au système dans le but que ce dernier lui fournissent des recommandations.
un algorithme qui utilise ces données de base et d’entrée pour générer les résultats.
On peut distinguer 5 types de systèmes de recommandation :
    
avec : 
I : ensemble d’objets sur lequel sont faites les recommandations
U : l’ensemble d’utilisateurs dont les préférences sont connues
u : l’utilisateur pour lequel les recommandations doivent être générées
i : objets pour lesquels on souhaiterait prédire une préférence de la part de u

Techniques de recommandation :
Collaborative : méthode la plus mature et répandue. Le système agrège les notes ou recommandations des objets, relève les similarités entre les appréciations des utilisateurs et en déduit de nouvelles recommandations pour les utilisateurs. Certains systèmes prennent le temps en paramètre dans leur évaluation afin de prendre en compte l’évolution de l’intérêt des utilisateurs au fil du temps (effet de mode, etc). L’évaluation peut être simplement binaire ou plus complexe en utilisant une échelle de graduation. Les systèmes peuvent être soit basés sur une mémoire, comparant les utilisateurs par corrélation ou autre, soit basés sur un modèle :  celui-ci est dérivé à partir de l'historique des données d'évaluation et utilisé pour faire les prédictions.
La plus grande force de ces techniques est qu’elles sont complètement indépendantes de la représentation informatique des objets recommandés.
Démographique : ces systèmes de recommandation ont pour but de catégoriser l’utilisateur à partir de ses caractéristiques propres et de faire des recommandations en fonction de son appartenance à l’une des classes démographiques prédéfinies. L’avantage d’une approche démographique est qu’elle ne requiert pas un historique des évaluations des utilisateurs à l’inverse des méthodes collaboratives et basées sur le contenu.
Basée sur le contenu : La recommandation basée sur le contenu est une excroissance et la poursuite de la recherche d'information de filtrage. Dans ces systèmes, les objets sont définis en fonction de leurs caractéristiques associées. Le système apprend à connaître le profil de l’utilisateur en se basant sur les caractéristiques des objets évalués par l’utilisateur. C’est une corrélation objet-à-objet. Le profil dérivé dépend évidemment du type d’apprentissage employé : arbre de décisions, réseaux de neurones et représentations par vecteurs sont utilisés.
Fondée sur l’utilité : ces systèmes font des suggestions en se basant sur une estimation de l’utilité de chaque objet pour l’utilisateur. Le problème central étant bien comment créer cette fonction d’utilité pour chaque utilisateur. D’abord évaluer les objets (différentes méthodes), puis le profil utilisateur, avant de calculer la correspondance entre les deux. L’avantage de la technique est qu’elle peut prendre en compte des attributs non directement propres aux objets évalués (fiabilité du vendeur, disponibilité du produit, etc.) pour proposer des recommandations plus pertinentes (besoin immédiat ou meilleur prix par ex.).
Basée sur le savoir/connaissance : tenter de recommander des objets en inférant les besoins et préférences de l’utilisateur. Ces systèmes se distinguent en ce qu’ils ont un savoir fonctionnel : ils ont connaissance que tel objet répond à tel besoin et peuvent alors abstraire la relation entre le besoin et une possible recommandation.
Les systèmes de recommandations basés sur l’utilité et sur le savoir n’essaient pas de construire des généralisations à long terme à propos de leurs utilisateurs, mais préfèrent baser leurs conseils sur une évaluation de la correspondance entre les besoins d’un utilisateur et un ensemble d’options disponibles.



	
%****************** Subsection 6 : Méthodologies de conception
	\subsection{Méthodologies de conception}
On peut naïvement imaginer deux approches de conception s’opposant dans la conception des jeux sérieux. La première consisterait, de partir des objectifs sérieux et de proposer une “gamification” de ceux-ci en ajoutant des éléments de jeu. La seconde, à l’inverse, serait de partir de la composante ludique du jeu pour y intégrer ensuite le contenu sérieux. Bien qu’ayant l’avantage d’être simples à concevoir, ces deux démarches ont pour limite d’avantager l’une ou l’autre des composantes. Un jeu sérieux conçu à partir d’une base ludique aurait un impact sérieux limité, alors que l’ajout d’une composante ludique à une finalité sérieuse serait peu convaincant.
Une troisième approche est donc de prendre en considération à la fois la composante ludique et l’intention sérieuse dès le début du processus de conception pour les fusionner au mieux. Si elle est correctement mise en place, cette approche promet une forte utilisation du jeu et un impact sérieux efficace. Elle présente néanmoins le défaut de devoir imaginer une nouvelle solution pour tout nouveau couple (jeu, objectif sérieux). C’est ce type d’approche que nous proposons ici, et nous verrons comment les résultats peuvent être réemployables.

	\subsubsection*{Participative design : conception participative}
La conception participative est une méthode de travail utilisée principalement en conception de logiciel interactif. Sa principale caractéristique est la participation active des utilisateurs au travail de conception. Il s'agit donc d'une méthode de conception centrée sur l'utilisateur où l'accent est mis sur le rôle actif des utilisateurs \cite{wiki:cp}.

\paragraph{}
Il existe dans la littérature de nombreuses variantes de la méthode et de nombreuses techniques utilisées pour impliquer efficacement les utilisateurs. On peut noter particulièrement~:
\begin{itemize}
    \item l'observation et entretiens
    \item la production de scénarios
    \item le brainstorming
    \item le prototypage papier
    \item le prototypage vidéo
\end{itemize}

\paragraph{}
Une première séance de conception a lieu en début de projet : celle-ci regroupe les chercheurs et les développeurs de l'application, mais aussi les utilisateurs futurs ou potentiels. Intégrer les utilisateurs au processus de conception permet d'entrevoir au mieux leurs besoins et d'éviter un maximum d'erreurs d'interprétation ou d'oublis (voir illustration figure \ref{projet_info}). Les utilisateurs aident aussi à définir les problèmes éventuels et leurs possibles solutions.
\begin{figure}
	\centering
	\includegraphics[width=16cm]{images/projet_info.jpg}
	\caption{Allégorie d'un projet informatique}
	\label{projet_info}
\end{figure}

\paragraph{}Par ailleurs, plusieurs séances peuvent avoir lieu tout au long du projet, afin de vérifier si les besoins ont évolués et si le développement actuel est effectivement en accord avec ceux-ci. Les utilisateurs aideront aussi l'équipe de recherche et développement à juger de la pertinence des solutions apportées.
\paragraph{}
D'un point de vue du développement, on notera que ce type de conception s'accorde parfaitement avec une méthodologie AGILE, et notamment la méthode SCRUM qui consiste en de courtes périodes de développement entre lesquelles on met en relief l'avancement par rapport au projet global.

	
%	\newpage
	
	\newpage
	\section{Propositions et réalisations}
		-dire ce que je propose et pourquoi (pourquoi et pour quoi/qui, comment ça sera censé être utilisé, dans quel but) :
	\paragraph{partie théorique : difficulté émotionnelle et agencement des théories-attributs du jeu vidéo}
	\paragraph{mes propositions dans Hammer and planks pour ajuster la difficulté ou, alors dans réalisation??}
-> proposition d'une méthode de conception participative de JV thérapeutiques. Préciser les points importants de la conception, comme l'adaptation de la difficulté.
* propositions (théoriques) d'adaptation de de gameplay pour des NUI (m'inspirant de l'existant et des méthodes de rééduc)
	- parler ici de me la phase de veille / test, en testant le leap motion, des jeux sur Wii et Kinect PSMove, me permettant de faire des propositions plus pertinentes sur les controles.
* Cas pratique défini : travail sur l'équilibre (et lombalgie) et devices (kinect et wii)
*définition de la difficulté dans le jeu vidéo thérapeutique (ma position est que c'est différent des jeux classiques)
	
%	\newpage
%	\section{Réalisation}
		*développement d'une IT (collab avec andy), les fichiers waves et autres propositions (paramètres de difficulté, groupe d'ennemis)
		-aspect technique (paramétrisation) 
		-usage (retour de lapeyronie et intégration)
	*conception participative avec Arnaud 
		-impact mapping, carte d'empathie et scénarios d'usage et storyboard
	
	-travail avec les thérapeutes
	-répondre aux objectifs thérapeutique par un gamedesign
		- hammer \& Planks et l'équilibre (voir rapport d'anais)
		- classement des objectifs thérapeutoque, carte d'impact
	-ajustement de la difficulté et lien avec la thérapie : impact des paramètres en terme de difficulté (équilibre et dos?)
	
	-proposition de controle naturels pour des jeux pour une utilisation thérapeutiques
	
	\newpage
	\section{Perspectives}
		\subsection{\emph{Hammer \& Planks}, un serious game modulable}
Le projet \emph{Hammer \& Planks} fut une composante forte de mon stage. Il m'a servi de terrain d'expérimentation des différentes idées inspirées par les connaissances acquises au cours de ces six mois. J'ai ainsi travaillé aussi bien sur la version thérapeutique que sur la version grand public, dont les enjeux sont différents. C'est aussi un projet mettant un place un système d'ajustement des paramètres, permettant d'adapter sa difficulté et les mouvements des joueurs.

	\paragraph{}Par rapport à ce qui est déjà mis en place dans le jeu, il est possible d'imaginer de nouvelles fonctionnalités. Une approche à laquelle nous réfléchissons actuellement est la mise en place d'un \textbf{gameplay asymétrique}. \\
Cette proposition trouve son intérêt à la fois dans la version grand public et la version thérapeutique. L'idée est de permettre à un second joueur de participer à l'expérience de jeu en utilisant un périphérique tiers : tablette ou smartphone. Ce joueur pourrait par exemple incarner le vent pour aider ou perturber la navigation du joueur principal, jouer le rôle d'un membre de l'équipage pour donner des bonus ou encore se poser comme concurrent. 

\paragraph{}De manière générale, l'objectif est de permettre une interaction entre les joueurs et d'enrichir l'expérience globale de jeu. Nous avons vu dans notre background l'intérêt et la volonté des patients en rééducation de pouvoir partager des activités avec leurs proches ou d'autres patients.

\paragraph{} Une dernière proposition de fonctionnalité pour ajuster le niveau de difficulté, concerne la gestion des ennemis. Actuellement, le système offre la possibilité de régler la fréquence d'apparition des ennemis, ainsi que la densité de répartition des différents types d'ennemis, comme le montre la figure~\ref{interface_parametres}. \\
Afin d'apporter plus de finesse au système, je propose la possibilité de créer des bataillons d'ennemis. On pourrait alors par exemple définir la taille maximale d'un bataillon (nombre d'unités en son sein), le type des unités le composant ou encore un schéma de patrouille. De manière plus globale, il serait possible de préciser le nombre maximum de bataillons simultanés ou le temps d'apparition entre deux bataillons. Cela laisserait la possibilité pour le joueur, le designer ou le thérapeute d'établir de nouvelles stratégies. 

	\subsection{Proposer un système de paramètres prédéfinis}
Afin d'enrichir le système d'ajustement de la difficulté de \emph{Hammer \& Planks}, on pourrait proposer quelques configurations prédéfinies de paramètres, comme il se fait classiquement dans les jeux vidéo. Dans la version grand public, cela se traduirait par la proposition de plusieurs niveaux de difficulté ("facile", "moyen", "difficile").

\paragraph{}
La proposition est particulièrement intéressante pour la version santé du jeu, notamment couplée avec l'interface thérapeutique. \\
Il serait en effet beaucoup plus ergonomique et intuitif pour les thérapeutes de pouvoir directement choisir des ensembles de valeurs de paramètres dont l'application correspondrait à un objectif thérapeutique. Par exemple, plutôt que de devoir modifier manuellement tous les paramètres ayant un impact sur l'utilisation du bras gauche du joueur-patient, il lui suffirait de choisir l'ensemble prédéfini correspondant. On peut aussi parler de meta-paramètres.

\paragraph{} Bien sur, plusieurs ensembles pourraient correspondre aux mêmes besoins. Enfin, afin de permettre une adaptation la plus complète possible, il faudrait laisser la possibilité aux soignants de créer ou de modifier eux-mêmes ce type de configurations. Cela permettrait de faciliter l'utilisation de la plateforme et des jeux par les thérapeutes.

\paragraph{} Dans le cadre de l'adaptation aux besoins thérapeutiques, il est aussi envisageable de créer plusieurs modes de jeu spécifiques ou d'ajouter de nouvelles possibilités dans l'interface thérapeutiques. Par exemple, si l'on cherche à faire réaliser par le joueur-patient un certain schéma moteur (suite de modifications de son centre de gravité par exemple), il serait pratique pour le thérapeute de créer à la volée le parcours du jeu incitant le joueur à réaliser un tel schéma.


	\subsection{Proposer un système de recommandation}
Comme nous l'avons vu dans la partie~\ref{recommandation}, les systèmes de recommandation permettent d'orienter des personnes vers des produits qui correspondent à leurs attentes ou besoins. 

\paragraph{} En s'inspirant de ces modèles, il semblerait judicieux de pouvoir établir un système de profil pour les joueurs. En analysant les capacités du joueur et ses affinités, on pourrait alors nourrir un système de recommandation. Celui-ci pourrait orienter le joueur vers des jeux en accord avec ses affinités ou lui proposer des sets de paramètres personnalisés correspondant à ses critères de difficulté par exemple. L'ajout d'un système de suivi de ses résultats pour prendre en compte l'évolution du niveau du joueur et de ses affinités permettrait d'améliorer encore la qualité des propositions.

\paragraph{} D'un point de vue thérapeutique, un tel système intégrerait le profil médical du joueur-patient. La prise en compte de ses capacités cognitives et motrices, de la nature de sa déficience et des exercices préconisés par les soignants alimenterait efficacement un système de recommandation. Celui-ci viendrait améliorer l'adaptation des jeux sérieux pour le patient, ce qui contribuerait d'autant plus à sa réhabilitation. N'oublions pas que les jeux sérieux renforcent la motivation du patient, et que c'est un élément essentiel à sa rééducation.

\paragraph{}
Enfin, en évaluant le niveau du joueur durant ses sessions de jeu, il serait possible de mettre en place un système d'ajustement dynamique de la difficulté. Ce système viendrait en complément de la paramétrisation initiale proposée grâce à l'évaluation initiale du profil du joueur. Adaptations initiale et dynamique seraient ainsi deux moyens de proposer aux joueurs une expérience de jeu la plus en adéquation avec leurs envies (et éventuellement, leurs besoins thérapeutiques).
	
	\subsection{Étendre et éprouver la méthodologie de conception}
	-étendre le domaine de la méthodologie à d'autre pathologies et matériels\newline
	-éprouver les propositions de méthodo, sur un cas concret et complet\newline
	
	- Le pb c'est que j'ai rien qui dise clairement "pour tel exercice/objectif, fais plutot ce type de jeu avec tels contrôles. Du coup, est-ce que ca pourrait pas être une forme de perspective? disant que ça nécessiterait du travail et serait moins parfait qu'un truc personnalisé, mais ca permettrait de gagner du temps tout en étant mieux que ce qui existe actuellement
	A l'inverse, la démarche que j'ai emprunté dans ma méthode semble mieux coller aux besoins, mais nécessite plus de travail.
 
 bien expliquer si ces démarches sont différentes, complémentaires ou partiellement complémentaires, dans quels contextes elles s'appliquent et quels sont les compétences nécessaires à quels moments pour les mettre en place.
 Pensez aux perspectives de l'ordre de "ca serait bien d'éprouver la méthodologie sur un projet complet" , "il faudrait une équipe avec telles compétences"
	
	\newpage
	\section{Conclusion}
	prise de comp technique, blabla sur ma premiere exp de 6 mois, que c'est bien vive la fac et les bisounours.
autonomie, responsabilité et confiance en mon travail
rencontre avec des professionnels de milieu différents, et travail ensemble
poursuite de la découverte du mond médical, intéret de faire de l'informatique pour autre chose que de l'info, et surtout pour une bonne cause, que ce soit en médical ou en serious game en général (je pense a l'éducatif avec les théories d'apprentissage).
	
	\newpage
	\printglossaries

	\newpage
	\bibliography{rapport}

\newpage
	\part{État de l'art} \appendix	
	%% PART 2 : SERIOUS GAMES
\subsection{Les Serious Games}
Les serious games ou jeux sérieux sont donc une catégorie de jeux ayant la particularité d'avoir une portée supplémentaire au divertissement. Ils restent cependant des jeux en ce qu'ils en intégrent les caractéristiques ludiques, et même plus les utilisent pour parvenir plus aisément ou efficacement à leur objectif sérieux. Dans cette partie, nous allons voir comment un jeu sérieux  peut parvenir à des résultats sérieux et quels sont les mécanismes sous-jacents qui permettent une telle efficacité, souvent inconsciente chez le joueur.
	%% A Serious Games et Théories Comportementales 
	\subsubsection{Serious Games et Théories Comportementales }
			\paragraph{}
Les jeux vidéo sérieux se présentent comme un potentiel médiateur dans la modification des habitudes comportementales en permettant d’inclure des connaissances pratiques dans un modèle ludique apprécié. Il est possible d’y mettre en place des procédures de changement comme l’établissement d’objectifs ou la modélisation et le développement de compétences dans un environnement attrayant, significatif et immersif [Baranowski et al, 2008]. \\
Les jeux vidéo promeuvent les interactions sociales et d’apprentissage [Wideman et all, 2008], créent un environnement où les actions du joueur ont un effet [Gee, 2004] , encouragent la résolution de problèmes [Gee, 2004] et renforcent la compréhension en créant des situations de réflexion ou en aidant le joueur dans ses objectifs [Gee, 2004]. Enfin, les jeux sérieux pour la santé sont fait pour distraire le joueur tout en l’éduquant, en l’entrainant ou en changeant ses comportements [Stokes, 2005].

		\subsubsection*{Théories comportementales}
Le comportement est la résultante d’influences multiples, rendant ainsi souvent les personnes réfractaires au changement [Baranowski, Lin \& al, 1997]. Le comportement doit alors être considéré comme un mécanisme complexe découlant de l’enchaînement de plusieurs étapes. Ainsi, plutôt que de chercher à impacter directement le comportement, les experts comportementaux valorisent une action sur ces facteurs intermédiaires, appelés médiateurs. Changer ces médiateurs permet de changer le comportement [Baranowski, Lin \& al 1997].
\paragraph{}Plusieurs grandes théories comportementales existent~:
\begin{itemize}
	\item la théorie d’inoculation comportementale [McGuire, 1961]
	\item la théorie socio-cognitive [Bandura, 1986]
	\item la théorie de l’auto-détermination [Ryan \& Deci, 2000]
	\item la théorie de l’immersion [Green \& Brock, 2000]
\end{itemize}

\paragraph{}De ces théories, l’on peut alors identifier un certains nombre de ces facteurs médiateurs tels que : l’immersion, l’attention, l’auto-régulation, le développement de compétences, la motivation interne et externe, l’autonomie ou encore le sentiment de compétence. La science du comportement fournit aussi des techniques qui facilitent le changement comportemental, et propose d’utiliser ces facteurs dans les média de divertissement tel que le jeu vidéo.\\
Le modèle : "Elaboration Likelihood Model" [Petty \& Cacioppo, 1986] soutient ainsi que des personnages crédibles, attrayants et sympathiques sont plus susceptibles d’être persuasifs que les autres et peuvent donc servir d’intermédiaires pour véhiculer un message. \\
La théorie d’inoculation comportementale [McGuire, 1961] met en garde contre une possible contre-productivité en identifiant et en réfutant les menaces potentielles à l’accomplissement des objectifs du changement désiré.\\
La théorie socio-cognitive [Bandura, 1986] préconise l’établissement d’objectifs et le développement de compétences comme paramètres importants dans le changement comportemental.\\
Enfin, les théories socio-cognitive [Bandura, 1986] et d’auto-détermination [Ryan \& Deci, 2000] mettent toutes deux l’accent sur l’importance du feedback pour guider et mettre en forme le comportement durant le processus de changement.	

		\subsubsection*{Théories comportementales dans un serious game éducatif}
			\paragraph{Identification \\ \quad}
Fait référence aux feedbacks ou aux conseils qui sont \emph{personnalisés} afin de mieux toucher le joueur [Kreuter, Strecher \& Glassman, 1999]. 
			\paragraph{Mini-jeux de connaissances \\ \quad}
Des connaissances pratiques et théoriques spécifiques sont un élément nécessaire, mais pas suffisant, d’une modification des habitudes comportementales [Bandura, 1986]. Ces mini-jeux peuvent être explicites (dans un mode à part) ou correspondrent aux boucles micro du gameplay.
			\paragraph{Ajustement des objectifs \\ \quad}
C’est un processus complexe qui permet à la fois de définir un objectif et les manières de l’atteindre [Gollwitzer, 1999], en prenant en considération les capacités et valeurs personnelles pour les lier à son objectif [Ryan \& Deci, 2000]. Parce que l’autonomie améliore la motivation personnelle [Ryan \& Deci, 2000], proposer et permettre au joueur de réaliser ses propres choix ou de paramétrer ses objectifs est donc conseillé.
			\paragraph{Résolution de problèmes \\ \quad}
Bien que les difficultés et obstacles interfèrent de premier abord dans la réalisation de ses objectifs, arriver à les surmonter est un moteur de motivation puissant [Frauenknecht \& Black, 1995]. Le joueur doit prévoir ces difficultés et mettre en place un plan pour arriver à les dépasser. Ces solutions doivent paraitre réalisables et plausibles pour ne pas décourager le joueur [Elaboration Likelihood Model : Petty \& Cacioppo, 1986].
			\paragraph{Exposition des motivations \\ \quad}
La théorie de l’auto-détermination [Ryan \& Deci, 2000] indique que la motivation personnelle est d’autant plus importante que la personne voit le lien entre son comportement et des choses importantes pour elle, comme une valeur qui lui est chère. Par exemple dans un serious game de rééducation motrice, si pouvoir faire du tennis est une chose importante pour le joueur-patient, on lui rappellera que réaliser des étirements quotidient permet de récupérer plus rapidement.
			\paragraph{Revue des objectifs et du travail accompli\\ \quad}
Cette activité renforce le sentiment d’accomplissement personnel et augmente l’auto-efficacité [Bandura, 1986].
			\paragraph{Feedback \\ \quad}
Un feedback bien conçu renforce l’efficacité personnelle [Schuk, 1986] et le sentiment de compétence. [Ryan \& Deci, 2000]
			
\paragraph{}
Pour vérifier l'application de ces théories sur un cas pratique, [Thompson et al, 09] ont participé à la création d'un Serious Game~: DIAB. DIAB est un jeu vidéo divertissant, mais sérieux et se basant sur ces concepts théoriques, conçu pour réduire les risques de diabète de type 2 et d’obésité infantile. Il émerge de cette expérience que les jeux sérieux fondés sur une base théorique peuvent être efficaces pour parvenir à un changement à la fois dans le régime alimentaire et l’activité physique. Peu est encore connu dans le domaine, mais il en sort que les mécanismes et processus de changements comportementaux ont aussi leur place dans les jeux sérieux et peuvent permettre d'améliorer l'efficacité de ces derniers. Cette expérience décrit aussi comment des experts du divertissement et d'une spécialité différente peuvent allier leurs talents respectifs pour créer un jeu sérieux efficace et divertissant basé sur un socle théorique. 
	%% Apprentissage et propriétés des jeux vidéo
	\subsubsection{Apprentissage et propriétés des jeux vidéo}
			\subsubsection*{Théories de l'apprentissage}
Les serious games sont aujourd'hui utilisés comme nouveaux vecteurs d'apprentissage. Ils sont en effet porteurs de nombreux éléments qui sont propices à l'apprentissage [Baranowski, 08]. On ne connait cependant pas la relation entre les attributs d'un jeux et les résultats de l'apprentissage, ni si ces liens sont directs ou indirects, si un seul paramètre ou une combinaison de ceux-ci auront un impact différent ou plus important [Wilson 09]. 

\paragraph{}Plusieurs théories de l'apprentissage permettent d'en expliquer les mécanismes, que nous essaieront par la suite de lier aux paramètres d'un jeu vidéo.\\

Cognitive learning outcomes :  Dans cette théorie, [Kraiger et all, 93] développent 3 notions : la connaissance déclarative (le quoi), la connaissance procédurale (le comment) et le savoir tacite ou stratégique (qui, quand, pourquoi). Ces trois catégories décrivent le processus cognitif d'apprentissage, et se rapprochent fortement des sous-catégories de la connaissance de [Bloom, 1956].\\ 

Skill-based learning outcomes : En suivant le développement des résultats cognitifs, les apprenants progressent vers des méthodes d'apprentissage basées sur les compétences se concentrant sur le développement de techniques ou de compétences motrices. Les compétences psychomotrices demande de la pratique et peuvent être catégorisées parmi 7 catégories de la plus simple à la plus complexe : perception (sentir les moteurs permettant de réaliser le mouvement), volonté d'agir, imitation, automatisme(réalisation d'un schéma moteur), réponse manifeste complexe, adaptation et originalité.\\ 

Affective learning outcomes [Kraiger et al 1993]. Selon cette théorie, l'apprentissage façonne aussi les sentiments d'un individu. Kraiger et al décrivent les résultats d'un apprentissage par les émotions comme impliquant des concepts tels que l'attitude, la motivation et les objectifs.

			\subsubsection*{Propriétés des jeux vidéo}
Selon [de Felix et Johnson , 1993], les jeux sont composés d'éléments visuels dynamiques, d'interactions, de règles et d'objectifs. [Malone et Lepper, 1987] mentionnent le challenge, la curiosité, le contrôle et la fantaisie comme caractéristiques intégrantes des jeux vidéo. Puis [Thiagarajan, 1999] affirme que le conflit, le contrôle, la terminalité et l'artifice sont les quatre éléments nécessaires d'un jeu. Enfin en 2001, [Garis and Ahlers, 01] donnent 39 descripteurs que [Wilson et all, 09] réduisent à 12 pour ne garder que les paramètres statistiquement les plus significatifs pour renforcer la sensation de "game-like".

La table \ref{game_attributes} indiquent et décrit ces douze attributs.
%TOTO faire la table de la page 230 de Wilson

\paragraph{TODO}
Finally, Driskell and Dwyer (1984) consulted the literature and found that several
gaming characteristics influence motivational and learning properties, specifically,
goals, challenge, fantasy, and mystery. They theorize that an increase in motivation
will in turn increase attention and lead to better retention, recall of learned information
(i.e., declarative knowledge), and focalization of attention (i.e., cognitive strategies).
Driskell and Dwyer also suggest that higher levels of fantasy and mystery
increase user motivation by arousing curiosity and excitement to play the game.
These relationships remain to be tested.

		\subsubsection*{Les jeux vidéo comme outil d'apprentissage}
Synthèse :
This article asks how good video and computer game designers manage to get new players to learn
long, complex and difficult games. The short answer is that designers of good games have hit on
excellent methods for getting people to learn and to enjoy learning. The good principles of learning
built into successful games. The author discusses 13 such principles under the headings of
"Empowered Learners", "Problem Solving" and "Understanding" and concludes that the main
impediment to implementing these principles in formal education is cost.
1. La participation des apprenants ( Empowered Learners)
• Co-design : l'apprenant doit être un acteur actif (producer) et non passif (consumer)
• Customisation : l'apprenant doit pouvoir prendre des décisions quant à la méthode
d'apprentissage (chacun est différent) et s'essayer (être encouragé) à de nouvelles.
• Identification : un apprentissage en profondeur requiert une réelle implication,
implication qui se fait bien plus présente lors d'un jeu de rôle
• Manipulation et connaissance distribuée : les sciences cognitives montrent que la
pratique, notamment au travers d'un tiers (robot, entité, etc) permet de ressentir soi
même les choses et de mieux les comprendre (empathie).
2. Résolution de problèmes
• Difficulté progressive (well-ordered problems) : être confronté trop tôt à un problème
trop difficile n'est pas efficace, bien que pouvant amener à des solutions très créatrices ;
alors qu'une évolution progressive de la difficulté permet à l'apprenant de comprendre
seul au fur et à mesure.
• Frustration plaisante : notion de challenge : dur mais faisable
• Cycle d'expertise : répéter et pratiquer ses compétences jusqu'à les maîtriser, puis
confronter l'apprenant à une difficulté lui demandant de mettre en pratique toute cette
maîtrise, ancienne ou récente. Puis répéter ce cycle à un niveau plus élevé (système
levels /boss)
• Information ‘On Demand’ and ‘Just in Time’ :
On part du constat que les humains ne peuvent / veulent pas emmagasiner une grande
quantité d'information sans contexte. On apprend beaucoup mieux une information si
elle est donnée dans un cas d'utilisation, un contexte, liée à une image, etc. C'est le « just
in time ». De la même manière, on peut vouloir la retrouver aisément lorsqu'on en a
besoin, le « on demand ». Cela se traduit dans les jeux par les tutoriels et manuelsd'utilisation.
• Aquarium : C'est le principe de l'écosystème simplifié permettant d'appréhender un sous
ensemble des paramètres, afin de ne pas être directement surchargé par la masse
d'information. Jeux-> version démo ou premiers niveaux simplifiés (didacticiels)
• Sandboxes : Ou Bac à sable : proposer un environnement identique à l'environnement
réel, mais plus sur et sécurisé. Dans ce contexte, l'apprenant va pouvoir expérimenter
sans risque avec un réel sentiment d'accomplissement et d'authenticité.
Il n'y a rien de pire qu'un jeu où après bien des efforts, on perd juste avant de pouvoir
sauver. Cela induit une manière de jouer très 'safe' où ne prend plus aucun risque sans
chercher à explorer les possibilités du jeu.
• Skills as Strategies : Sur le principe de « c'est en forgeant que l'on devient forgeron »,
c'est à force de pratiquer une compétence qu'on finit par la maîtriser. Mais on n'aime pas
pratiquer sans raison, il faut donc arriver à lier ses objectif perso avec l'utilisation de la
compétence. Ingame -> passer par une phase obligatoire d'utilisation d'un skill (zelda)
3. Compréhension
• Pensée globale (System Thinking ): Comprendre comment s'intègre la notion ou
compétence nouvelle dans le système global permet de mieux l'appréhender. Pousser
cette compréhension (en montrant quels leviers d'action permet ou non chaque
élément)de manière à être capable d'inférer les règles 'sémantiquement' liées.
• Meaning as Action Image : Le fait que quand les gens pensent à un concept, ils n'y
pensent pas à travers sa définition générale, mais à travers l'expérience personnelle qu'ils
ont de ce concept. Lié au principe « Just on time and on demand »
Réf :
Gee [2003], Gee[2004],
jugement perso :
Comparaison entre le monde de l'école / du travail et celui de JV intéressante. Effectivement, on
n'aime rarement apprendre spontanément des choses très difficiles, et on le fait pourtant dans
certains bons JV, comprendre et analyser pourquoi est une bonne démarche.		

	%% Serious games thérapeutiques
	\subsubsection{Serious games pour la santé}
TODO 
\begin{figure}
TODO Succinte explication du lien entre mes composantes. Vérifier que ca se trouve avec le reste et pas en fin de document.
	\centering
	\includegraphics[height=19.6cm]{images/lien_theories}
	\caption{Relation entre les principaux ressorts psychologiques d'un jeu vidéo}
	\label{lien_theories}
\end{figure}
	\subsection{Réhabilitation et serious games pour la santé TODO}

		\subsubsection{Adaptation de la difficulté en rééducation fonctionnelle }
		aujourd'hui (exercices de récup sensorielle, de mouvements, etc)
Existence de jeux sérieux à but thérapeutique ayant pour but de faciliter la réhabilitation en maintenant la motivation du patient. Cependant, ces jeux sont encore rares et ne remplissent pas encore parfaitement leur rôle à cause du problème de l'ajustement de la difficulté en fonction du joueur.
Or comme l'a vu, la difficulté joue un rôle important dans la satisfaction et la motivation du joueur. La question se pose donc de savoir comment ajuster de manière la difficulté d'un jeu afin qu'elle sied au mieux à chaque joueur, à chacune de ses sessions, dans le but final de renforcer la récupération motrice du joueur-patient. On va par ailleurs chercher à fournir le meilleur environnement virtuel possible pour chaque situation, avec ici comme objectif de contexte à terme un couple patient-thérapeute avec considérations des objectifs thérapeutiques. 
		
		\subsubsection{personnes âgées}
pas/peu d'expérience de jeu, voir même des nouvelles technologies, capacités physiques restreintes, background social à prendre en compte (jeux de cartes préférés aux jeux de guerre)
		\subsubsection{hémiplégiques}		
Contraintes spécifiques.  
	%%PART 3 : LA DIFFICULTE
\subsection{Difficulté}
Dans son livre \emph{La cigale : jeux, vie et utopie}, le philosophe Bernard SUITS indiquait : \begin{quote}{“Jouer consiste à tenter volontairement de surmonter des obstacles inutiles”}.  \end{quote}
		
	\subsubsection{Définition et propriétés}
La difficulté s’inscrit comme l’un des principes de base dans la création d’un jeu vidéo, et l’un des mécanismes pourvoyeurs de plaisir principaux de celui-ci. Il n’y en en effet pour un joueur rien de plus frustrant qu’un jeu à la difficulté inexistante ou à l’inverse tout bonnement injouable de part sa difficulté excessive. \paragraph{}

Dans sa thèse, Guillaume Levieux [Levieux, 2011]\cite{Levi11} propose de définir la difficulté d’un jeu vidéo comme l’effort fourni par le joueur pour atteindre ses objectifs. La difficulté d’un jeu n’est pas une donnée stable et suit un processus qui doit être en constante évolution. Le niveau du joueur varie en effet au fil du jeu, du fait de son expérience et de son apprentissage, et la difficulté doit donc s’adapter. 
La difficulté n’est donc en fait pas une propriété du jeu mais la valeur de la relation entre le jeu et le joueur. Or en jouant, le joueur progresse, découvre l’univers du jeu et parfait sa connaissance de la mécanique du jeu, devient capable d’heuristiques pour prévoir les conséquences de ses actions, augmente ses capacités de coordination oculo-manuelle et sa vitesse de réalisation des actions. La difficulté est donc variable, et tend à diminuer au cours du temps. La figure \ref{evolution_difficulte} illustre l'évolution de la difficulté d'un jeu au cours du temps.\\
Notons que le niveau du joueur peut aussi varier à la baisse, si il ne joue pas au jeu pendant un certain temps par exemple. \\

G. Levieux précise donc dans sa définition de la difficulté, qu’il est important de prendre en compte son aspect relationnel avec le joueur et introduit alors les notions de difficulté absolue et relative~:
	\begin{itemize}
		\item la difficulté absolue d’un jeu décrit l’effort que doit fournir un joueur type, aux capacités statiques, pour atteindre les objectifs que son gameplay propose. 
		\item la difficulté relative d’un jeu décrit l’effort que doit fournir le joueur, dont les capacités évoluent tout au long du jeu, pour atteindre les objectifs que son gameplay propose.
	\end{itemize}
\paragraph{}Pour maintenir la difficulté relative du jeu, il est donc nécessaire d’augmenter la difficulté absolue du jeu en fonction de l’évolution des capacités du joueur.

\begin{figure}[!htbp]
	\centering
	\includegraphics[width=11cm]{images/evolution_difficulte.png}
	\caption{Evolution de la difficulté d'un jeu au cours du temps}
	\label{evolution_difficulte}
\end{figure}

\paragraph{}La difficulté augmente si l’on resserre les contraintes et délais d’exécutions des actions, si on ajoute de nouveaux éléments qui augmentent la complexité du système ou si l’on découvre une nouvelle partie de l’univers demandant ainsi une appréhension du système plus étend. A l’inverse, la difficulté tend à baisser pour le joueur qui travaille son habileté, ou enregistre le mécanisme de nouveaux éléments ou parties du jeu.

		\subsubsection{Types de difficulté}
Lorsqu'on pense aux jeux vidéo, on envisage naïvement deux types de difficultés : la difficulté de compréhension, et la difficulté d’exécution. Autrement dit, des jeux où il est difficile de savoir ce qu’il faut faire, et d’autres où il est difficile de réussir à le faire. En fait, tout jeu relève à la fois des deux types de difficultés, du moins dans une certaine mesure. C’est d’ailleurs un moyen de différencier un jeu casual (un peu des deux difficultés) d’un jeu hardcore (une des deux ou les deux, mais bien plus conséquentes). Cependant, ces deux types de difficultés ne s’opposent pas de manière binaire. Les jeux à haute difficulté d’exécution vont souvent être des jeux basés sur un gameplay classique mais en une version très difficile et poussée, alors que les jeux à haute difficulté de compréhension vont relever soit de leur propre genre, soit d’un genre nouveau unique. \\

Durant sa thèse, Guillaume Levieux\cite{Levi11} a tenté de mesurer le niveau de difficulté de plusieurs jeux, comme \emph{PacMan}(qui dépend directement de la vitesse de déplacement du joueur et des fantômes). En s’inspirant d’un modèle de traitement de l’information, il a identifié trois niveaux de difficulté~:
	\begin{itemize}
		\item la difficulté sensorielle qui correspond à la perception de l’univers,
		\item la difficulté logique se référant à la compréhension de l’univers,
		\item et la difficulté motrice, en rapport avec l'exécution physique de l’action à effectuer.
\end{itemize}
\paragraph{}L’effort du joueur n’est pas directement mesurable à partir de l’historique de jeu, mais ses résultats le sont. Le problème c’est que l’effort n’est pas normalisé et dépend de chaque style de jeu. La difficulté réside donc dans une relation entre un joueur et le défi qu’il doit relever. La difficulté est en effet relative aux capacités des joueurs : nous n’éprouvons pas tous les mêmes difficultés pour les mêmes jeux ni aux mêmes endroits. Ce qui signifie qu’il faut définir le niveau de capacité des joueurs pour évaluer le niveau de difficulté du jeu.\\
La difficulté d’un problème n’a rien à voir avec la complexité : c’est un point de vue humain sur un problème. Une solution est de mesurer les échecs et leur évolution dans un jeu, le taux d’échec étant le résultat visible du niveau de difficulté pour une personne.

\begin{figure}[!hbtp]
	\centering
	\includegraphics[width=\linewidth]{images/dimensions_difficulte.png}
	\caption{Dimensions de difficulté}
	\label{dimensions_difficulte}
\end{figure}

\paragraph{}Guillaume Levieux [réf] définit donc trois types de difficultés dans le jeu vidéo :
	\begin{itemize}
		\item la difficulté sensorielle : décrit l’effort que doit fournir le joueur pour obtenir de nouvelles informations sur l’état de l’univers du jeu. Ces informations nouvelles correspondent à toute information que le joueur ne peut pas déduire des faits et règles logiques qu’il connaît déjà.
		\item la difficulté logique : décrit l’effort que doit fournir le joueur pour exploiter les informations dont il dispose, c’est à dire comprendre le fonctionnement de l’univers par induction, et choisir la prochaine action à réaliser par déduction.
		\item la difficulté motrice : décrit le niveau de précision spatiale et temporelle dont le joueur doit faire preuve lorsqu’il exécute une action.
\end{itemize}
A titre d’exemple, on peut associer un type de jeu par type de difficulté. Les jeux d’aventure se basent essentiellement sur la difficulté sensorielle, les jeux de stratégie sur la difficulté logique, et les jeux d’actions sur la difficulté motrice. Bien sur, chaque jeu est composé de chacune des trois dimensions, mais exploitées dans des proportions différentes.
		
		\paragraph{\emph{Punitivité}\\ \quad}
Il s’agit de différencier la difficulté du jeu de la punition en cas d’échec. Ces punitions peuvent être dans l’ordre de sévérité : le respawn instantané, celui avec délai, la sauvegarde libre, le checkpoint, les vies limitées et la permadeath (mort immédiate et définitive). Ainsi, un jeu peut être très difficile mais peu punitif (\emph{Super Meat Boy}) ou plus facile mais très punitif (\emph{Binding of Isaac}, \emph{Diablo} en mode hardcore). Lorsqu’il est à la fois difficile et punitif, le jeu entre alors dans la catégorie des jeux Hardcore.

		\paragraph{\emph{Le casual et le hardcore}\\ \quad}
Difficulté et punitivité contribuent donc, parmi d’autres facteurs, à créer une relation entre le jeu et le joueur. Plus celles-ci vont être élevées, plus on va s’éloigner du jeu casual pour se rapprocher du jeu hardcore, où un véritable investissement devient nécessaire pour accomplir le jeu. Il nécessite alors un temps d’investissement important ou une concentration soutenue (difficulté), chaque action va peser (punitivité) et demander au joueur de s’investir, à l’inverse du jeu casual.		
		
		\subsubsection{Pourquoi aime-t'on la difficulté?}
			\paragraph{\emph{Chimiquement} \\ \quad}
Le jeu vidéo est capable de fournir aux joueurs des sensations permettant de délivrer au cerveau dopamine ou adrénaline. L’expérimentation du flow state permet aussi au joueur un ressenti qu’il va chercher à renouveler.

			\paragraph{\emph{L’engagement} \\ \quad}
Le jeu vidéo a par ailleurs cette particularité de faire que le joueur va avoir la volonté de recommencer un niveau ou une partie après un échec. Et cette volonté aura tendance à augmenter tant que le joueur n’aura pas atteint son objectif. Cette constatation peut être expliquée par la théorie psychosociale de l’engagement, et plus particulièrement du concept de dépense gâchée. Selon cette théorie, plus on a passé de temps dans une activité, à apprendre quelque chose ou dans une réalisation, moins on est enclin à y renoncer, sous prétexte du temps inutilement passé à s’y consacrer. Dans le jeu “je ne vais pas abandonner après être arrivé aussi loin !”. L’engagement (et l’attachement aux valeurs) est d’autant plus important que l’investissement a été important, que ce soit en terme de temps, d’efforts, de sacrifices, de souffrance, etc.

			\paragraph{\emph{Dissonance cognitive} \\ \quad}
Par ailleurs, l’humain (entre autre) est mal à l’aise et ressent une tension désagréable lorsqu’il est en état de dissonance cognitive. Cette dissonance est ressentie lorsque l’individu est en présence de cognitions (connaissances, croyances ou perceptions de soi ou son environnement) contradictoires ou incompatibles entre elles.
Cet état entraîne un inconfort psychologique, parfois une réaction émotionnelle, qui pousse la personne à penser ou agir. pour rétablir son équilibre cognitif à l’aide de stratégies inconscientes de rationalisation. L’éveil peut prendre bien des formes, la soumission, la rationalisation, la fuite, un comportement ou une action délibérée, la modification de ses croyances, attitudes ou connaissances pour les accorder avec la nouvelle cognition. Dans le jeu vidéo, cela se traduit par une auto justification de la persévérance du joueur, ou un rejet radical de l’activité. On va se trouver des excuses, etc.

			\paragraph{\emph{Découverte et apprentissage}  \\ \quad}
Ces principes sont primordiaux pour un certain nombre de joueurs.  Que ce soit la découverte d’un monde immense, des capacités de son personnage, des mécanismes du jeu ou encore d’un univers particulier, le plaisir réside dans le fait que rien n’est acquis et se découvre à force d’expérimentations et d’échecs. Au fur et à mesure de ses expériences et observations, le joueur va alors suivre une courbe de progression généralement logarithmique très gratifiante qui va l’inciter à poursuivre son apprentissage pour parfaire sa maîtrise du jeu. Cet intérêt est d’ailleurs suffisamment fort pour qu’une certaine communauté de joueurs complète ses connaissances à l’aide de forums, wiki ou vidéos qu’elle aura elle même mis en ligne.

			\paragraph{\emph{Auto-détermination} \\ \quad}
R. Ryan et al propose d’expliquer la motivation du joueur à travers l’auto-détermination. Ils considèrent que les jeux vidéo satisfont des besoins psychologiques et permettent le développement d’un sentiment d’autonomie, de compétence et de connexion. L’autonomie décrit à la fois le fait que l’investissement du joueur est volontaire et que le joueur possède une autonomie au sein du jeu.

			\paragraph{\emph{Auto satisfaction et dépassement de soi} \\ \quad}
Le plus grand plaisir qu’un joueur peut ressentir en jouant à un jeu difficile ou hardcore, est le sentiment d’auto satisfaction lorsqu’il réussit enfin à accomplir son action. A force d’efforts ou d’entraînement, il réussit à réaliser ce qu’il croyait impossible au premier abord, parce qu’il ne comprenait pas comment y arriver ou n’était simplement pas capable de le faire, par manque de techniques, d’imagination ou d’entraînement. Ce dépassement de soi (technique, intellectuel ou physique) est déjà gratifiant en soi, et récompense le long apprentissage auquel s’est adonné le joueur.

C'est la \textcolor{orange}{réussite} d'un challenge \textcolor{orange}{difficile} qui est satisfaisante, et non directement son accomplissement. À l'inverse, une majorité va choisir une difficulté normale plutôt que facile ou difficile, car les gens aiment \textcolor{vert}{faire} quelque chose qui représente un \textcolor{vert}{challenge modéré} : pas le choisir ou le réussir, moins glorieux.

			\paragraph{\emph{L’enjeu} \\ \quad}
C’est aussi un grand pourvoyeur de plaisir. Un fort enjeux va inciter le joueur à ne pas jouer à la légère, à s’impliquer et donc à s’appliquer dans sa partie. On notera que l’enjeu est d’autant plus fort lors de partie multijoueur : les actions d’un joueur peuvent potentiellement influencer l’expérience de jeu de chacun des autres joueurs. En confrontation, il faut arriver à surpasser l’autre joueur, qui va faire de son mieux pour vous en empêcher. En collaboration, où l’erreur de l’un peut alors aussi coûter aux autres. L’enjeu crée alors un sentiment de tension, qui va lui même renforcer l’immersion du joueur.

\paragraph{}Enfin l’intérêt des joueurs pour les jeux difficiles ou réputés comme tels, peut aussi s’expliquer par une certaine \emph{nostalgie}, une forme d’\emph{élitisme} voir de snobisme envers les jeux/joueurs dits casuals, mais surtout aussi par le plaisir ressenti par la réussite d’un défi qui leur est posé. Une forme de frustration idéalement dosée et que l’on a surmontée.

\newpage
		
\subsubsection{\emph{Encart proposition \\} Proposition d'une nouvelle composante : la difficulté émotionnelle}
Nous avons vu dans notre recherche documentaire que sont définies trois types de difficultés dans les jeux vidéo. [Levieux, 2011]\cite{Levi11} définit ainsi la difficulté sensorielle, la difficulté logique et la difficulté motrice.

\paragraph{}Il est aussi possible d’envisager une dimension émotionnelle dans la difficulté. Cette difficulté peut se manifester lors de la réalisation d’une action donc la réussite ou non est importante pour le joueur, lors d’une confrontation avec une situation, un problème ou un objet dont le joueur a peur ou le rend particulièrement mal à l’aise par exemple : mise en situation d’une phobie, d’une scène en désaccord avec ses moeurs ou convictions, lui rappelant des évènements difficiles ou traumatisants, etc.
 \paragraph{}
On peut citer l’exemple du jeu \emph{Paper Please}, dans lequel on incarne un employé travaillant à un poste de frontière et qui contrôle l’accès au pays. Dans ce jeu, le joueur sera partagé entre respecter les consignes strictes d’immigration et le caractère émotionnel et personnel des personnes souhaitant entrer dans le territoire avec des histoires et des motivations personnelles, personnes pour lesquelles il nous faudra décider si on autorise ou on restreint l’accès. Cette décision pourra être particulièrement difficile car elle se fera au risque de perdre son emploi et ne plus pouvoir faire survivre sa famille ou d’être en profond désaccord, voir en situation de dégoût, avec soi-même...\paragraph{}
Le joueur peut aussi s’imposer lui-même un certain nombre de contraintes, pour être en accord avec ses principes. Ces contraintes peuvent être d’ordre moral ou éthique (refus de tuer un personnage dans le jeu ou d’effectuer une mauvaise action), ou plus artificiel comme vouloir jouer de manière “Role Play” et donc s’interdire certaines actions ou au contraire s’en imposer d’autres. Ainsi, même si le joueur sait qu’il gagnerait à réaliser une action particulière et qu’il est en mesure d’y parvenir, il ne passera pas nécessairement à l’œuvre. 
\paragraph{}On pourra ainsi citer l’exemple du jeu \emph{Valkyrie Profile}, dans lequel le joueur peut contrôler un personnage principal, ainsi qu’un groupe de personnages secondaires. Durant les phases de combats tactiques, le joueur a la possibilité de sacrifier un personnage secondaire afin d’obtenir une puissance phénoménale, qui se révélera souvent nécessaire d’acquérir tant la difficulté du jeu est élevée. Mais ces sacrifices sont permanents et l’avatar, ainsi que le joueur, devront les assumer et vivre avec la conscience d’avoir tuer ces personnages, ce qui fera évoluer différemment l’histoire.
\paragraph{}
Un autre aspect émotionnel se trouve dans l’acceptation du déroulement du jeu. Dans un jeu multijoueur compétitif ou opposant une IA, si l’adversaire emploie une stratégie ou une technique particulièrement frustrante pour le joueur, celui-ci peut s’en trouver affecté (colère, énervement, mauvaise estime de soi, mauvaise foi). Si cette situation continue ou est répétée, ou bien que malgré une difficulté modérée notre joueur continue de perdre ou de se faire mener en bateau pour une raison ou une autre, la difficulté émotionnelle deviendra telle qu’il pourra préférer abandonner. Cette situation particulièrement fréquente dans les jeux multijoueur en ligne peut mener à ce que l’on appelle couramment un rage quit, qui désigne familièrement le fait pour un joueur de quitter une partie en cours sous l'effet de la colère.

\paragraph{}On peut noter que cette difficulté peut en fait impacter directement les trois autres aspects de la difficulté précédemment cités. Un joueur qui aura réellement peur perdra de ses capacités sensorielles, logiques ou motrices par exemple. Cet impact n’est cependant pas systématique : une situation obligeant le joueur à réaliser une action allant à l’encontre de ses principes n’affectera pas ses capacités, mais le joueur hésitera cependant à réaliser l’action qu’il sait nécessaire ; il aura compris la situation, trouvé la solution à sa réalisation et est physiquement capable de la réaliser, mais ne souhaitant pas la faire, différera son exécution, voir l’évitera si possible.

\paragraph{}Dans le cas particulier d’un Serious Game pour la réhabilitation motrice, la difficulté émotionnelle peut se situer dans l’intérêt particulier qu’a le joueur patient dans l’évolution de sa pathologie. Il est nécessaire en phase de rééducation que le patient soit capable de sentir qu’il progresse afin de garder sa motivation et poursuive son travail. Une confrontation trop fréquente à des gestes qu’il n’est pas encore/toujours pas capable de réaliser parce que trop difficile, aura pour conséquence de lui rappeler sa déficience et pourra lui faire perdre toute ambition thérapeutique.
		
	\subsubsection*{Relation entre les différents paramètres d'un jeu vidéo}
Afin de mieux comprendre pourquoi le jeu vidéo est un média apprécié et comment ses différents paramètres peuvent être utilisés pour des objectifs sérieux, j'ai cherché à trouver le lien entre ces composantes. Dans un contexte de rééducation, on va aussi chercher à connaître quelles théories sont intéressantes pour les thérapeutes, pour qu'ils puissent réaliser un classement par importance pour la thérapie. Comprendre les relations qu'il existe entre le jeu vidéo et le joueur pourrait aussi permettre de mieux comprendre les mécanismes en jeu et mieux orienter les exercices d'éducation ou de réhabilitation.
\begin{figure}[htbp]
Schéma inspiré des théories comportementales et psychologiques et de concepts mis en place dans les jeux vidéo.
	\centering
	\includegraphics[height=19.6cm]{images/lien_theories}
	\caption{Relation entre les principaux ressorts psychologiques d'un jeu vidéo}
	\label{lien_theories}
\end{figure}			

	\subsubsection{Difficulté dans les Serious Games}
Les SG ont la particularité de conjuguer les mécanismes classiques du jeu vidéo à des objectifs sérieux de nature différente. Ces objectifs peuvent être la transmission de connaissances ou de valeurs si la visée est intellectuelle, ou bien un travail sur la forme ou les capacités physiques du joueur. Dès lors, la difficulté du jeu se dote d’une nouvelle composante relative à cet objectif thérapeutique.\\
Dans le cas de SG physiques, qui utilisent des périphériques comme la wii board, la kinect ou le PSmove par exemple, on pourra assimiler cette nouvelle composante à la difficulté motrice déjà définie. A la difficulté de synchronisation oculo-motrice de la main sur le contrôleur, s’ajoute des difficultés physique telles que la précision, l’endurance, l’équilibre ou la souplesse.
Dans les jeux dont l’aspect sérieux est intellectuel, l’objectif sérieux peut venir enrichir la difficulté logique du jeu (difficulté de compréhension, de raisonnement, de mémoire).
Dans les jeux sérieux dont le but est une rééducation psychomotrice, il est aussi important d’envisager un nouvel aspect de difficulté de type émotionnel. Il faut en effet prendre en compte l’enjeu médical et la possible fragilité du joueur, dont la progression ou non peut avoir un impact important sur son mental.
		
	\subsubsection{Ajustement de la difficulté}
Il est nécessaire d'adapter la difficulté pour chaque joueur pour garantir une expérience de jeu optimale.			
\begin{quote}Malone : “Pour être stimulant, un jeu doit proposer un but que le joueur n’est pas certain d’atteindre”.
\end{quote}

L’objectif de l’ajustement de la difficulté est de pouvoir faire correspondre la difficulté du jeu aux capacités et niveau de jeu du joueur, de manière à ce que quel que soit son niveau, le feedback difficulté puisse être identique. Dans leur ouvrage \emph{On Game Design} \cite{Andr03} Andrew Rollings et Ernest Adams précisent cependant que l’ajustement ne doit pas être trop évident ni visible, afin d’éviter toute forme d’exploit, évidemment non voulu. Ils précisent aussi que cet ajustement ne doit pas se faire au détriment de l’impact décisif de l’action du joueur ; celui-ci doit rester le facteur décisif de sa réussite ou non, indépendemment de l’ajustement réalisé par le système. On pourra noter qu’un niveau de difficulté idéal mènerait le joueur à un taux d’échecs/réussites de 50/50.

		\paragraph{\emph{Évaluation de la difficulté}\\ \quad}
Modifier la difficulté d'un jeu vidéo ne semble pertinent qu'après en avoir évaluer le niveau lors de phases de tests. Deux types de tests peuvent être mis en place. 
\begin{itemize}
	\item des tests de jouabilité, réalisés par des joueurs testeurs : fidèles mais couteux et complexes à mettre en place, définis dans le temps et subjectifs.
	\item des tests par joueur synthétique : rapides, peu chers et répétables mais basiques, sans perception subjective, ignorent les aspects perceptifs.
\end {itemize}

	\paragraph{\emph{Jeux de progression VS jeux d’émergence} \\ \quad}
Les jeux de progression sont scénarisés et le level design est défini et fixé à l’avance. La difficulté y est donc statique et prédéfinie selon les choix des designers. L’ensemble des situations du jeu a été pensé et calibré. Le parcours est relativement contraint et organisé. \\
Dans les jeux émergents, on ne peut prévoir l’évolution de la partie. Seul un certain nombre de règles permet de définir et de faire évoluer le contenu du jeu par application de ces règles et de l’interaction du joueur avec les objets du jeu. \\
Progression et émergence constituent en fait les deux formes opposées d’une catégorie décrivant le contrôle que possède le level designer sur l’expérience du joueur. On remplacera alors le contrôle manuel des designer par des algorithmes de génération appropriés. Un jeu pourra ainsi se situer le long de l’axe décrivant ce contrôle selon sa conception du level design (figure \ref{axe_scenaristique}). 
\begin{figure}[hbtp]
\centering
\includegraphics[width=5cm]{images/axe_scenaristique.png}
\caption{Axe scénaristique d'un jeu vidéo}
\label{axe_scenaristique}
\end{figure}

\paragraph{}Il existe un sous domaine de l’intelligence artificielle qui cherche à lier contenus émergents et scénarisés afin de tirer parti des avantages respectifs de l’un et l’autre domaine tout en limitant leurs contraintes : la narration interactive. On cherche alors à créer un univers virtuel où l’on va pouvoir “aller n’importe où et faire n’importe quoi, quand on le souhaite” de manière cohérente à la fois d’un point de vu gameplay et scénario. Cela en adaptant par exemple le scénario en fonction du comportement du joueur (utilisation par exemple d’un Drama Manager).

			\paragraph{\emph{Scénarisation VS adaptation dynamique} \\ \quad}
Le problème de l’adaptation de la difficulté est abordable à partir de deux méthodes de game design : la scénarisation et l’adaptation dynamique de la difficulté.  Ces deux méthodes pourraient bénéficier d’une méthode générale de la mesure de la difficulté, pour s’adapter au mieux.

\paragraph{}La scénarisation est une manière d’encadrer l’apprentissage du joueur à l’aide d’un découpage du gameplay en phases successives. 

\paragraph{}L’adaptation dynamique permet de maintenir le niveau de difficulté du jeu cohérent avec les capacités du joueur. Ces méthodes d’auto-adaptation peuvent être très efficaces, notamment pour ajuster un certain nombre de paramètres. Des algorithmes peuvent ainsi restreindre la lisibilité du jeu ou relâcher les contraintes de réalisation d’une action en fonction des résultats du joueur. On peut alors soit jouer sur la difficulté de la tâche ou de l’action à réaliser, soit sur les moyens employables pour y arriver (vie, temps, capacités, temps de réaction, propriétés du joueur / de l’environnement, etc.). Attention tout de même à ne pas faire un “système parfait” qui ferait que le joueur ne serait alors plus le facteur déterminant de sa réussite.

		\paragraph{\emph{Équilibrage dynamique} \\ \quad}
L’équilibrage dynamique, ou ajustement dynamique, consiste à modifier un certain nombre de paramètres du gameplay afin de s’adapter au comportement du joueur [Levieux, 11]. Il faut cependant pour cela d’abord être capable d’évaluer l’équilibre du jeu, sans quoi toute modification serait infondée . Dans cet objectif, l’apprentissage dynamique est particulièrement exploré. Ces techniques permettent en effet de calculer automatiquement les meilleurs paramètres pour atteindre un but donné, la difficulté du gameplay en l'occurrence. Idéalement, le jeu serait ainsi capable de détecter quand et comment le joueur parvient à surpasser les obstacles qui lui sont proposés, puis d’y répondre en proposant des modifications visant à obliger le joueur à reconsidérer sa stratégie de jeu. Bien sur, il est possible que le système ne parvienne pas à détecter la stratégie (voir l’exploit) du joueur, ou soit incapable de fournir une réponse appropriée ou cohérente avec l’univers du jeu, d’un point de vue autre que le gameplay pur. En effet, de tels systèmes ne prennent pas en compte l’intégralité de la pensée du game designer, qui reste complexe et non entièrement formalisable (visée artistique, morale, intentions, proposition d’une ‘expérience’ de jeu, etc.).

\paragraph{}Un jeu est donc une activité qui demande un effort au joueur. Cet effort librement consenti par le joueur  doit être utile pour la réussite ou non à ce jeu, puisqu’il pourrait aisément être rendu vain ou superflu en modifiant les règles du jeu. La difficulté, définie comme l’effort réalisé par le joueur, est donc une composante essentielle du jeu de part sa relation avec le joueur.

		\subsubsection{Techniques d'adaptation dans les jeux ludiques et sérieux}
L'adaptation de la difficulté dans les jeux vidéo est une fonctionnalité importante qui permet d’individualiser et de contextualiser l'expérience de jeu. Dans le cas de serious games, elle permet également de gérer la frustration des joueurs-apprenants tout en augmentant leurs motivations [Hocine et al 11]. L'individualisation et la contextualisation du jeu pour chaque joueur-apprenant, notions déjà définies par [Gee, 05] comme importantes pour l'apprentissage, ont pour conséquence d'augmenter sa satisfaction tout en améliorant l’efficacité de la formation.
				
		\paragraph{}
La génération dynamique d’IA permet de modifier le niveau de difficulté en créant de nouvelles entités avec un niveau et des règles donnés, ou en modifiant des paramètres du gameplay en cours de jeu.\\
Par exemple, Andrade et al utilisent l’apprentissage dynamique pour ajuster la difficulté. L’algorithme consiste à utiliser une base de couples (action, état de jeu) associés chacun à une valeur d’efficacité, afin que le jeu choisisse pour chaque situation un comportement de l’efficacité souhaitée.			
				
		\paragraph{\emph{Systèmes adaptables VS systèmes auto-adaptatifs} \\ \quad} 
L’adaptation peut être définie comme une caractéristique exprimée au niveau d’un système, dans notre cas un système informatique, qui reflète sa capacité à se modifier structurellement en réaction à certains évènements bien identifiés (Andresen K. et al, 2005). Nous parlerons de système adaptable lorsque l’intervention humaine est nécessaire pour enclencher le processus de modification et de système auto-adaptatif si aucune intervention extérieure n'est nécessaire (Moisuc B., 2001)

\paragraph{}
Levieux précise que la génération dynamique d’IA permet de modifier le niveau de difficulté en créant de nouvelles entités avec un niveau et des règles donnés, ou en modifiant des paramètres du gameplay en cours de jeu.\\
Par exemple, Andrade et al utilisent l’apprentissage dynamique pour ajuster la difficulté. L’algorithme consiste à utiliser une base de couples (action, état de jeu) associés chacun à une valeur d’efficacité, afin que le jeu choisisse pour chaque situation un comportement de l’efficacité souhaitée.

		\paragraph{\emph{Adaptation de la difficulté dans les jeux sérieux} \\ \quad}
Contribuer à l'acceptation et à l'utilisation des jeux sérieux constitue un enjeu majeur pour la réussite et l'efficacité de ceux-ci. En effet, ces systèmes sont destinés à satisfaire les joueurs-apprenants et à répondre à leurs besoins en termes d'acquisition de compétences et/ou de divertissement. L’adaptation a pour but d’améliorer l’utilisabilité d’un jeu sérieux ou ludique en restructurant certaines de ses propriétés.

\paragraph{}
[Hocine et al, 11] se proposent d'étudier les différents systèmes d'adaptation dans les jeux ludiques et sérieux. Pour évaluer ces systèmes, ils définissent trois critères majeurs :
\begin{itemize}
	\item L’efficacité d’un système évalue le degré de succès avec lequel les utilisateurs réalisent leurs objectifs dans le système.
	\item L’efficience évalue les moyens mis en œuvre par les utilisateurs pour accomplir leurs objectifs.
	\item La satisfaction évalue le niveau d’acceptation par les utilisateurs.
\end{itemize}

\paragraph{}
Afin d'évaluer et de comparer les différents systèmes, ils utilisent un système d'évaluation des techniques d'adaptation intéressant et très parlant du fait qu'il repose sur un modèle MVC (voir figure \ref{criteres_adaptation}).

\begin{figure}[!hbtp]
	\centering
	\includegraphics[width=1\linewidth]{images/criteres_adaptation.png}
	\caption{Critères d’analyse des techniques d’adaptation [Hocine et al, 2011]\cite{Hoci11}}
	\label{criteres_adaptation}
\end{figure}

	\paragraph{Périmètre d'adaptation\\}
Le périmètre de l'adaptation identifie le périmètre dans lequel l'adaptation est appliquée, selon le modèle MVC. 
\begin{itemize}
	\item \emph{Adaptation de la présentation (la Vue)}. L'adaptation peut donc avoir lieu au niveau de l'interface, du son ou de feedbacks envers l'utilisateur. Dans Hammer \& Planks, nous avons ainsi rendu possible la modification de ces paramètres : ajustement du volume de la musique ou des bruitages, contraste, taille des objets, vitesse de jeu et possibilité de désactiver les éléments cosmétiques facultatifs au jeu (animation de la mer, effets visuels etc.).
		\item  \emph{Adaptation du contrôle} : ce niveau englobe les règles du jeu et les règles métier qui spécifient la dynamique du jeu (ou le gameplay) en réaction aux actions des joueurs. C'est dans ce niveau que l'on adaptera le niveau de difficulté du jeu. Suite à mon travail sur le jeu Hammer \& Planks, il est possible de modifier le nombre et le type d'ennemis que doit affronter le joueur, la vitesse du jeu, les propriétés des personnages (joueur ou ennemis) ou encore la fréquence des obstacles par exemple.
		\item  \emph{Adaptation du contenu (le Modèle)} : l'adaptation à ce niveau modifie dynamiquement soit les schémas de données utilisés ou bien le contenu. L'adaptation s'efforce donc de produire un contenu lié au contexte de jeu et aux compétences des joueurs. Nous pouvons citer à titre d'exemple la génération automatique des dialogues et textes narratifs (Barry G., 2007) ou d'ambiance sonore (Chen Y et al, 2006)
\end{itemize} 

	\paragraph{Paramètres d'adaptation\\}
Ce sont les éléments sur lesquels repose la prise de décision du processus d'adaptation. Ces éléments vont être utilisés soit comme déclencheurs de l'adaptation soit comme sources de données. Hocine et al distinguent deux types de paramètres en fonction de l'utilisateur~:
\begin{itemize}
	\item \emph{Modèle utilisateur} : ensemble de variables et métriques décrivant les caractéristiques de l’utilisateur dans le système. Ces caractéristiques peuvent être des données représentant les préférences de l’utilisateur, son état attentionnel, ses émotions et/ou ses compétences. Ces données sont stockées dans le profil de l’utilisateur qui sera utilisé comme paramètre de processus d’adaptation.
	\item \emph{Paramètre non-utilisateur ou variable système }: ce paramètre représente les variables propres au système et qui ne dépendent pas du modèle utilisateur. A titre d'exemple, nous pouvons citer les paramètres liés à la configuration matérielle et logicielle du système hôte.
\end{itemize}

	\paragraph{Modèle d'adaptation\\}
	L’adaptation peut être implémentée dans le système sous forme d’un module qui interagit avec le système pour modifier son comportement et sa structure. Ce module peut être~:
\begin{itemize}
	\item \emph{Implicite} : dans ce cas les procédures d'adaptation se retrouvent éparpillées et étroitement liées aux différents composants du système. Il serait dans ce cas difficile de séparer dans les instructions (ou le code source) les éléments qui incombent à l'adaptation des autres aspects.
	\item \emph{Explicite }: la technique d'adaptation utilise des modèles explicites comme un moteur de règles logiques, une matrice de décision ou des algorithmes d’IA.
\end{itemize}
	
	\paragraph{Adaptation Mono ou Multi-joueurs\\}
Contrairement aux jeux mono-joueur, l'adaptation dans un système multi-joueurs doit prendre en compte l'aspect collaboratif et l'hétérogénéité entre les joueurs tout  en maintenant une cohérence globale du jeu.
	
	\paragraph{\emph{Tour d'horizon de systèmes d'ajustement dynamiques dans les jeux vidéo}	 \\ \quad}
Durant son travail sur la difficulté dans les jeux vidéo, Levieux se propose d'étudier les différentes solutions d’équilibrage dynamique qui ont été mis en place dans le jeu vidéo. Bien que de tels systèmes, de part leur aspect automatique, ne nous seront pas directement utiles, il peut être intéressant d'en connaître les principales mises en œuvre. On distingue plusieurs types de solutions~: 
\begin{enumerate}
	\item l’apprentissage par renforcement [Sutton 98]~:
	\begin{itemize}
		\item jeux de combat : [Andrade 05], [Graepel 04]
		\item RTS : [Madeira 04], [Madeira 06], [Ula 05]
		\item FPS : [Lee-Urban 08]
	\end{itemize}
	\item le scripting dynamique, qui calcule des préférences à partir de règles écrites par le designer~:
	\begin{itemize}
		\item jeux d’aventure : Neverwinter Nights - Bioware
		\item RTS : Wargus [Spronck 05], [Spronck 06], [Spronck 08], [Timuri 07], [Ludwig 07]
	\end{itemize}
		\item évolution génétique et réseau de neurones (l’un et/ou l’autre)~:
	\begin{itemize}
		\item jeux d’actions [Demasi 05], [Spronck 02]
		\item RTS : [Ponsen 05], [Agogino 00]
		\item FPS : [Cole 04], [Thurau 03]
		\item jeux de sport : Fifa 99 -EA Games [Chan 04]
		\item puzzle : Tetris - Nintendo [Bohm 05]
		\item réalité virtuelle : [Yannakakis 09], [Yannakakis 07]
	\end{itemize}
		\item algorithmes de champs potentiels pour comportements stratégiques dans les FPS [Thurau 04]
		\item raisonnement au cas par cas dans les RTS : [Aha 05]
\end{enumerate}

	\subsubsection*{Comparaison avec les besoins de NaturalPad}
Dans leur état de l'art des techniques d'adaptation dans les jeux vidéo, [Hocine et al] définissent la différence entre les systèmes adaptables et les systèmes auto-adaptatif. Leur étude, comme celle de Levieux, se concentrent cependant sur les systèmes dont l'ajustement est automatique. Or, comme nous l'avons déjà dit, nos besoins et propositions sont plutôt de proposer des jeux mettant en place un système d'ajustement manuel. Un tel système permet à chaque thérapeute amené à utiliser un jeu sérieux pour la santé de notre environnement, d'adapter le jeu aux besoins thérapeutiques dont il aura précisément besoin selon le contexte : préférences thérapeutiques, pathologies et spécificités du patient ou encore état de santé ou de forme de celui-ci pour ne citer que quelques exemples.
		%%PART 4 : REHABILITATION			
	\subsection{La réhabilitation}
Un autre aspect est la dimension médicale de la rééducation. Connaître les enjeux, les contraintes, le contexte médico-social d'une réhabilitation ainsi que les techniques existantes ou les jeux thérapeutiques existants s'avéraient donc nécessaire.
	
		\subsubsection{Adaptation de la difficulté dans une rééduc fonctionnelle aujourd'hui (exercices de récup sensorielle, de mouvements, etc)}
		%% PART 6 : METHODES DE CONCEPTION	
	\newpage
	\subsection{Méthodes de conception}
		\subsubsection{Participative design : conception participative}
La conception participative est une méthode de travail utilisée principalement en conception de logiciel interactif. Sa principale caractéristique est la participation active des utilisateurs au travail de conception. Il s'agit donc d'une méthode de conception centrée sur l'utilisateur où l'accent est mis sur le rôle actif des utilisateurs \cite{wiki:cp}.

\paragraph{}
Il existe dans la littérature de nombreuses variantes de la méthode et de nombreuses techniques utilisées pour impliquer efficacement les utilisateurs. On peut noter particulièrement~:
\begin{itemize}
    \item l'observation et entretiens
    \item la production de scénarios
    \item le brainstorming
    \item le prototypage papier
    \item le prototypage vidéo
\end{itemize}

\paragraph{}
Une première séance de conception a lieu en début de projet : celle-ci regroupe les chercheurs et les développeurs de l'application, mais aussi les utilisateurs futurs ou potentiels. Intégrer les utilisateurs au processus de conception permet d'entrevoir au mieux leurs besoins et d'éviter un maximum d'erreurs d'interprétation ou d'oublis (voir illustration figure \ref{projet_info}). Les utilisateurs aident aussi à définir les problèmes éventuels et leurs possibles solutions.
\begin{figure}
	\centering
	\includegraphics[width=16cm]{images/projet_info.jpg}
	\caption{Allégorie d'un projet informatique}
	\label{projet_info}
\end{figure}

\paragraph{}Par ailleurs, plusieurs séances peuvent avoir lieues tout au long du projet, afin de vérifier si les besoins ont évolués et si le développement actuel est effectivement en accord avec ceux-ci. Les utilisateurs aideront aussi l'équipe de recherche et développement à juger de la pertinence des solutions apportées.
\paragraph{}
D'un point de vue du développement, on notera que ce type de conception s'accorde parfaitement avec une méthodologie AGILE, et notamment la méthode SCRUM qui consiste en de courtes périodes de développement entre lesquelles on met en relief l'avancement par rapport au projet global.

		\subsubsection{Innovation games}
	Product Vision on : \\	
	http://www.joelonsoftware.com/articles/jimhighsmithonproductvisi.html \\
	Vaut le coup de faire un résumé?	
		\subsubsection{Impact mapping}
L'impact mapping est une technique de planification stratégique qui permet aux entreprises de ne pas s'égarer durant les phases de développement ou de livraison de projets, en identifiant clairement les hypothèses et en se concentrant sur l'impact que doit avoir le livrable, à la base du projet.

\paragraph{}
Les logiciels et projets fonctionnent en relation avec leur environnement. Ce sont des relations dynamiques et interdépendantes avec leurs utilisateurs, d'autres projets, l'entreprise et plus largement avec tout un écosystème. Pourtant, les méthodes actuelles de planification reposent soit sur le fait que cet environnement reste inchangé, soit ne permettent pas de donner une vision de haut niveau. L'impact mapping permet de visualiser la relation dynamique entre les livrables et leur environnement, capturant les hypothèses les plus importantes en même temps que le périmètre à livrer\cite{Adzi12}.		

\paragraph{}L'impact mapping contribue à réduire le gaspillage en évitant de trop élargir le périmètre fonctionnel ou de complexifier inutilement les solutions techniques à appporter. Cette technique permet de se concentrer sur les livrables en les mettant dans le contexte d'impact qu'ils sont censés avoir sur leur environnement. Elle améliore la collaboration en créant une vue haut-niveau que les responsables métier et les équipes de développement peuvent utiliser pour une meilleur priorisation, et comme une référence pour une mesure des progrès réalisés ayant vraiment un intérêt. Enfin, elle permet de s'assurer que les vrais objectifs métier sont atteints (ou que des projets peu réalistes sont stoppés avant qu'ils ne coûtent trop cher) en communiquant clairement les hypothèses qui ont justifiées ces projets, et en permettant tout au moins de les tester.

\paragraph{}L'impact mapping possède de nombreux avantages~:
	\begin{itemize}
	    \item Elle facilite la participation de groupes de personnes venant de métiers différents, aidant ainsi à tirer parti de la sagesse des foules.
	    \item Elle permet de visualiser des hypothèses. L'impact mapping permet aux équipes de prendre de meilleurs décisions dans un environnement changeant constamment comme les nouvelles technologies. La nature visuelle de cette méthode permet de tenir des réunions efficaces et renforce les réflexions haut-niveau, ce qui permet d'aligner les parties prenantes du projet.
	    \item C'est une méthode rapide. Pour cette raison, cette méthode s'insère parfaitement dans un processus de livraisons itératif qui devient courant de nos jours dans le domaine du logiciel.
	\end{itemize}
	
	\paragraph{\emph{Carte d'impact}\\}
Une carte d'impact est une visualisation du périmètre et des hypothèses sous-jacentes, créées de façon collaborative par des personnes senior, techniques et métier.
C'est une mind map développée durant une discussion et facilitée par le fait de répondre aux questions suivantes :
		\begin{itemize}
			\item \emph{Pourquoi ?} C'est la question la plus importante : Pourquoi faisons-nous ce que nous faisons? C'est l'objectif que nous voulons atteindre.
			\item \emph{Qui?}
Qui peut provoquer le changement attendu ? Qui peut l'empêcher ? Qui sont les clients ou utilisateurs de notre produit ? Qui sera impacté par celui-ci? Ce sont les acteurs qui peuvent influencer les résultats escomptés.
			\item \emph{Comment?}Comment le comportent des acteurs doit-il changer? Comment peuvent-ils nous aider à atteindre notre objectif? Comment peuvent-ils empêcher l'objectif de se réaliser? Ce sont les impacts que nous voulons créer.
			\item \emph{Quoi?}Que pouvons-nous faire, en tant qu'entreprise ou équipe de livraison, pour encourager ses impacts ? Ce sont les livrables, les caractéristiques du logiciel ou les activités de l'entreprise.
	\end{itemize}
\begin{figure}[htbp]
	\centering
	\includegraphics[scale=0.6]{images/impact_map.png}
	\caption{Carte d'impact}
	\label{impact_map}
\end{figure}

	\part{Annexes}\appendix	
	\addcontentsline{toc}{section}{Types de jeux vidéo classés par Gameplay}
		%\section{Types de jeux vidéo classés par Gameplay}
\pagebreak
\makeatletter \label{types_jeux}
\includepdf[pages=-,picturecommand={%
            \setlength{\@tempdimb}{.5\paperwidth}%
            \setlength{\@tempdimc}{3.5cm}%
            \setlength{\unitlength}{1pt}%
            \put(\strip@pt\@tempdimb,\strip@pt\@tempdimc){%
        \makebox (0,0){\thepage}}%
}]{chapters/types_jeux_video}
\makeatother
	%\section{Esquisse d'énumération des éléments d'un jeu et association avec les théories cognitives et comportementales}
%	\label{matching}
%	\includepdf[pages=-]{chapters/matching_objets_et_theories_v1}
%	
%	%\section{Proposition de contrôles naturels pour des jeux existants}
%	 \label{propositions_controles}
%	\includepdf[pages=-]{chapters/propositions_controles}

	
%Voir dans quelle mesure il peut être intéressant de mettre mes comptes rendus d'entretien. Dans des annexes séparées éventuellement.
\end{document}