	\subsection{Réhabilitation}
Un autre aspect est la dimension médicale de la rééducation. Connaître les enjeux, les contraintes, le contexte médico-social d'une réhabilitation ainsi que les techniques existantes s'avérait donc nécessaire. Cette partie est la synthèse de mes entretiens et investigations dans ce domaine qui ne m'était encore que peu connu.

		\subsubsection{Contexte socio-médical : connaître l'AVC}
L'accident vasculaire cérébral, ou AVC, parfois aussi appelé attaque cérébrale, est un déficit neurologique soudain d'origine vasculaire causé par un infarctus ou une hémorragie au niveau du cerveau. Les symptômes peuvent être très variés d'un cas à l'autre selon la nature de l'AVC ou l'endroit et la taille de la lésion cérébrale, ce qui explique un large spectre : aucun signe remarquable, perte de la motricité, perte de la sensibilité, trouble du langage, perte de la vue, perte de connaissance, décès, etc.	

		\paragraph{\emph{L’accident vasculaire cérébral (AVC) est un événement de santé fréquent}\\}
Dans les pays occidentaux, près d'un individu sur 600 est victime d'un AVC chaque année, soit près de 120 000 accidents par an en France. L'AVC est une cause fréquente d'hémiplégie chez les victimes et reste une des principales causes d'invalidité. \\
La récupération des fonctions motrices, de la parole ou de la compréhension dépendent pour beaucoup de l'âge du patient et de son atteinte au niveau du cerveau.

\paragraph{•}L’âge moyen de survenue d’un AVC est de 73 ans : 70 ans pour les hommes et 76 ans pour les femmes.

\paragraph{•}   Les données préoccupantes concernant l’augmentation des AVC dans la population des moins de 65 ans concernent plus les femmes (+ 12,9\%) que les hommes (+9,7\%). Ce phénomène épidémiologique préoccupant est à examiner dans un contexte de diminution de l’espérance de vie en bonne santé (sans incapacité) qui est passée de 62,7 ans à 61,9 ans pour les hommes et de 64,6 ans à 63,5 ans pour les femmes, en seulement deux ans (entre 2008 et 2010)

\paragraph{\emph{L’accident vasculaire cérébral (AVC) est un événement de santé grave}\\}

	\paragraph{En termes de mortalité :}
    L’AVC représente la troisième cause de mortalité pour les hommes (après les cancers « de la plèvre, de la trachée, du larynx ou des poumons » et les cardiopathies ischémiques) et la première pour les femmes avant le cancer du sein.\newline
    En 2008, l’AVC a été responsable de 33 000 décès.

	\paragraph{En termes de handicap :}
Dans les pays occidentaux, l’accident vasculaire cérébral est une cause majeure de handicap acquis de l’adulte, la deuxième cause de démence après la maladie d’Alzheimer et la troisième cause de mortalité.	\newline

    L’AVC est souvent responsable de séquelles qui affectent la qualité de vie des patients. 

 \paragraph{•}
Les atteintes peuvent être motrices, sensitives, sensorielles et cognitives (avec notamment des troubles de la mémoire) 

\paragraph{•}Un mois après l’AVC, pour les personnes ayant survécues, les dépressions sont fréquentes et il est à noter que seulement 41\% d’entre elles n’ont plus de symptômes, ainsi~:
\begin{itemize}
	\item 25\% présentent un handicap léger ou modéré
	\item 34\% ne peuvent marcher sans assistance.
\end{itemize}
 
		\paragraph{L'AVC et Hammer \& Planks\\}
Hammer \& Planks est né d'un projet d'une étudiante en ergothérapie dont le but était de proposer un jeu servant à travailler l'équilibre chez des personnes hémiplégiques ayant été victimes d'un AVC. Aujourd'hui, H\&P peut être utilisé aussi bien pour travailler son équilibre, ses membres supérieurs son tronc ou sa capacité d'attention.   	
	\paragraph{}
Lors de mes six mois de stage, j'ai ainsi été en relation avec plusieurs docteurs en médecine, des kinésithérapeutes ainsi que des ergothérapeutes.

	\subsubsection{Enjeux et objectifs thérapeutiques de la réhabilitation} \label{objectifs_therapeutiques}
Le principe de base est de faire un bilan initial.  A partir de ce dernier on dégage des objectifs  de rééducation. Pour atteindre ces objectifs, il existe des moyens de rééducation (exercices, physiothérapie, jeux vidéos, etc.) qu'il faut adapter au patient.

\paragraph{Les grand axes de rééducation}
\begin{itemize} 
	\item Récupérer la commande motrice 
	\item Diminuer la douleur
	\item Assouplir les muscles 
	\item Récupérer la sensibilité 
	\item Renforcement musculaire 
	\item Transfert du poids du corps 
	\item Travail de l'équilibre bipodal / unipodal 
	\item Travail du schéma de marche
\end{itemize}

	\textbf{\emph{Définitions}\\}
• \emph{La spasticité} consiste en un étirement rapide d'un muscle qui entraîne trop facilement sa contraction réflexe qui dure un certain temps. \\
• \emph{L’hypertonie spastique} (musculaire) est une contraction réflexe du muscle qui s'oppose à l'étirement.\\

	\textbf{\emph{Signifiant VS significatif}\\}
L’aspect signifiant/significatif est un terme utilisé en ergothérapie pour définir le sens qu'a une activité auprès d’un patient.\\
• Une activité peut être qualifiée de \emph{\textcolor{orange}{significative}} quand elle revêt un \textcolor{orange}{sens social commun}. Par exemple dans notre cadre, les jeux vidéo et notamment la Wii, peuvent être significatifs pour les personne âgées de 10 à 30 ans, car ils sont beaucoup utilisés dans cette tranche d'âge.\\
• Une activité est dite \emph{\textcolor{vert}{signifiante}} quand elle a du sens pour la personne de \textcolor{vert}{manière personnelle}. Par exemple, faire la vaisselle est \textcolor{orange}{significatif }car on \textcolor{orange}{doit} la faire. Mais en général les personnes n’aiment pas la faire et elle est donc pour eux non signifiante. Alors que ‘moi’, c'est une activité que ‘j’’apprécie :  elle a donc pour ‘moi’ un sens motivationnel et est alors signifiante.

	\paragraph{\emph{Cas de l’hémiplégie} \\}
C’est une pathologie qui n’est pas évolutive, on ne peut que récupérer. Cette récupération est différente chez tous les patients, en fonction de leur âge, leur condition physique, de la gravité de la pathologie et de la rééducation mise en place.

	\subsubsection{Cas de l'hémiplégie}
\paragraph{\emph{Mouvements analytiques / Motricité fine} \\ }
Les hémiplégiques vont présenter des troubles de la sensibilité et des troubles de la commande volontaire. Un problème nerveux peut entraîner une augmentation du tonus musculaire pouvant provoquer une spasticité ou une hypertonie. 

\paragraph{}
Parmi les objectifs, on va donc tout d’abord chercher à récupérer une commande motrice normale, puis à rééquilibrer les capacités (musculaires et nerveuses) et enfin renforcer les capacités musculaires. Dans la pratique, le patient va donc tout d’abord chercher à retrouver un mouvement précis, puis à être capable de l’effectuer de manière fluide et plus rapide, puis à pouvoir mettre de la force dans son geste, pour vaincre la gravité ou supporter un poids/une résistance par exemple.

\paragraph{}
Pour ‘vaincre’ la spasticité du patient, le praticien doit pratiquer des manipulations sur le patient pour étirer ses muscles. Il est donc difficilement envisageable de gamifier cet aspect de la rééducation.\newline
Déficit léger : travail en 3D possible     \quad$ \rightarrow $\quad       Épaules\\
Déficit lourd : travail en 2D uniquement  \quad $\rightarrow $\quad      Poignets et mains

	\paragraph{Paramètres\\}
Récupération de la commande volontaire, amplitude articulaire, spasticité, phase aiguë VS phase chronique( $ \rightarrow $ spastique)\\
Plasticité cérébrale : intensive et ciblée. \newline

Les thérapeutes veulent~: 
\begin{itemize}
	\item des mouvements répétitifs fins
	\item un type précis de mouvement (rotation par exemple)
	\item pouvoir paramétrer les exercices
	\item lâcher / écarter (voir le système PABLO)
\end{itemize}

	\paragraph{\emph{Sensibilité} \\ \quad}
Exercice : le patient ferme les yeux, le thérapeute lui donne un petit objet dans la main. Que ressent-il? Quelle texture, quelle forme? \\
En fait, on lui présente avant l’exercice une série d’objets (éponges de différentes tailles et souplesses, figurines en bois de différentes formes). Puis, yeux fermés, il doit essayer de reconnaître l’objet que lui présente le thérapeute en identifiant ses propriétés haptiques et tactiles.
\paragraph{} 
Ce travail de la sensibilité est particulièrement important pour les mains, riches en capteurs tactiles. Il est par ailleurs nécessaires de travailler cet aspect pour que le cerveau “n’oublie pas” en réutilisant les connexions neuronales spécifiques. Par ailleurs, on peut aussi travailler la sensibilité d’autres membres comme les pieds, ce qui est particulièrement important notamment dans le cadre de la marche où la perception des aspérités/irrégularités du sol est importante.\\
Cet aspect de la rééducation est complémentaire de celui de la récupération de la préhension. Le patient aura ainsi la capacité (commande, force) de tenir un objet, mais le lâchera faute d’informations sensorielles suffisantes.

	\paragraph{\emph{Équilibre} \\ }
Équilibre assis d’abord si la personne n’est pas capable de se tenir debout. Si son équilibre est vraiment précaire, il peut être nécessaire de placer le patient au milieu d’un cercle de maintien, ou de surveiller la personne.\\
On peut ensuite passer à un travail de l’équilibre debout, en diminuant progressivement l’écart entre les deux pieds. Un autre facteur de difficulté va être de fermer les yeux, le patient doit être capable de maintenir son équilibre.

	\paragraph{\emph{Diminuer la douleur} \\ }
Pour cela, plusieurs moyens. On peut prescrire des traitements : au niveau neurologique, articulaire ou musculaire ; de la physiothérapie ou une mobilisation passive (manipulation kiné). Mais pas de jeux.
Pour les lombalgiques : en cas de crise aiguë, du repos, puis rapidement refaire de l’exercice.
En phase chronique, augmentation du tonus musculaire et renforcement musculaire.

	\paragraph{\emph{Transfert de poids} \\ }
Chez les hémiplégiques gauche, le centre de pression va se situer à droite. On va donc chercher à déplacer le centre de pression du patient vers la gauche jusqu’à atteindre le milieu du corps, car il a la force et la commande. Voir équilibre.

	\paragraph{\emph{Renforcement musculaire} \\ }
A première vue, peu adapté à l’hémiplégie, on visera plutôt à retrouver un mouvement, une commande motrice (ou alors en fin de rééducation).

	\paragraph{Paramètres \\}
Fréquence, amplitude, durée du maintien, nombre de mouvements réalisés / répétitions, vitesse, fluidité.

\paragraph{}En phase finale de récupération : endurance, périmètre de marche, performance : vitesse (de marche), marche sur terrain plat ou accidenté, contrôle moteur, coordination, travail dans les escaliers, relevé du sol (fonctionnel), retournement et transfert (passage du lit à assis, de assis à debout et inversement).

	\subsubsection{La réhabilitation au quotidien}
Dans le cadre de mon travail sur le projet Hammer \& Planks, j'ai réalisé plusieurs séances d'observations de séances de rééducations chez des kinésithérapeutes. Afin de mieux cibler les attentes et besoins à la fois des thérapeutes et des patients, il me semblait primordial d'assister directement à ces étapes de la réhabilitation. Je souhaite ici partager mon expérience au travers le compte rendu d'une demie journée d'observation chez le kinésithérapeute Didier Costeau, exerçant à Montpellier. Ce résumé est important en ce qu'il permet de saisir les enjeux sociaux, médicaux et techniques de la réhabilitation. M. Costeau a ça de particulier qu'il utilise dans son métier des jeux vidéo classiques dont il dérive l'utilisation dans un but thérapeutique.

	\subsubsection*{Organisation}
Une grande salle remplie de machines, dans laquelle les patients viennent en groupe tout au long de la journée. Les patients sont majoritairement hémiplégiques ou tétraplégiques, et se déplacent pour beaucoup en fauteuil.

\paragraph{}
D. Costeau passe d'un patient à un autre pour les préparer sur la banquette ou sur une des machines à sa disposition. Il dispose entre autre d'appareils de musculation (traction, pectoraux, etc), de vélos, d'un stepper ou d'un Huber (machine permettant de faire travailler l'équilibre, 80 muscles du corps), ainsi que d'un Segway. Il les aide donc à se positionner sur la machine et éventuellement à la régler (poids, attaches).\\
Puis, selon les exercices à réaliser, M. Costeau propose de réaliser cet exercice à l'aide d'un jeu vidéo en utilisant des consoles et accessoires de la Wii. En attachant une wiimote sur le stepper ou un vélo, le patient va être capable de profiter de l'environnement ludique du jeu vidéo tout en faisant ses exercices de rééducation. Les patients peuvent aussi jouer à plusieurs avec le jeu de tennis de Wii Sport, en tenant directement la wiimote dans leur main.\\
Par ailleurs, il utilise aussi la balance de la wii board, en ayant mis en place un système permettant de poser une planche sur la board, afin qu'une personne en fauteuil puisse y monter et l'utiliser. Cela permet de travailler l'équilibre ou la force des jambes d'un patient. Une variante propose même 2 bâtons de ski, afin qu'une personne ne pouvant pas beaucoup se pencher puisse s'aider des bâtons et s'appuyer dessus. Dernière variante, accrocher une wiimote au dessus d'un casque, système notamment utilisé pour les personnes présentant une déficience motrice cérébrale, pour leur faire travailler leur port de tête.\\
A chaque fois, l'idée est évidemment de proposer un jeu afin d'offrir un feedback, ludique qui plus est, un suivi des progrès et une ambiance décontractée.

\paragraph{}Autre fait notable, de nombreuses cordes de tension traversent la pièce ; elles ont été installées afin de pouvoir garder le dos des patients cambré. En effet, le dos a besoin d'être cambré, mais la colonne a tendance à se tasser à force de rester assis; fait malheureusement inévitable chez les personnes en fauteuil.

\paragraph{}Dans les pratiques plus classiques, j'ai pu assister à des manipulations directes du praticien sur les patients (travail sur les membres inférieurs ou supérieurs : extensions, flexions, jeu de force) ou à l'aide au réapprentissage de la marche, en mettant un patient debout et en le maintenant tout au long d'une marche.

	\subsubsection*{Remarques, ressentis (mots clefs) }
		\paragraph{\emph{Aspect Social – Jouer ensemble}\\}
Première impression en arrivant : les séances se déroulent en groupe, et tout le monde \textbf{communique} !
Les gens se parlent, blaguent, jouent ensemble si c'est possible, etc. \\
C'est une constatation évidente, et cela m'a été confirmé oralement après, les patients préfèrent être en groupe : "On ne viendrait pas sinon". En arrivant, une personne jouait au tennis avec la Wii, et dès son arrivée, une patiente a voulu la rejoindre pour partager une partie multijoueur. Malgré la différence de niveau, tout le monde y trouvait du plaisir et y mettait du sien. Même les non joueurs se sentaient impliqués et commentaient la scène. \\
A mon avis, c'est un élément très important à prendre en compte. Il faut tout de même penser à intégrer le fait que les patients n'ont ni le même handicap, ni a priori le même niveau de joueur dans un jeu particulier.

\paragraph{}Une suggestion qui tient à cœur le kiné, car très importante, serait la possibilité de jouer à plusieurs même à des endroits différents. Par exemple deux patients dans des chambres différentes, mais qui pourraient tout de même jouer une partie ensemble.

		\paragraph{\emph{Les Jeux : gamification de la thérapie}\\}
De manière générale, les patients, l'âge ayant l'air d'un critère peu discriminant, ont clairement adopté l'usage des jeux vidéo pour leur séance de kiné. De la jeune fille ou jeune adulte, au monsieur de près de 80ans, les patients jouent, et aiment ça. C'est à la fois surprenant et non. Si les dernières générations ont toujours connu les divertissements numériques, ce n'est pas le cas de la plupart des patients qui viennent au cabinet. Bien que peu voir pas à l'aise avec la technologie, ils reconnaissent aisément que cela peut être divertissant et se prêtent vite au jeu de la comparaison des résultats par exemple.
\begin{quotation}
Ex : un patient de 78 ans (militaire, strict, sportif), qui "ne connaît pas Facebook, Google, Twitter et iPad, n'a pas d'ordinateur et n'en veut pas", apprécie les jeux et la console, car "Ça vous gnaque ! J'ai fait 8km, à mon âge !" après une séance de stepper avec la Wii.    
\end{quotation}
Au niveau des performances, on peut citer un autre exemple, d'un patient utilisant un vélo équipé d'une Wiimote (cf partie feedback) : en l'observant, j'ai pu remarqué qu'il avait réalisé la majorité du temps de course les yeux fermés, comme pour se concentrer. Cependant, sur les 5 dernières minutes des 30, il a réouvert les yeux et j'ai pu constater que son allure augmentait alors significativement (de l'ordre de 30\%). On peut supposer que le feedback donné par le jeu n'était pas complètement étranger à cette différence (bien que probablement pas le seul facteur).    
Une autre patiente utilisant la wii balance board avec son fauteuil pour jouer au snowboard, m'a confié que cela la lui permet de travailler aussi bien, avec une fatigue équivalente, mais que c'est « plus sympa »(bien que répétitif aussi à la longue) et « donne un objectif » (celui du jeu).

\paragraph{}Pour nuancer ces propos, tous les patients n'ont cependant pas joué pendant leur séance ce jour là. Un monsieur dont l'hémiplégie était importante m'a expliqué qu'il ne pouvait pas jouer, aucun jeu ne pouvant s'adapter à ses capacités motrices réduites, malgré les nombreux arrangements qu'a réalisés Didier Costeau. Pour d'autres, ce n'était tout simplement pas prévu/possible de part la nature des exercices qu'ils devaient réaliser. Enfin, au moins une personne ne semblait pas du tout intéressée, mais je n'ai pas réussi à en connaître la raison .

		\paragraph{\emph{Diversité – Variété}\\}
C'est LE problème décrit par tous les patients et le kiné : le manque flagrant de diversité, de représentations. Diversité des jeux, des niveaux, des bonus, de musiques (tout le monde se plaint !), d'obstacles, d'exercices, de la difficulté, etc.

		\paragraph{\emph{Cohérence - Adaptation}\\}
Autre problème actuellement, parfois un problème de cohérence entre les gestes / mouvements du patient et le feedback du jeu. Notamment pour l'instant, un vélo customisé pour pouvoir être utilisé par une personne dans un fauteuil, vélo auquel on rajoute deux bras elliptiques sur lesquels on accroche une wiimote. La wiimote effectue donc des mouvements verticaux, et actuellement, le jeu associé est un jeu de course à pied. Malheureusement, la vitesse de mouvement de la wiimote fait que l'avatar du patient court extrêmement lentement, sans cohérence avec sa performance à vélo.
Il faut donc diversifier l'offre pour s'adapter au besoin, en proposant gameplay et feedback cohérents. Avoir une solution de jeu pour chaque type de mouvements / exercices que l'on souhaite faire. Cela passe par l'utilisation d'un panel de périphériques plus large.

\paragraph{}Dans le même domaine, un problème reste le temps passé par le soignant à configurer les jeux, matériels et parties. Normalement, la configuration matérielle se fait une fois pour toute (lors de ma séance d'observation, il a fallu le refaire suite au changement de la console), mais c'est un point à penser pour éviter de faire perdre du temps dans le cas ou cela doit se reproduire.
Certains patients sont capables de choisir et lancer eux même leur jeu ainsi qu'une partie, mais ce n'est pas le cas de tous, et le soignant doit alors intervenir régulièrement. La navigation dans les menus est difficile. Il faudrait imaginer un système permettant de s'affranchir de cette aide, notamment si on entre dans l'hypothèse où les patients jouent chez eux sans personne alentours pour les aider (ou dans une optique de recherche d'autonomie). Cela peut cependant se révéler difficile pour les patients les plus atteints (capacités motrices extrêmement limitées, locution fortement diminuée). Cette remarque est d'autant plus importante qu'un certain nombre de patients ont acheté chez eux la Wii (c'est quelque chose à vérifier sur une échelle plus importante) afin de pouvoir s'entraîner chez eux.

\paragraph{}Le soignant veut aussi pouvoir adapter facilement la difficulté d'un exercice en fonction du patient. Avec les appareils de musculation, il peut par exemple changer la valeur de poids à soulever ou la tension exercée, et l'adapte pour chaque patient.
Dans ce genre là, s'inspirer du système de HUBER, qui offre un suivi des résultats pour chaque patient.

	\subsubsection*{Aspect Médical}

		\paragraph{\emph{Lien avec la vision}\\}
Vu aussi lors de mes recherches documentaires, le lien étroit entre la vision et les capacités motrices. J'ai donc voulu confirmer cette hypothèse auprès des patients et du kiné. Celui-ci m'a alors parlé du problème de plasticité cérébrale. Brièvement, cela représente l'aspect déformable du cerveau et sa capacité à modifier l'organisation de ses réseaux de neurones en fonction des expériences vécues par l'organisme. 

En plus d'avoir sa vision affectée par son AVC et l'hémiplégie, le patient va avoir tendance à délaisser son coté diminué, que ce soit physiquement ou visuellement. Il est donc important de faire travailler simultanément sa vue périphérique du coté atteint en même temps que les membres affectés, dans le but de recréer des connections neuronales pour re-développer sa vision.

		\paragraph{\emph{Importance des exercices sous-progressifs}\\}
Ce sont des exercices qui décomposent les mouvements et font travailler individuellement ces sous mouvements.

		\paragraph{\emph{Travail bilatéral}\\}
Toujours d'après les recherches dans le domaine, penser à l'importance d'un travail bilatéral dans la rééducation. Voir section \ref{bilateral}


		\paragraph{\emph{Shemes de diagonale}\\}
Les schemes de diagonale représentent le fait que la plupart des gestes que nous faisons (bras et jambes) se font sur un modèle de diagonale (car plus naturelle) dans la mesure du possible. Attraper un objet devant soi ou en hauteur (on croise le bras pour être bras tendu), la marche, le ski ou le snowboard, l'escalade, le tennis, etc.	
	
