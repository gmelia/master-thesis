	\subsection{\emph{Hammer \& Planks}, un serious game modulable}
Le projet \emph{Hammer \& Planks} fut une composante forte de mon stage. Il m'a servi de terrain d'expérimentation des différentes idées inspirées par les connaissances acquises au cours de ces six mois. J'ai ainsi travaillé aussi bien sur la version thérapeutique que sur la version grand public, dont les enjeux sont différents. C'est aussi un projet mettant un place un système d'ajustement des paramètres, permettant d'adapter sa difficulté et les mouvements des joueurs.

	\paragraph{}Par rapport à ce qui est déjà mis en place dans le jeu, il est possible d'imaginer de nouvelles fonctionnalités. Une approche à laquelle nous réfléchissons actuellement est la mise en place d'un \textbf{gameplay asymétrique}. \\
Cette proposition trouve son intérêt à la fois dans la version grand public et la version thérapeutique. L'idée est de permettre à un second joueur de participer à l'expérience de jeu en utilisant un périphérique tiers : tablette ou smartphone. Ce joueur pourrait par exemple incarner le vent pour aider ou perturber la navigation du joueur principal, jouer le rôle d'un membre de l'équipage pour donner des bonus ou encore se poser comme concurrent. 

\paragraph{}De manière générale, l'objectif est de permettre une interaction entre les joueurs et d'enrichir l'expérience globale de jeu. Nous avons vu dans notre background l'intérêt et la volonté des patients en rééducation de pouvoir partager des activités avec leurs proches ou d'autres patients.

\paragraph{} Une dernière proposition de fonctionnalité pour ajuster le niveau de difficulté, concerne la gestion des ennemis. Actuellement, le système offre la possibilité de régler la fréquence d'apparition des ennemis, ainsi que la densité de répartition des différents types d'ennemis, comme le montre la figure~\ref{interface_parametres}. \\
Afin d'apporter plus de finesse au système, je propose la possibilité de créer des bataillons d'ennemis. On pourrait alors par exemple définir la taille maximale d'un bataillon (nombre d'unités en son sein), le type des unités le composant ou encore un schéma de patrouille. De manière plus globale, il serait possible de préciser le nombre maximum de bataillons simultanés ou le temps d'apparition entre deux bataillons. Cela laisserait la possibilité pour le joueur, le designer ou le thérapeute d'établir de nouvelles stratégies. 

	\subsection{Proposer un système de paramètres prédéfinis}
Afin d'enrichir le système d'ajustement de la difficulté de \emph{Hammer \& Planks}, on pourrait proposer quelques configurations prédéfinies de paramètres, comme il se fait classiquement dans les jeux vidéo. Dans la version grand public, cela se traduirait par la proposition de plusieurs niveaux de difficulté ("facile", "moyen", "difficile").

\paragraph{}
La proposition est particulièrement intéressante pour la version santé du jeu, notamment couplée avec l'interface thérapeutique. \\
Il serait en effet beaucoup plus ergonomique et intuitif pour les thérapeutes de pouvoir directement choisir des ensembles de valeurs de paramètres dont l'application correspondrait à un objectif thérapeutique. Par exemple, plutôt que de devoir modifier manuellement tous les paramètres ayant un impact sur l'utilisation du bras gauche du joueur-patient, il lui suffirait de choisir l'ensemble prédéfini correspondant. On peut aussi parler de meta-paramètres.

\paragraph{} Bien sur, plusieurs ensembles pourraient correspondre aux mêmes besoins. Enfin, afin de permettre une adaptation la plus complète possible, il faudrait laisser la possibilité aux soignants de créer ou de modifier eux-mêmes ce type de configurations. Cela permettrait de faciliter l'utilisation de la plateforme et des jeux par les thérapeutes.

\paragraph{} Dans le cadre de l'adaptation aux besoins thérapeutiques, il est aussi envisageable de créer plusieurs modes de jeu spécifiques ou d'ajouter de nouvelles possibilités dans l'interface thérapeutiques. Par exemple, si l'on cherche à faire réaliser par le joueur-patient un certain schéma moteur (suite de modifications de son centre de gravité par exemple), il serait pratique pour le thérapeute de créer à la volée le parcours du jeu incitant le joueur à réaliser un tel schéma.


	\subsection{Proposer un système de recommandation}
Comme nous l'avons vu dans la partie~\ref{recommandation}, les systèmes de recommandation permettent d'orienter des personnes vers des produits qui correspondent à leurs attentes ou besoins. 

\paragraph{} En s'inspirant de ces modèles, il semblerait judicieux de pouvoir établir un système de profil pour les joueurs. En analysant les capacités du joueur et ses affinités, on pourrait alors nourrir un système de recommandation. Celui-ci pourrait orienter le joueur vers des jeux en accord avec ses affinités ou lui proposer des sets de paramètres personnalisés correspondant à ses critères de difficulté par exemple. L'ajout d'un système de suivi de ses résultats pour prendre en compte l'évolution du niveau du joueur et de ses affinités permettrait d'améliorer encore la qualité des propositions.

\paragraph{} D'un point de vue thérapeutique, un tel système intégrerait le profil médical du joueur-patient. La prise en compte de ses capacités cognitives et motrices, de la nature de sa déficience et des exercices préconisés par les soignants alimenterait efficacement un système de recommandation. Celui-ci viendrait améliorer l'adaptation des jeux sérieux pour le patient, ce qui contribuerait d'autant plus à sa réhabilitation. N'oublions pas que les jeux sérieux renforcent la motivation du patient, et que c'est un élément essentiel à sa rééducation.

\paragraph{}
Enfin, en évaluant le niveau du joueur durant ses sessions de jeu, il serait possible de mettre en place un système d'ajustement dynamique de la difficulté. Ce système viendrait en complément de la paramétrisation initiale proposée grâce à l'évaluation initiale du profil du joueur. Adaptations initiale et dynamique seraient ainsi deux moyens de proposer aux joueurs une expérience de jeu la plus en adéquation avec leurs envies (et éventuellement, leurs besoins thérapeutiques).
	
	\subsection{Étendre et éprouver méthodologie et outils de conception}
Si j'ai pu mettre en application la méthodologie proposée dans plusieurs projets de conception, je n'ai malheureusement pas eu le temps l'appliquer sur un projet complet. J'ai  moi-même mené une séance de conception participative pour un projet d'application d'aide à la verticalisation. Cette séance a mené à la création d'une carte d'empathie (voir annexe~\ref{empathie_elsa}), puis à la création de scénarios d'usage. Mais la fin de mon stage est survenue avant la possibilité de créer les storyboards, mais surtout avant la phase de développement, qui risque de ne survenir que bien plus tard. Il serait donc intéressant, afin d'éprouver la méthodologie, de la mettre en application sur un projet complet et de critiquer différents aspects comme le temps de conception, la qualité de l'application conçue, son impact sur les objectifs thérapeutiques ou la satisfaction des joueurs ou des thérapeutes.

\paragraph{}
Un autre aspect serait de pouvoir étendre le domaine de la méthodologie à d'autres pathologies et matériels. Nous nous sommes jusqu'alors concentrés sur des pathologies d'ordre moteur, et il serait judicieux de vérifier qu'elle est aussi pertinente dans le cas de pathologies mentales par exemple. Pour aller encore plus loin, étendre la méthodologie dans des domaines différents du monde médical, pour le développement de serious games éducatifs par exemple, serait une perspective intéressante à mettre en place.

\paragraph{} Enfin, comparer les résultats d'une telle méthodologie dans divers domaines avec ceux obtenus à partir de méthodologies plus classiques permettrait de juger de l'intérêt de la méthode, de possibles améliorations ou encore de trouver dans quels contextes elle est la plus pertinente. Un tel travail pourrait avoir lieu dans le cadre d'un stage ou d'une thèse portant sur des méthodologies de conception.

\paragraph{} Au niveau de l'ensemble d'outils d'aide à la conception, on peut envisager des perspectives d'évolution du travail. On trouverait ainsi un intérêt dans un document identifiant clairement pour chaque type de jeux vidéo un type de contrôle adapté. Cela permettrait d'orienter des concepteurs ayant un objectif thérapeutique précis à mettre place vers un ou plusieurs types de jeu en accord.

\paragraph{} On pourrait proposer pour chaque type de jeu un inventaire des \emph{objets} (feedbacks, menus, GUI, méthodes de contrôle, etc.) qui lui sont propres. A partir de cette liste, faire le lien entre ces objets et les différents processus cognitifs et comportementaux sur lesquels il agissent fournirait un outil puissant. Il pourrait être utilisé pour guider les concepteurs et développeurs à créer des serious games plus efficace d'un point de vue sérieux. Les créateurs du jeu DIAB\cite{Wils09} se sont ainsi inspirés des théories comportementales pour créer un jeu pour aider à lutter contre le diabète de type 2 et l'obésité infantile.