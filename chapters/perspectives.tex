	hammer and planks : (paramètres de difficulté, groupe d'ennemis)\newline
	
	-avoir une information de profil pour cibler la difficulté -> nourrir un systeme de recommandation\\
	-ajustement dynamique de la difficulté selon le profil/résultats du joueur\newline
	-étendre le domaine de la méthodologie à d'autre pathologies et matériels\newline
	-éprouver les propositions de méthodo, sur un cas concret et complet\newline
	-permettre un paramétrage par objectifs non par paramètres permettant de faciliter l'utilisation de la plateforme et des jeux par les thérapeutes \newline
	- Le pb c'est que j'ai rien qui dise clairement "pour tel exercice/objectif, fais plutot ce type de jeu avec tels contrôles. Du coup, est-ce que ca pourrait pas être une forme de perspective? disant que ça nécessiterait du travail et serait moins parfait qu'un truc personnalisé, mais ca permettrait de gagner du temps tout en étant mieux que ce qui existe actuellement
	A l'inverse, la démarche que j'ai emprunté dans ma méthode semble mieux coller aux besoins, mais nécessite plus de travail.
 
 bien expliquer si ces démarches sont différentes, complémentaires ou partiellement complémentaires, dans quels contextes elles s'appliquent et quels sont les compétences nécessaires à quels moments pour les mettre en place.
 Pensez aux perspectives de l'ordre de "ca serait bien d'éprouver la méthodologie sur un projet complet" , "il faudrait une équipe avec telles compétences"