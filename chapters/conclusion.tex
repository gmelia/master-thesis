Globalement, ce stage était sur le thème des échanges et au carrefour entre le monde informatique et le monde médical. Nous avons vu comment allier l'expérience d'une société dans le développement de jeux vidéo, à la connaissance et l'expertise de professionnels de la santé, pour proposer un jeu sérieux efficace et apprécié des joueurs.

\paragraph{}
Ce stage de fin d'étude fut aussi l'occasion de vérifier sur une période longue de six mois que mon niveau de compétence et mon autonomie de travail étaient suffisants. Si j'avais déjà réalisé un stage de deux mois chez NaturalPad, l'accent avait alors était mis sur le développement et l'acquisition de connaissances web. J'avais à cette époque parlé de mon intérêt pour les échanges avec des professionnels d'autres corps de métier. En ayant reparlé avec Antoine Seilles, nous avons axé mon travail de stage vers de la conception avec des thérapeutes. Cela m'a montré l'importance de la communication dans une équipe et nous a permis de trouver un sujet de stage en accord avec nos besoins et envies respectifs.

\paragraph{}
Durant ce stage, j'ai donc participé au développement du serious game Hammer \& Planks, et ma connaissance du moteur de Unity3d fut un atout. J'ai travaillé aussi bien pour sa version grand public, enrichissant mes compétences en game design, que sur la version thérapeutique d'aide à la rééducation motrice. Celle-ci me permit de tester et mettre en application un système d'ajustement des paramètres du jeu, dans le but d'agir sur la difficulté ou l'adaptation du jeu aux capacités du joueur-patient. Cette adaptation se fait au moyen d'une interface web thérapeutique permettant de modifier la valeurs des variables du jeu ou des paramètres de réglage des contrôles par exemple.

\paragraph{}
Nous avons aussi réalisé des séances de tests avec des thérapeutes et des patients représentant les utilisateurs futurs du jeu. Cela m'a permis de confronter mon travail aux critiques et appréciations de personnes complètement étrangères au travail de développement. Ce fut très constructif à la fois pour mieux comprendre les joueurs, mais aussi d'un point de vue humain. Voir une jeune fille de moins de 20 ans, hémiplégique, et se régalant de jouer à Hammer \& Planks, fut réellement touchant. De la même manière, s'entendre féliciter à plusieurs reprises par un homme de plus de 70 ans en disant que grâce à notre travail il trouvait enfin amusant une activité à l'hôpital, est en soi une récompense.

\paragraph{} C'est d'ailleurs l'aspect positif que je suis venu chercher et trouver en venant travailler chez NaturalPad. Je souhaitais utiliser mes connaissances et compétences informatiques dans un contexte différent que le développement d'applications classiques, et basé sur les échanges humains. NaturalPad est une société dont les valeurs correspondent aux miennes, que ce soit dans les pratiques ou les solutions développées.

\paragraph{} Cela m'amène au point suivant, puisque, comme nous l'avons vu, NaturalPad est une société innovante se positionnant sur le marché des serious games. En correspondance avec ma formation, elle m'a permis d'étendre très largement mes connaissances à ce sujet en m'offrant la possibilité d'effectuer un travail de recherche et d'état de l'art. J'ai ainsi beaucoup appris sur les serious games en général, les théories d'apprentissage ou encore la réhabilitation. Ces recherches me seront particulièrement utiles tout d'abord au niveau des connaissances théoriques acquises, mais aussi au niveau de la démarche de travail effectuée. Je souhaite ainsi poursuivre ce type de travail au moyen d'une thèse, sur les serious games ou la difficulté notamment. 

\paragraph{} J'ai aussi proposé plusieurs outils d'aide à la conception ainsi qu'une méthodologie de conception  de serious games pour la santé. Ces travaux m'ont permis d'explorer différentes pistes, de communiquer avec des chercheurs et des thérapeutes et 
leur proposer ces solutions. De nombreuses perspectives sont possibles pour faire évoluer ces outils ou valider la méthodologie, mais les résultats acquis durant le stage sont prometteurs.

\paragraph{} Durant mon stage, l'équipe de NaturalPad a su me faire confiance en me laissant autonomie, prises de décisions et responsabilités. Après avoir assisté à plusieurs séances de conception participative, Antoine Seilles a jugé que j'étais en mesure de mener ma propre séance en m'invitant à diriger une séance au centre hospitalier d'Alès Cévennes (CHAC). \\
J'ai aussi rencontré de nombreux professionnels de la santé de ma propre initiative, ergothérapeutes ou kinésithérapeutes, en fonction des besoins, contribuant à faire évoluer mon travail dans la bonne direction.

\paragraph{}
Enfin, ce stage fut globalement une très bonne expérience. J'ai beaucoup appris technologiquement grâce aux connaissances et aux partages de mes collègues. J'ai aussi passé six mois de réelle bonne humeur, chacun étant libre de s'exprimer et de proposer.
Dans cette optique, l'équipe a par ailleurs réalisé deux game jam d'une semaine durant la période où j'y ai travaillé. Une game jam consiste en la création d'un jeu sur une période courte. Ce fut l'occasion pour chacun de faire jouer son imagination, de mettre en place ses idées et de changer de contexte de travail d'une manière des plus agréables. Les mini-jeux réalisés nous plaisent d'ailleurs particulièrement, et je recommande à tout fan de jeux vidéo de se lancer dans cette expérience.