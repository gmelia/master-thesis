\subsection{Réhabilitation et serious games pour la santé TODO}

		\subsubsection{Adaptation de la difficulté en rééducation fonctionnelle }
		aujourd'hui (exercices de récup sensorielle, de mouvements, etc)
Existence de jeux sérieux à but thérapeutique ayant pour but de faciliter la réhabilitation en maintenant la motivation du patient. Cependant, ces jeux sont encore rares et ne remplissent pas encore parfaitement leur rôle à cause du problème de l'ajustement de la difficulté en fonction du joueur.
Or comme l'a vu, la difficulté joue un rôle important dans la satisfaction et la motivation du joueur. La question se pose donc de savoir comment ajuster de manière la difficulté d'un jeu afin qu'elle sied au mieux à chaque joueur, à chacune de ses sessions, dans le but final de renforcer la récupération motrice du joueur-patient. On va par ailleurs chercher à fournir le meilleur environnement virtuel possible pour chaque situation, avec ici comme objectif de contexte à terme un couple patient-thérapeute avec considérations des objectifs thérapeutiques. 
		
		\subsubsection{personnes âgées}
pas/peu d'expérience de jeu, voir même des nouvelles technologies, capacités physiques restreintes, background social à prendre en compte (jeux de cartes préférés aux jeux de guerre)
		\subsubsection{hémiplégiques}		
Contraintes spécifiques.  