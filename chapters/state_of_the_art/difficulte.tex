%%PART 3 : LA DIFFICULTE
\subsection{Difficulté}
Dans son livre \emph{La cigale : jeux, vie et utopie}, le philosophe Bernard SUITS indiquait : \begin{quote}{“Jouer consiste à tenter volontairement de surmonter des obstacles inutiles”}.  \end{quote}
		
	\subsubsection{Définition et propriétés}
La difficulté s’inscrit comme l’un des principes de base dans la création d’un jeu vidéo, et l’un des mécanismes pourvoyeurs de plaisir principaux de celui-ci. Il n’y en en effet pour un joueur rien de plus frustrant qu’un jeu à la difficulté inexistante ou à l’inverse tout bonnement injouable de part sa difficulté excessive. \paragraph{}

Dans sa thèse, Guillaume Levieux [Levieux, 2011]\cite{Levi11} propose de définir la difficulté d’un jeu vidéo comme l’effort fourni par le joueur pour atteindre ses objectifs. La difficulté d’un jeu n’est pas une donnée stable et suit un processus qui doit être en constante évolution. Le niveau du joueur varie en effet au fil du jeu, du fait de son expérience et de son apprentissage, et la difficulté doit donc s’adapter. 
La difficulté n’est donc en fait pas une propriété du jeu mais la valeur de la relation entre le jeu et le joueur. Or en jouant, le joueur progresse, découvre l’univers du jeu et parfait sa connaissance de la mécanique du jeu, devient capable d’heuristiques pour prévoir les conséquences de ses actions, augmente ses capacités de coordination oculo-manuelle et sa vitesse de réalisation des actions. La difficulté est donc variable, et tend à diminuer au cours du temps. La figure \ref{evolution_difficulte} illustre l'évolution de la difficulté d'un jeu au cours du temps.\\
Notons que le niveau du joueur peut aussi varier à la baisse, si il ne joue pas au jeu pendant un certain temps par exemple. \\

G. Levieux précise donc dans sa définition de la difficulté, qu’il est important de prendre en compte son aspect relationnel avec le joueur et introduit alors les notions de difficulté absolue et relative~:
	\begin{itemize}
		\item la difficulté absolue d’un jeu décrit l’effort que doit fournir un joueur type, aux capacités statiques, pour atteindre les objectifs que son gameplay propose. 
		\item la difficulté relative d’un jeu décrit l’effort que doit fournir le joueur, dont les capacités évoluent tout au long du jeu, pour atteindre les objectifs que son gameplay propose.
	\end{itemize}
\paragraph{}Pour maintenir la difficulté relative du jeu, il est donc nécessaire d’augmenter la difficulté absolue du jeu en fonction de l’évolution des capacités du joueur.

\begin{figure}[!htbp]
	\centering
	\includegraphics[width=11cm]{images/evolution_difficulte.png}
	\caption{Evolution de la difficulté d'un jeu au cours du temps}
	\label{evolution_difficulte}
\end{figure}

\paragraph{}La difficulté augmente si l’on resserre les contraintes et délais d’exécutions des actions, si on ajoute de nouveaux éléments qui augmentent la complexité du système ou si l’on découvre une nouvelle partie de l’univers demandant ainsi une appréhension du système plus étend. A l’inverse, la difficulté tend à baisser pour le joueur qui travaille son habileté, ou enregistre le mécanisme de nouveaux éléments ou parties du jeu.

		\subsubsection{Types de difficulté}
Lorsqu'on pense aux jeux vidéo, on envisage naïvement deux types de difficultés : la difficulté de compréhension, et la difficulté d’exécution. Autrement dit, des jeux où il est difficile de savoir ce qu’il faut faire, et d’autres où il est difficile de réussir à le faire. En fait, tout jeu relève à la fois des deux types de difficultés, du moins dans une certaine mesure. C’est d’ailleurs un moyen de différencier un jeu casual (un peu des deux difficultés) d’un jeu hardcore (une des deux ou les deux, mais bien plus conséquentes). Cependant, ces deux types de difficultés ne s’opposent pas de manière binaire. Les jeux à haute difficulté d’exécution vont souvent être des jeux basés sur un gameplay classique mais en une version très difficile et poussée, alors que les jeux à haute difficulté de compréhension vont relever soit de leur propre genre, soit d’un genre nouveau unique. \\

Durant sa thèse, Guillaume Levieux\cite{Levi11} a tenté de mesurer le niveau de difficulté de plusieurs jeux, comme \emph{PacMan}(qui dépend directement de la vitesse de déplacement du joueur et des fantômes). En s’inspirant d’un modèle de traitement de l’information, il a identifié trois niveaux de difficulté~:
	\begin{itemize}
		\item la difficulté sensorielle qui correspond à la perception de l’univers,
		\item la difficulté logique se référant à la compréhension de l’univers,
		\item et la difficulté motrice, en rapport avec l'exécution physique de l’action à effectuer.
\end{itemize}
\paragraph{}L’effort du joueur n’est pas directement mesurable à partir de l’historique de jeu, mais ses résultats le sont. Le problème c’est que l’effort n’est pas normalisé et dépend de chaque style de jeu. La difficulté réside donc dans une relation entre un joueur et le défi qu’il doit relever. La difficulté est en effet relative aux capacités des joueurs : nous n’éprouvons pas tous les mêmes difficultés pour les mêmes jeux ni aux mêmes endroits. Ce qui signifie qu’il faut définir le niveau de capacité des joueurs pour évaluer le niveau de difficulté du jeu.\\
La difficulté d’un problème n’a rien à voir avec la complexité : c’est un point de vue humain sur un problème. Une solution est de mesurer les échecs et leur évolution dans un jeu, le taux d’échec étant le résultat visible du niveau de difficulté pour une personne.

\begin{figure}[!hbtp]
	\centering
	\includegraphics[width=\linewidth]{images/dimensions_difficulte.png}
	\caption{Dimensions de difficulté}
	\label{dimensions_difficulte}
\end{figure}

\paragraph{}Guillaume Levieux [réf] définit donc trois types de difficultés dans le jeu vidéo :
	\begin{itemize}
		\item la difficulté sensorielle : décrit l’effort que doit fournir le joueur pour obtenir de nouvelles informations sur l’état de l’univers du jeu. Ces informations nouvelles correspondent à toute information que le joueur ne peut pas déduire des faits et règles logiques qu’il connaît déjà.
		\item la difficulté logique : décrit l’effort que doit fournir le joueur pour exploiter les informations dont il dispose, c’est à dire comprendre le fonctionnement de l’univers par induction, et choisir la prochaine action à réaliser par déduction.
		\item la difficulté motrice : décrit le niveau de précision spatiale et temporelle dont le joueur doit faire preuve lorsqu’il exécute une action.
\end{itemize}
A titre d’exemple, on peut associer un type de jeu par type de difficulté. Les jeux d’aventure se basent essentiellement sur la difficulté sensorielle, les jeux de stratégie sur la difficulté logique, et les jeux d’actions sur la difficulté motrice. Bien sur, chaque jeu est composé de chacune des trois dimensions, mais exploitées dans des proportions différentes.
		
		\paragraph{\emph{Punitivité}\\ \quad}
Il s’agit de différencier la difficulté du jeu de la punition en cas d’échec. Ces punitions peuvent être dans l’ordre de sévérité : le respawn instantané, celui avec délai, la sauvegarde libre, le checkpoint, les vies limitées et la permadeath (mort immédiate et définitive). Ainsi, un jeu peut être très difficile mais peu punitif (\emph{Super Meat Boy}) ou plus facile mais très punitif (\emph{Binding of Isaac}, \emph{Diablo} en mode hardcore). Lorsqu’il est à la fois difficile et punitif, le jeu entre alors dans la catégorie des jeux Hardcore.

		\paragraph{\emph{Le casual et le hardcore}\\ \quad}
Difficulté et punitivité contribuent donc, parmi d’autres facteurs, à créer une relation entre le jeu et le joueur. Plus celles-ci vont être élevées, plus on va s’éloigner du jeu casual pour se rapprocher du jeu hardcore, où un véritable investissement devient nécessaire pour accomplir le jeu. Il nécessite alors un temps d’investissement important ou une concentration soutenue (difficulté), chaque action va peser (punitivité) et demander au joueur de s’investir, à l’inverse du jeu casual.		
		
		\subsubsection{Pourquoi aime-t'on la difficulté?}
			\paragraph{\emph{Chimiquement} \\ \quad}
Le jeu vidéo est capable de fournir aux joueurs des sensations permettant de délivrer au cerveau dopamine ou adrénaline. L’expérimentation du flow state permet aussi au joueur un ressenti qu’il va chercher à renouveler.

			\paragraph{\emph{L’engagement} \\ \quad}
Le jeu vidéo a par ailleurs cette particularité de faire que le joueur va avoir la volonté de recommencer un niveau ou une partie après un échec. Et cette volonté aura tendance à augmenter tant que le joueur n’aura pas atteint son objectif. Cette constatation peut être expliquée par la théorie psychosociale de l’engagement, et plus particulièrement du concept de dépense gâchée. Selon cette théorie, plus on a passé de temps dans une activité, à apprendre quelque chose ou dans une réalisation, moins on est enclin à y renoncer, sous prétexte du temps inutilement passé à s’y consacrer. Dans le jeu “je ne vais pas abandonner après être arrivé aussi loin !”. L’engagement (et l’attachement aux valeurs) est d’autant plus important que l’investissement a été important, que ce soit en terme de temps, d’efforts, de sacrifices, de souffrance, etc.

			\paragraph{\emph{Dissonance cognitive} \\ \quad}
Par ailleurs, l’humain (entre autre) est mal à l’aise et ressent une tension désagréable lorsqu’il est en état de dissonance cognitive. Cette dissonance est ressentie lorsque l’individu est en présence de cognitions (connaissances, croyances ou perceptions de soi ou son environnement) contradictoires ou incompatibles entre elles.
Cet état entraîne un inconfort psychologique, parfois une réaction émotionnelle, qui pousse la personne à penser ou agir. pour rétablir son équilibre cognitif à l’aide de stratégies inconscientes de rationalisation. L’éveil peut prendre bien des formes, la soumission, la rationalisation, la fuite, un comportement ou une action délibérée, la modification de ses croyances, attitudes ou connaissances pour les accorder avec la nouvelle cognition. Dans le jeu vidéo, cela se traduit par une auto justification de la persévérance du joueur, ou un rejet radical de l’activité. On va se trouver des excuses, etc.

			\paragraph{\emph{Découverte et apprentissage}  \\ \quad}
Ces principes sont primordiaux pour un certain nombre de joueurs.  Que ce soit la découverte d’un monde immense, des capacités de son personnage, des mécanismes du jeu ou encore d’un univers particulier, le plaisir réside dans le fait que rien n’est acquis et se découvre à force d’expérimentations et d’échecs. Au fur et à mesure de ses expériences et observations, le joueur va alors suivre une courbe de progression généralement logarithmique très gratifiante qui va l’inciter à poursuivre son apprentissage pour parfaire sa maîtrise du jeu. Cet intérêt est d’ailleurs suffisamment fort pour qu’une certaine communauté de joueurs complète ses connaissances à l’aide de forums, wiki ou vidéos qu’elle aura elle même mis en ligne.

			\paragraph{\emph{Auto-détermination} \\ \quad}
R. Ryan et al propose d’expliquer la motivation du joueur à travers l’auto-détermination. Ils considèrent que les jeux vidéo satisfont des besoins psychologiques et permettent le développement d’un sentiment d’autonomie, de compétence et de connexion. L’autonomie décrit à la fois le fait que l’investissement du joueur est volontaire et que le joueur possède une autonomie au sein du jeu.

			\paragraph{\emph{Auto satisfaction et dépassement de soi} \\ \quad}
Le plus grand plaisir qu’un joueur peut ressentir en jouant à un jeu difficile ou hardcore, est le sentiment d’auto satisfaction lorsqu’il réussit enfin à accomplir son action. A force d’efforts ou d’entraînement, il réussit à réaliser ce qu’il croyait impossible au premier abord, parce qu’il ne comprenait pas comment y arriver ou n’était simplement pas capable de le faire, par manque de techniques, d’imagination ou d’entraînement. Ce dépassement de soi (technique, intellectuel ou physique) est déjà gratifiant en soi, et récompense le long apprentissage auquel s’est adonné le joueur.

C'est la \textcolor{orange}{réussite} d'un challenge \textcolor{orange}{difficile} qui est satisfaisante, et non directement son accomplissement. À l'inverse, une majorité va choisir une difficulté normale plutôt que facile ou difficile, car les gens aiment \textcolor{vert}{faire} quelque chose qui représente un \textcolor{vert}{challenge modéré} : pas le choisir ou le réussir, moins glorieux.

			\paragraph{\emph{L’enjeu} \\ \quad}
C’est aussi un grand pourvoyeur de plaisir. Un fort enjeux va inciter le joueur à ne pas jouer à la légère, à s’impliquer et donc à s’appliquer dans sa partie. On notera que l’enjeu est d’autant plus fort lors de partie multijoueur : les actions d’un joueur peuvent potentiellement influencer l’expérience de jeu de chacun des autres joueurs. En confrontation, il faut arriver à surpasser l’autre joueur, qui va faire de son mieux pour vous en empêcher. En collaboration, où l’erreur de l’un peut alors aussi coûter aux autres. L’enjeu crée alors un sentiment de tension, qui va lui même renforcer l’immersion du joueur.

\paragraph{}Enfin l’intérêt des joueurs pour les jeux difficiles ou réputés comme tels, peut aussi s’expliquer par une certaine \emph{nostalgie}, une forme d’\emph{élitisme} voir de snobisme envers les jeux/joueurs dits casuals, mais surtout aussi par le plaisir ressenti par la réussite d’un défi qui leur est posé. Une forme de frustration idéalement dosée et que l’on a surmontée.

\newpage
		
\subsubsection{\emph{Encart proposition \\} Proposition d'une nouvelle composante : la difficulté émotionnelle}
Nous avons vu dans notre recherche documentaire que sont définies trois types de difficultés dans les jeux vidéo. [Levieux, 2011]\cite{Levi11} définit ainsi la difficulté sensorielle, la difficulté logique et la difficulté motrice.

\paragraph{}Il est aussi possible d’envisager une dimension émotionnelle dans la difficulté. Cette difficulté peut se manifester lors de la réalisation d’une action donc la réussite ou non est importante pour le joueur, lors d’une confrontation avec une situation, un problème ou un objet dont le joueur a peur ou le rend particulièrement mal à l’aise par exemple : mise en situation d’une phobie, d’une scène en désaccord avec ses moeurs ou convictions, lui rappelant des évènements difficiles ou traumatisants, etc.
 \paragraph{}
On peut citer l’exemple du jeu \emph{Paper Please}, dans lequel on incarne un employé travaillant à un poste de frontière et qui contrôle l’accès au pays. Dans ce jeu, le joueur sera partagé entre respecter les consignes strictes d’immigration et le caractère émotionnel et personnel des personnes souhaitant entrer dans le territoire avec des histoires et des motivations personnelles, personnes pour lesquelles il nous faudra décider si on autorise ou on restreint l’accès. Cette décision pourra être particulièrement difficile car elle se fera au risque de perdre son emploi et ne plus pouvoir faire survivre sa famille ou d’être en profond désaccord, voir en situation de dégoût, avec soi-même...\paragraph{}
Le joueur peut aussi s’imposer lui-même un certain nombre de contraintes, pour être en accord avec ses principes. Ces contraintes peuvent être d’ordre moral ou éthique (refus de tuer un personnage dans le jeu ou d’effectuer une mauvaise action), ou plus artificiel comme vouloir jouer de manière “Role Play” et donc s’interdire certaines actions ou au contraire s’en imposer d’autres. Ainsi, même si le joueur sait qu’il gagnerait à réaliser une action particulière et qu’il est en mesure d’y parvenir, il ne passera pas nécessairement à l’œuvre. 
\paragraph{}On pourra ainsi citer l’exemple du jeu \emph{Valkyrie Profile}, dans lequel le joueur peut contrôler un personnage principal, ainsi qu’un groupe de personnages secondaires. Durant les phases de combats tactiques, le joueur a la possibilité de sacrifier un personnage secondaire afin d’obtenir une puissance phénoménale, qui se révélera souvent nécessaire d’acquérir tant la difficulté du jeu est élevée. Mais ces sacrifices sont permanents et l’avatar, ainsi que le joueur, devront les assumer et vivre avec la conscience d’avoir tuer ces personnages, ce qui fera évoluer différemment l’histoire.
\paragraph{}
Un autre aspect émotionnel se trouve dans l’acceptation du déroulement du jeu. Dans un jeu multijoueur compétitif ou opposant une IA, si l’adversaire emploie une stratégie ou une technique particulièrement frustrante pour le joueur, celui-ci peut s’en trouver affecté (colère, énervement, mauvaise estime de soi, mauvaise foi). Si cette situation continue ou est répétée, ou bien que malgré une difficulté modérée notre joueur continue de perdre ou de se faire mener en bateau pour une raison ou une autre, la difficulté émotionnelle deviendra telle qu’il pourra préférer abandonner. Cette situation particulièrement fréquente dans les jeux multijoueur en ligne peut mener à ce que l’on appelle couramment un rage quit, qui désigne familièrement le fait pour un joueur de quitter une partie en cours sous l'effet de la colère.

\paragraph{}On peut noter que cette difficulté peut en fait impacter directement les trois autres aspects de la difficulté précédemment cités. Un joueur qui aura réellement peur perdra de ses capacités sensorielles, logiques ou motrices par exemple. Cet impact n’est cependant pas systématique : une situation obligeant le joueur à réaliser une action allant à l’encontre de ses principes n’affectera pas ses capacités, mais le joueur hésitera cependant à réaliser l’action qu’il sait nécessaire ; il aura compris la situation, trouvé la solution à sa réalisation et est physiquement capable de la réaliser, mais ne souhaitant pas la faire, différera son exécution, voir l’évitera si possible.

\paragraph{}Dans le cas particulier d’un Serious Game pour la réhabilitation motrice, la difficulté émotionnelle peut se situer dans l’intérêt particulier qu’a le joueur patient dans l’évolution de sa pathologie. Il est nécessaire en phase de rééducation que le patient soit capable de sentir qu’il progresse afin de garder sa motivation et poursuive son travail. Une confrontation trop fréquente à des gestes qu’il n’est pas encore/toujours pas capable de réaliser parce que trop difficile, aura pour conséquence de lui rappeler sa déficience et pourra lui faire perdre toute ambition thérapeutique.
		
	\subsubsection*{Relation entre les différents paramètres d'un jeu vidéo}
Afin de mieux comprendre pourquoi le jeu vidéo est un média apprécié et comment ses différents paramètres peuvent être utilisés pour des objectifs sérieux, j'ai cherché à trouver le lien entre ces composantes. Dans un contexte de rééducation, on va aussi chercher à connaître quelles théories sont intéressantes pour les thérapeutes, pour qu'ils puissent réaliser un classement par importance pour la thérapie. Comprendre les relations qu'il existe entre le jeu vidéo et le joueur pourrait aussi permettre de mieux comprendre les mécanismes en jeu et mieux orienter les exercices d'éducation ou de réhabilitation.
\begin{figure}[htbp]
Schéma inspiré des théories comportementales et psychologiques et de concepts mis en place dans les jeux vidéo.
	\centering
	\includegraphics[height=19.6cm]{images/lien_theories}
	\caption{Relation entre les principaux ressorts psychologiques d'un jeu vidéo}
	\label{lien_theories}
\end{figure}			

	\subsubsection{Difficulté dans les Serious Games}
Les SG ont la particularité de conjuguer les mécanismes classiques du jeu vidéo à des objectifs sérieux de nature différente. Ces objectifs peuvent être la transmission de connaissances ou de valeurs si la visée est intellectuelle, ou bien un travail sur la forme ou les capacités physiques du joueur. Dès lors, la difficulté du jeu se dote d’une nouvelle composante relative à cet objectif thérapeutique.\\
Dans le cas de SG physiques, qui utilisent des périphériques comme la wii board, la kinect ou le PSmove par exemple, on pourra assimiler cette nouvelle composante à la difficulté motrice déjà définie. A la difficulté de synchronisation oculo-motrice de la main sur le contrôleur, s’ajoute des difficultés physique telles que la précision, l’endurance, l’équilibre ou la souplesse.
Dans les jeux dont l’aspect sérieux est intellectuel, l’objectif sérieux peut venir enrichir la difficulté logique du jeu (difficulté de compréhension, de raisonnement, de mémoire).
Dans les jeux sérieux dont le but est une rééducation psychomotrice, il est aussi important d’envisager un nouvel aspect de difficulté de type émotionnel. Il faut en effet prendre en compte l’enjeu médical et la possible fragilité du joueur, dont la progression ou non peut avoir un impact important sur son mental.
		
	\subsubsection{Ajustement de la difficulté}
Il est nécessaire d'adapter la difficulté pour chaque joueur pour garantir une expérience de jeu optimale.			
\begin{quote}Malone : “Pour être stimulant, un jeu doit proposer un but que le joueur n’est pas certain d’atteindre”.
\end{quote}

L’objectif de l’ajustement de la difficulté est de pouvoir faire correspondre la difficulté du jeu aux capacités et niveau de jeu du joueur, de manière à ce que quel que soit son niveau, le feedback difficulté puisse être identique. Dans leur ouvrage \emph{On Game Design} \cite{Andr03} Andrew Rollings et Ernest Adams précisent cependant que l’ajustement ne doit pas être trop évident ni visible, afin d’éviter toute forme d’exploit, évidemment non voulu. Ils précisent aussi que cet ajustement ne doit pas se faire au détriment de l’impact décisif de l’action du joueur ; celui-ci doit rester le facteur décisif de sa réussite ou non, indépendemment de l’ajustement réalisé par le système. On pourra noter qu’un niveau de difficulté idéal mènerait le joueur à un taux d’échecs/réussites de 50/50.

		\paragraph{\emph{Évaluation de la difficulté}\\ \quad}
Modifier la difficulté d'un jeu vidéo ne semble pertinent qu'après en avoir évaluer le niveau lors de phases de tests. Deux types de tests peuvent être mis en place. 
\begin{itemize}
	\item des tests de jouabilité, réalisés par des joueurs testeurs : fidèles mais couteux et complexes à mettre en place, définis dans le temps et subjectifs.
	\item des tests par joueur synthétique : rapides, peu chers et répétables mais basiques, sans perception subjective, ignorent les aspects perceptifs.
\end {itemize}

	\paragraph{\emph{Jeux de progression VS jeux d’émergence} \\ \quad}
Les jeux de progression sont scénarisés et le level design est défini et fixé à l’avance. La difficulté y est donc statique et prédéfinie selon les choix des designers. L’ensemble des situations du jeu a été pensé et calibré. Le parcours est relativement contraint et organisé. \\
Dans les jeux émergents, on ne peut prévoir l’évolution de la partie. Seul un certain nombre de règles permet de définir et de faire évoluer le contenu du jeu par application de ces règles et de l’interaction du joueur avec les objets du jeu. \\
Progression et émergence constituent en fait les deux formes opposées d’une catégorie décrivant le contrôle que possède le level designer sur l’expérience du joueur. On remplacera alors le contrôle manuel des designer par des algorithmes de génération appropriés. Un jeu pourra ainsi se situer le long de l’axe décrivant ce contrôle selon sa conception du level design (figure \ref{axe_scenaristique}). 
\begin{figure}[hbtp]
\centering
\includegraphics[width=5cm]{images/axe_scenaristique.png}
\caption{Axe scénaristique d'un jeu vidéo}
\label{axe_scenaristique}
\end{figure}

\paragraph{}Il existe un sous domaine de l’intelligence artificielle qui cherche à lier contenus émergents et scénarisés afin de tirer parti des avantages respectifs de l’un et l’autre domaine tout en limitant leurs contraintes : la narration interactive. On cherche alors à créer un univers virtuel où l’on va pouvoir “aller n’importe où et faire n’importe quoi, quand on le souhaite” de manière cohérente à la fois d’un point de vu gameplay et scénario. Cela en adaptant par exemple le scénario en fonction du comportement du joueur (utilisation par exemple d’un Drama Manager).

			\paragraph{\emph{Scénarisation VS adaptation dynamique} \\ \quad}
Le problème de l’adaptation de la difficulté est abordable à partir de deux méthodes de game design : la scénarisation et l’adaptation dynamique de la difficulté.  Ces deux méthodes pourraient bénéficier d’une méthode générale de la mesure de la difficulté, pour s’adapter au mieux.

\paragraph{}La scénarisation est une manière d’encadrer l’apprentissage du joueur à l’aide d’un découpage du gameplay en phases successives. 

\paragraph{}L’adaptation dynamique permet de maintenir le niveau de difficulté du jeu cohérent avec les capacités du joueur. Ces méthodes d’auto-adaptation peuvent être très efficaces, notamment pour ajuster un certain nombre de paramètres. Des algorithmes peuvent ainsi restreindre la lisibilité du jeu ou relâcher les contraintes de réalisation d’une action en fonction des résultats du joueur. On peut alors soit jouer sur la difficulté de la tâche ou de l’action à réaliser, soit sur les moyens employables pour y arriver (vie, temps, capacités, temps de réaction, propriétés du joueur / de l’environnement, etc.). Attention tout de même à ne pas faire un “système parfait” qui ferait que le joueur ne serait alors plus le facteur déterminant de sa réussite.

		\paragraph{\emph{Équilibrage dynamique} \\ \quad}
L’équilibrage dynamique, ou ajustement dynamique, consiste à modifier un certain nombre de paramètres du gameplay afin de s’adapter au comportement du joueur [Levieux, 11]. Il faut cependant pour cela d’abord être capable d’évaluer l’équilibre du jeu, sans quoi toute modification serait infondée . Dans cet objectif, l’apprentissage dynamique est particulièrement exploré. Ces techniques permettent en effet de calculer automatiquement les meilleurs paramètres pour atteindre un but donné, la difficulté du gameplay en l'occurrence. Idéalement, le jeu serait ainsi capable de détecter quand et comment le joueur parvient à surpasser les obstacles qui lui sont proposés, puis d’y répondre en proposant des modifications visant à obliger le joueur à reconsidérer sa stratégie de jeu. Bien sur, il est possible que le système ne parvienne pas à détecter la stratégie (voir l’exploit) du joueur, ou soit incapable de fournir une réponse appropriée ou cohérente avec l’univers du jeu, d’un point de vue autre que le gameplay pur. En effet, de tels systèmes ne prennent pas en compte l’intégralité de la pensée du game designer, qui reste complexe et non entièrement formalisable (visée artistique, morale, intentions, proposition d’une ‘expérience’ de jeu, etc.).

\paragraph{}Un jeu est donc une activité qui demande un effort au joueur. Cet effort librement consenti par le joueur  doit être utile pour la réussite ou non à ce jeu, puisqu’il pourrait aisément être rendu vain ou superflu en modifiant les règles du jeu. La difficulté, définie comme l’effort réalisé par le joueur, est donc une composante essentielle du jeu de part sa relation avec le joueur.

		\subsubsection{Techniques d'adaptation dans les jeux ludiques et sérieux}
L'adaptation de la difficulté dans les jeux vidéo est une fonctionnalité importante qui permet d’individualiser et de contextualiser l'expérience de jeu. Dans le cas de serious games, elle permet également de gérer la frustration des joueurs-apprenants tout en augmentant leurs motivations [Hocine et al 11]. L'individualisation et la contextualisation du jeu pour chaque joueur-apprenant, notions déjà définies par [Gee, 05] comme importantes pour l'apprentissage, ont pour conséquence d'augmenter sa satisfaction tout en améliorant l’efficacité de la formation.
				
		\paragraph{}
La génération dynamique d’IA permet de modifier le niveau de difficulté en créant de nouvelles entités avec un niveau et des règles donnés, ou en modifiant des paramètres du gameplay en cours de jeu.\\
Par exemple, Andrade et al utilisent l’apprentissage dynamique pour ajuster la difficulté. L’algorithme consiste à utiliser une base de couples (action, état de jeu) associés chacun à une valeur d’efficacité, afin que le jeu choisisse pour chaque situation un comportement de l’efficacité souhaitée.			
				
		\paragraph{\emph{Systèmes adaptables VS systèmes auto-adaptatifs} \\ \quad} 
L’adaptation peut être définie comme une caractéristique exprimée au niveau d’un système, dans notre cas un système informatique, qui reflète sa capacité à se modifier structurellement en réaction à certains évènements bien identifiés (Andresen K. et al, 2005). Nous parlerons de système adaptable lorsque l’intervention humaine est nécessaire pour enclencher le processus de modification et de système auto-adaptatif si aucune intervention extérieure n'est nécessaire (Moisuc B., 2001)

\paragraph{}
Levieux précise que la génération dynamique d’IA permet de modifier le niveau de difficulté en créant de nouvelles entités avec un niveau et des règles donnés, ou en modifiant des paramètres du gameplay en cours de jeu.\\
Par exemple, Andrade et al utilisent l’apprentissage dynamique pour ajuster la difficulté. L’algorithme consiste à utiliser une base de couples (action, état de jeu) associés chacun à une valeur d’efficacité, afin que le jeu choisisse pour chaque situation un comportement de l’efficacité souhaitée.

		\paragraph{\emph{Adaptation de la difficulté dans les jeux sérieux} \\ \quad}
Contribuer à l'acceptation et à l'utilisation des jeux sérieux constitue un enjeu majeur pour la réussite et l'efficacité de ceux-ci. En effet, ces systèmes sont destinés à satisfaire les joueurs-apprenants et à répondre à leurs besoins en termes d'acquisition de compétences et/ou de divertissement. L’adaptation a pour but d’améliorer l’utilisabilité d’un jeu sérieux ou ludique en restructurant certaines de ses propriétés.

\paragraph{}
[Hocine et al, 11] se proposent d'étudier les différents systèmes d'adaptation dans les jeux ludiques et sérieux. Pour évaluer ces systèmes, ils définissent trois critères majeurs :
\begin{itemize}
	\item L’efficacité d’un système évalue le degré de succès avec lequel les utilisateurs réalisent leurs objectifs dans le système.
	\item L’efficience évalue les moyens mis en œuvre par les utilisateurs pour accomplir leurs objectifs.
	\item La satisfaction évalue le niveau d’acceptation par les utilisateurs.
\end{itemize}

\paragraph{}
Afin d'évaluer et de comparer les différents systèmes, ils utilisent un système d'évaluation des techniques d'adaptation intéressant et très parlant du fait qu'il repose sur un modèle MVC (voir figure \ref{criteres_adaptation}).

\begin{figure}[!hbtp]
	\centering
	\includegraphics[width=1\linewidth]{images/criteres_adaptation.png}
	\caption{Critères d’analyse des techniques d’adaptation [Hocine et al, 2011]\cite{Hoci11}}
	\label{criteres_adaptation}
\end{figure}

	\paragraph{Périmètre d'adaptation\\}
Le périmètre de l'adaptation identifie le périmètre dans lequel l'adaptation est appliquée, selon le modèle MVC. 
\begin{itemize}
	\item \emph{Adaptation de la présentation (la Vue)}. L'adaptation peut donc avoir lieu au niveau de l'interface, du son ou de feedbacks envers l'utilisateur. Dans Hammer \& Planks, nous avons ainsi rendu possible la modification de ces paramètres : ajustement du volume de la musique ou des bruitages, contraste, taille des objets, vitesse de jeu et possibilité de désactiver les éléments cosmétiques facultatifs au jeu (animation de la mer, effets visuels etc.).
		\item  \emph{Adaptation du contrôle} : ce niveau englobe les règles du jeu et les règles métier qui spécifient la dynamique du jeu (ou le gameplay) en réaction aux actions des joueurs. C'est dans ce niveau que l'on adaptera le niveau de difficulté du jeu. Suite à mon travail sur le jeu Hammer \& Planks, il est possible de modifier le nombre et le type d'ennemis que doit affronter le joueur, la vitesse du jeu, les propriétés des personnages (joueur ou ennemis) ou encore la fréquence des obstacles par exemple.
		\item  \emph{Adaptation du contenu (le Modèle)} : l'adaptation à ce niveau modifie dynamiquement soit les schémas de données utilisés ou bien le contenu. L'adaptation s'efforce donc de produire un contenu lié au contexte de jeu et aux compétences des joueurs. Nous pouvons citer à titre d'exemple la génération automatique des dialogues et textes narratifs (Barry G., 2007) ou d'ambiance sonore (Chen Y et al, 2006)
\end{itemize} 

	\paragraph{Paramètres d'adaptation\\}
Ce sont les éléments sur lesquels repose la prise de décision du processus d'adaptation. Ces éléments vont être utilisés soit comme déclencheurs de l'adaptation soit comme sources de données. Hocine et al distinguent deux types de paramètres en fonction de l'utilisateur~:
\begin{itemize}
	\item \emph{Modèle utilisateur} : ensemble de variables et métriques décrivant les caractéristiques de l’utilisateur dans le système. Ces caractéristiques peuvent être des données représentant les préférences de l’utilisateur, son état attentionnel, ses émotions et/ou ses compétences. Ces données sont stockées dans le profil de l’utilisateur qui sera utilisé comme paramètre de processus d’adaptation.
	\item \emph{Paramètre non-utilisateur ou variable système }: ce paramètre représente les variables propres au système et qui ne dépendent pas du modèle utilisateur. A titre d'exemple, nous pouvons citer les paramètres liés à la configuration matérielle et logicielle du système hôte.
\end{itemize}

	\paragraph{Modèle d'adaptation\\}
	L’adaptation peut être implémentée dans le système sous forme d’un module qui interagit avec le système pour modifier son comportement et sa structure. Ce module peut être~:
\begin{itemize}
	\item \emph{Implicite} : dans ce cas les procédures d'adaptation se retrouvent éparpillées et étroitement liées aux différents composants du système. Il serait dans ce cas difficile de séparer dans les instructions (ou le code source) les éléments qui incombent à l'adaptation des autres aspects.
	\item \emph{Explicite }: la technique d'adaptation utilise des modèles explicites comme un moteur de règles logiques, une matrice de décision ou des algorithmes d’IA.
\end{itemize}
	
	\paragraph{Adaptation Mono ou Multi-joueurs\\}
Contrairement aux jeux mono-joueur, l'adaptation dans un système multi-joueurs doit prendre en compte l'aspect collaboratif et l'hétérogénéité entre les joueurs tout  en maintenant une cohérence globale du jeu.
	
	\paragraph{\emph{Tour d'horizon de systèmes d'ajustement dynamiques dans les jeux vidéo}	 \\ \quad}
Durant son travail sur la difficulté dans les jeux vidéo, Levieux se propose d'étudier les différentes solutions d’équilibrage dynamique qui ont été mis en place dans le jeu vidéo. Bien que de tels systèmes, de part leur aspect automatique, ne nous seront pas directement utiles, il peut être intéressant d'en connaître les principales mises en œuvre. On distingue plusieurs types de solutions~: 
\begin{enumerate}
	\item l’apprentissage par renforcement [Sutton 98]~:
	\begin{itemize}
		\item jeux de combat : [Andrade 05], [Graepel 04]
		\item RTS : [Madeira 04], [Madeira 06], [Ula 05]
		\item FPS : [Lee-Urban 08]
	\end{itemize}
	\item le scripting dynamique, qui calcule des préférences à partir de règles écrites par le designer~:
	\begin{itemize}
		\item jeux d’aventure : Neverwinter Nights - Bioware
		\item RTS : Wargus [Spronck 05], [Spronck 06], [Spronck 08], [Timuri 07], [Ludwig 07]
	\end{itemize}
		\item évolution génétique et réseau de neurones (l’un et/ou l’autre)~:
	\begin{itemize}
		\item jeux d’actions [Demasi 05], [Spronck 02]
		\item RTS : [Ponsen 05], [Agogino 00]
		\item FPS : [Cole 04], [Thurau 03]
		\item jeux de sport : Fifa 99 -EA Games [Chan 04]
		\item puzzle : Tetris - Nintendo [Bohm 05]
		\item réalité virtuelle : [Yannakakis 09], [Yannakakis 07]
	\end{itemize}
		\item algorithmes de champs potentiels pour comportements stratégiques dans les FPS [Thurau 04]
		\item raisonnement au cas par cas dans les RTS : [Aha 05]
\end{enumerate}

	\subsubsection*{Comparaison avec les besoins de NaturalPad}
Dans leur état de l'art des techniques d'adaptation dans les jeux vidéo, [Hocine et al] définissent la différence entre les systèmes adaptables et les systèmes auto-adaptatif. Leur étude, comme celle de Levieux, se concentrent cependant sur les systèmes dont l'ajustement est automatique. Or, comme nous l'avons déjà dit, nos besoins et propositions sont plutôt de proposer des jeux mettant en place un système d'ajustement manuel. Un tel système permet à chaque thérapeute amené à utiliser un jeu sérieux pour la santé de notre environnement, d'adapter le jeu aux besoins thérapeutiques dont il aura précisément besoin selon le contexte : préférences thérapeutiques, pathologies et spécificités du patient ou encore état de santé ou de forme de celui-ci pour ne citer que quelques exemples.