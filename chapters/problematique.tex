\subsection{Sujet et objectifs initiaux}
	Dans ce stage, nous allons nous intéresser à l’adaptation de la difficulté dans un jeu à but thérapeutique sous contrôle d’un médecin. Les variables permettant d’ajuster la difficulté d’un jeu sont nombreuses et variées. Elles peuvent être facilement modifiées et combinées pour créer des objectifs de jeu. Cependant, ces objectifs de jeu n’ont pas nécessairement un sens pour le médecin. Il s’agira dans ce stage de définir avec un thérapeute les variables ou paramètres adaptées à l’adaptation de la difficulté pour le soignant.
	\paragraph{}
	Dans le cadre de ce stage, le stagiaire aura pour mission de:
	\begin{itemize}
		\item {récupérer auprès d’un soignant une liste exhaustive d’objectifs de jeu thérapeutique}
		\item {traduire ces objectifs en paramètres dans le jeu Hammer \& Planks}
		\item {proposer une solution pour suivre et analyser visuellement les progrès du patient relativement aux objectifs fixés par le soignant}
	\end{itemize}
	Le stagiaire devra participer à des séances de coconception avec un thérapeute et sera amené à se déplacer pour suivre des séances de tests auprès de patients.
		
\subsection{Contexte et besoins}
Lors de mon arrivée au sein de NaturalPad, il existant déjà une version du jeu Hammer \& Planks, outil dans lequel devait s'insérer mon travail. Présenté lors du MIG 2012, le jeu était encore surtout orienté grand public. Ma première mission fut donc de m'approprier l'application et de l'adapter pour une utilisation paramétrable dans un contexte thérapeutique. En effet, une utilisation thérapeutique implique d'ajuster les différents paramètres en fonction des capacités et des besoins du patient.

	\paragraph{}Les échanges avec les professionnels de la santé ont été au coeur de mon stage. La compréhension des besoins et des contraintes médicales étant primordiales pour proposer un produit adapté, je me suis naturellement tourné vers les thérapeutes et soignants pour acquérir les connaissances et le background qu'il me manquait. Il m'est par ailleurs rapidement apparu qu'on ne pouvait répondre aux différents besoins thérapeutiques explicités par les soignants avec un seul jeu vidéo, même paramétrable. C'est pourquoi je me suis aussi concentré sur l'aspect de conception avec les thérapeutes. 

	\paragraph{Orientation du travail de stage \\}
Rappelons que la société NaturalPad propose comme outil une plateforme web permettant d'accéder à des jeux sérieux, et de les paramétrer directement depuis celle-ci. Hammer \& Planks constitue ainsi le premier jeu accessible depuis cette plateforme et, bien que servant d'exemple des possibilités d'un jeu vidéo pour la santé, il est amené à être rejoint par d'autres serious games. C'est dans cette optique que mon travail durant ce stage s'est progressivement orienté vers une méthode de conception de serious games pour la santé. Cela a pour objectif de proposer une solution appropriée aux différents besoins et contraintes de chaque situation. Ces derniers peuvent correspondre aux objectifs thérapeutiques, à la pathologie du patient, ses capacités, son âge, sa maitrise des nouvelles technologies ou son aisance avec les jeux vidéo par exemple.
Ce travail s'inscrit donc toujours dans le but d'adapter le jeu vidéo aux besoins thérapeutiques, et est complémentaire à une adaptation des paramètres de jeux, qu'elle soit manuelle ou automatique.
 
\paragraph{}Pour cela, je redéfinirais le sujet de ce stage comme suit :\\
\textcolor{marron}{\emph{ {\large Proposition d'une méthodologie de conception de jeux vidéo sérieux à but thérapeutique, et adaptation de la difficulté.}}}

\subsection{Background}
	%%Définition d'un Jeu 
	\subsubsection{Définition d'un Jeu}
Le jeu vidéo est un média participatif en pleine extension et de plus en plus largement accepté et plébiscité par la population. \\Dans sa définition, un jeu vidéo est une extension du jeu au monde numérique en utilisant les technologies informatiques. Il s'agit donc en amont de bien comprendre ce qu'est un jeu. \\Dans son travail d’analyse, Juul\cite{Juul05} donne une synthèse qui regroupe les points partagés par toutes les définitions existantes~:
\begin{quotation}
A game is a rule-based formal system with a variable and quantifiable outcome, where different outcomes are assigned different values, the player exerts effort in order to influence the outcome, the player feels attached to the outcome, and the consequences of the activity are optional and negotiable. 
\end{quotation}
\paragraph{Points clefs \\}
Les 6 points clefs identifiés par Juul sont :
\begin{itemize}
	\item les règles
	\item le résultat quantifiable variable
	\item la valorisation du résultat
	\item l’effort du joueur
	\item l’attachement du joueur au résultat (identifié par Gendler comme Alief, 2009)
	\item des conséquences négociables.
\end{itemize}

		\subsubsection*{Pratique et consommation des jeux vidéo}
Le jeu vidéo est un média récent qui trouve ses origines dans le début de la seconde moitié du XX$^{eme}$ siècle. D'abord majoritairement accessibles sur des bornes d'arcade, les jeux vidéo se sont progressivement répandus avec la commercialisation de consoles de salon. Selon une étude commandée par le Centre National du Cinéma et de l'image animée\href{http://www.cnc.fr}{(CNC)}, c'est près de 60\% de la population française qui jouaient à des jeux vidéo au cours de l'année 2011\cite{Cnc11}. L'âge moyen des joueurs était alors de 34,7 ans et cette population constituée de 54,1\% d'hommes fin 2011\cite{Cnc11}. L'arrivée de la console Wii et de jeux orientés plus casuals\footnote{Se dit d'une personne ne jouant à des jeux vidéo que de manière occasionnelle. Peut aussi se dire d'un jeu dont la cible sont des joueurs occasionnels.} a permi en France une meilleure diffusion du média au sein de la population, qui accepte de plus en plus de se livrer à cette activité. Contrairement à ce que l'on pourrait croire, notamment à cause de leur méconnaissance des nouvelles technologies et du jeu vidéo plus spécifiquement, les seniors se prêtent aussi volontier à la pratique de ce loisir numérique. C'est une constation que nous avons pu vérifier par nous-même lors de nos différentes sessions de tests au centre hospitalier de Lapeyronie et dans une maison de retraite à Lille, mais aussi dans différentes lectures qui seront détaillées ci-après.
	
		\subsubsection*{Propriétés des jeux vidéo}
En 1984, [Driskell and Dwyer]\cite{Dris84}réalisent un premier état de l'art et trouvent que plusieurs caractéristiques des jeux vidéo peuvent influer sur les propriétés d'apprentissage et de motivation : des objectifs spécifiques, un challenge, de la fantaisie et du mystère. Ils théorisent qu'une augmentation de la motivation produit une augmentation de l'attention et mène à une meilleure mémorisation des acquis (connaissance déclarative) et une focalisation de l'attention (stratégie cognitive).\newline
[Malone et Lepper, 1987] mentionnent le challenge, la curiosité, le contrôle et la fantaisie comme caractéristiques intégrantes des jeux vidéo. Selon [de Felix et Johnson , 1993], les jeux sont composés d'éléments visuels dynamiques, d'interactions, de règles et d'objectifs. Puis [Thiagarajan, 1999] affirme que le conflit, le contrôle, la terminalité et l'artifice sont les quatre éléments nécessaires d'un jeu. En 2001, [Garis and Ahlers, 01] donnent 39 descripteurs qui seront réduits à 12 pour ne garder que les paramètres statistiquement les plus significatifs pour renforcer la sensation de "game-like". \\ Finalement en 2002, [Garris et al] proposent un sous-ensemble de tous ces attributs qui seront considérés commes les paramètres de jeu clefs de l'apprentissage~:
\begin{itemize}
	\item la fantaisie
	\item les règles
	\item la stimulation sensorielle
	\item le challenge
	\item le mystère
	\item le contrôle
\end{itemize}

\paragraph{}En 2009, [Wilson et al]\cite{Wils09} partent de l'ensemble de paramètres de [Garris et al] pour enrichir le modèle avec de nouveaux paramètres qui leur semblent avoir un impact sur l'apprentissage.
Les tableaux \ref{game_attributes_one} et \ref{game_attributes_two} indiquent et décrivent ces douze attributs.

\begin{figure}[htbp]
	\centering
	\includegraphics[width=\linewidth, height=\textheight ]{images/game_attributes_one}
	\caption{Propriétés des jeux vidéo et leur définition. (1/2) \cite{Wils09}}
	\label{game_attributes_one}
\end{figure}
\begin{figure}[htbp]
	\centering
	\includegraphics[width=\linewidth, height=\textheight ]{images/game_attributes_two}
	\caption{Propriétés des jeux vidéo et leur définition. (2/2) \cite{Wils09}}
	\label{game_attributes_two}
\end{figure}	
	
	\subsubsection{Jeux sérieux et Théories Comportementales}
Les jeux vidéo se révèlent être un outil dont l’impact peut dépasser la simple portée ludique. Les serious games se proposent de profiter du ressort ludique du jeu vidéo pour servir volontairement un objectif sérieux distinct. Jeux éducationnels, commerciaux, idéologiques ou d’entraînement font partie de cette famille des jeux sérieux. D’un point de vue thérapeutique, il est possible de les utiliser afin de rendre le travail de réhabilitation ou de remise en forme plus motivant pour le patient en combinant les aspects ludique et thérapeutique.

\paragraph{}
Les jeux vidéo sérieux se présentent comme un potentiel médiateur dans la modification des habitudes comportementales en permettant d’inclure des connaissances pratiques dans un modèle ludique apprécié. Il est possible d’y mettre en place des procédures de changement comme l’établissement d’objectifs ou la modélisation et le développement de compétences dans un environnement attrayant, significatif et immersif [Baranowski et al, 2008]\cite{Bara08}. \\
Les jeux vidéo promeuvent les interactions sociales et d’apprentissage [Wideman et all, 2008], créent un environnement où les actions du joueur ont un effet [Gee, 2004]\cite{Gee04} , encouragent la résolution de problèmes [Gee, 2004]\cite{Gee04} et renforcent la compréhension en créant des situations de réflexion ou en aidant le joueur dans ses objectifs [Gee, 2004]\cite{Gee04}. Enfin, les jeux sérieux pour la santé sont fait pour distraire le joueur tout en l’éduquant, en l’entrainant ou en changeant ses comportements [Stokes, 2005]\cite{Stok05}.

		\subsubsection*{Théories comportementales}
Le comportement est la résultante d’influences multiples, rendant ainsi souvent les personnes réfractaires au changement [Baranowski, Lin \& al, 1997]\cite{Bara97}. Le comportement doit alors être considéré comme un mécanisme complexe découlant de l’enchaînement de plusieurs étapes. Ainsi, plutôt que de chercher à impacter directement le comportement, les experts comportementaux valorisent une action sur ces facteurs intermédiaires, appelés médiateurs. Changer ces médiateurs permet de changer le comportement [Baranowski, Lin \& al 1997]\cite{Bara97}.
\paragraph{}Plusieurs grandes théories comportementales existent~:
\begin{itemize}
	\item la théorie d’inoculation comportementale [McGuire, 1961]
	\item la théorie socio-cognitive [Bandura, 1986]
	\item la théorie de l’auto-détermination [Ryan \& Deci, 2000]
	\item la théorie de l’immersion [Green \& Brock, 2000]
\end{itemize}

\paragraph{}De ces théories, l’on peut alors identifier un certains nombre de ces facteurs médiateurs tels que : l’immersion, l’attention, l’auto-régulation, le développement de compétences, la motivation interne et externe, l’autonomie ou encore le sentiment de compétence. La science du comportement fournit aussi des techniques qui facilitent le changement comportemental, et propose d’utiliser ces facteurs dans les média de divertissement tel que le jeu vidéo.\\
Le modèle : "Elaboration Likelihood Model" [Petty \& Cacioppo, 1986] soutient ainsi que des personnages crédibles, attrayants et sympathiques sont plus susceptibles d’être persuasifs que les autres et peuvent donc servir d’intermédiaires pour véhiculer un message. \\
La théorie d’inoculation comportementale [McGuire, 1961] met en garde contre une possible contre-productivité en identifiant et en réfutant les menaces potentielles à l’accomplissement des objectifs du changement désiré.\\
La théorie socio-cognitive [Bandura, 1986] préconise l’établissement d’objectifs et le développement de compétences comme paramètres importants dans le changement comportemental.\\
Enfin, les théories socio-cognitive [Bandura, 1986] et d’auto-détermination [Ryan \& Deci, 2000] mettent toutes deux l’accent sur l’importance du feedback pour guider et mettre en forme le comportement durant le processus de changement.	

	\subsubsection{Méthodologies de conception}
On peut naïvement imaginer deux approches de conception s’opposant dans la conception des jeux sérieux. La première consisterait, de partir des objectifs sérieux et de proposer une “gamification” de ceux-ci en ajoutant des éléments de jeu. La seconde, à l’inverse, serait de partir de la composante ludique du jeu pour y intégrer ensuite le contenu sérieux. Bien qu’ayant l’avantage d’être simples à concevoir, ces deux démarches ont pour limite d’avantager l’une ou l’autre des composantes. Un jeu sérieux conçu à partir d’une base ludique aurait un impact sérieux limité, alors que l’ajout d’une composante ludique à une finalité sérieuse serait peu convaincant.
Une troisième approche est donc de prendre en considération à la fois la composante ludique et l’intention sérieuse dès le début du processus de conception pour les fusionner au mieux. Si elle est correctement mise en place, cette approche promet une forte utilisation du jeu et un impact sérieux efficace. Elle présente néanmoins le défaut de devoir imaginer une nouvelle solution pour tout nouveau couple (jeu, objectif sérieux). C’est ce type d’approche que nous proposons ici, et nous verrons comment les résultats peuvent être réemployables.

	\subsubsection*{Participative design : conception participative}
La conception participative est une méthode de travail utilisée principalement en conception de logiciel interactif. Sa principale caractéristique est la participation active des utilisateurs au travail de conception. Il s'agit donc d'une méthode de conception centrée sur l'utilisateur où l'accent est mis sur le rôle actif des utilisateurs \cite{wiki:cp}.

\paragraph{}
Il existe dans la littérature de nombreuses variantes de la méthode et de nombreuses techniques utilisées pour impliquer efficacement les utilisateurs. On peut noter particulièrement~:
\begin{itemize}
    \item l'observation et entretiens
    \item la production de scénarios
    \item le brainstorming
    \item le prototypage papier
    \item le prototypage vidéo
\end{itemize}

\paragraph{}
Une première séance de conception a lieu en début de projet : celle-ci regroupe les chercheurs et les développeurs de l'application, mais aussi les utilisateurs futurs ou potentiels. Intégrer les utilisateurs au processus de conception permet d'entrevoir au mieux leurs besoins et d'éviter un maximum d'erreurs d'interprétation ou d'oublis (voir illustration figure \ref{projet_info}). Les utilisateurs aident aussi à définir les problèmes éventuels et leurs possibles solutions.
\begin{figure}
	\centering
	\includegraphics[width=16cm]{images/projet_info.jpg}
	\caption{Allégorie d'un projet informatique}
	\label{projet_info}
\end{figure}

\paragraph{}Par ailleurs, plusieurs séances peuvent avoir lieues tout au long du projet, afin de vérifier si les besoins ont évolués et si le développement actuel est effectivement en accord avec ceux-ci. Les utilisateurs aideront aussi l'équipe de recherche et développement à juger de la pertinence des solutions apportées.
\paragraph{}
D'un point de vue du développement, on notera que ce type de conception s'accorde parfaitement avec une méthodologie AGILE, et notamment la méthode SCRUM qui consiste en de courtes périodes de développement entre lesquelles on met en relief l'avancement par rapport au projet global.

	\subsubsection{Jeux sérieux thérapeutiques}
Dans notre contexte, l’aspect sérieux recherché du jeu est la réhabilitation motrice ou la rééducation physique du joueur. Ces objectifs sont indiqués par des thérapeutes, médecins ou kinés, notamment dans le cadre de réhabilitation de personnes hémiplégiques ou souffrant de douleurs lombaires. L’intérêt du jeu vidéo est alors de proposer un environnement de réhabilitation plus agréable et de faciliter l’acceptation des travaux de rééducation par le patient grâce aux éléments de gameplay. Le jeu peut permettre un plus grand volume de travail de la part du patient car celui-ci sera plus enclin à les réaliser dans le cadre du jeu sérieux. Surtout, le jeu sérieux augmente la répétition de mouvements, ce qui améliore l'état physique du joueur-patient. L’objectif à terme est donc d’améliorer les résultats de l’objectif thérapeutique.

	\subsubsection{Contexte socio-médical}
		\paragraph{L'AVC et Hammer \& Planks\\}
Dans les pays occidentaux, près d'un individu sur 600 est victime d'un AVC chaque année, soit près de 120 000 accidents par an en France. L'AVC est une cause fréquente d'hémiplégie chez les victimes et reste une des principales causes d'invalidité. \\
La récupération des fonctions motrices, de la parole ou de la compréhension dépendent pour beaucoup de l'âge du patient et de son atteinte au niveau du cerveau.
\paragraph{}Hammer \& Planks est né d'un projet d'une étudiante en ergothérapie dont le but était de proposer un jeu servant à travailler l'équilibre chez des personnes hémiplégiques ayant été victimes d'un AVC. Aujourd'hui, H\&P peut être utilisé aussi bien pour travailler son équilibre, ses membres supérieurs son tronc ou sa capacité d'attention.

	\paragraph{Récupération de lombalgie\\}
Si la réhabilitation post AVC a été au cœur de mon travail et de celui de NaturalPad, son objectif est de pouvoir proposer ou accéder à des solutions pour divers types de pathologies. Ainsi, un projet de NaturalPad pour lequel j'ai participé à la phase de conception a pour objectif de créer un serious game pour la rééducation de personnes lombalgiques.
\paragraph{}
La lombalgie est un état douloureux du rachis lombaire qui peut être aiguë ou chronique. Les lombalgies affectent une forte proportion de personnes puisque entre 40 et 70\% de la population est touchée à un moment ou un autre. Sous l'effet de la douleur, une majorité des patients va cesser toute activité physique voir même professionnelle. Une de ses conséquences est aussi une démotivation de la personne pouvant aller jusqu'à un état de dépression, notamment dû à l'inactivité et la douleur. Comme préconisé dans le Guide du Dos\cite{backbook}, la reprise et le maintien d'une activité physique sont primordiaux dans le processus de récupération. \\
Le projet de NaturalPad est une application de coaching sportif adaptée à ce besoin et proposant un certain nombre d'exercices physiques gamifiés afin d'encourager la reprise d'activité des utilisateurs.	


\input{chapters/background_rehabilitation}

	\subsubsection{Système de recommandation }
	
Les systèmes de recommandation représentent les préférences de l'utilisateur dans le but de proposer des articles à acheter ou à examiner notamment. Dans notre problématique, un système de recommandation pourrait servir à sélectionner les paramètres de jeux, voir le jeu lui même, qui correspondraient le mieux aux besoins du joueur. Rappelons que ces besoins peuvent être soit explicites, notamment à travers les recommandations et exigences du thérapeute, soit plus inconscients. Ces besoins inconscients représentent pas exemple les préférences du joueur-patient en terme de gameplay. Un jeu plus distrayant et motivant pour le patient renforcera son implication dans le programme de réhabilitation, et donc son rétablissement. Pour cela il faut donc à la fois connaître les préférences du patient, explicites ou `découvertes'  grâce à un système d'apprentissage par exemple, mais aussi s'appuyer sur un certain nombre de théories et connaissances que l'on sait efficaces pour renforcer cette immersion. 	
	 
 \paragraph{}
 La proposition est ici de s'inspirer du monde la musique (ou des livres, des films, ou encore des ventes en ligne) et de son système de recommandation.\\
 On pense rapidement à deux types de recommandations. La recommandation sociale, qui consiste par exemple à conseiller à un utilisateur des musiques qu'apprécient des personnes de son réseau, surtout si elles écoutent généralement des musiques identiques. Un autre exemple sur les sites de vente en ligne, où l'on propose à un utilisateur venant d'acheter un objet, une liste de produits ayant été achetés en même temps par d'autres utilisateurs. \\
Le second type de recommandation se base pas non pas sur l'environnement social de l'utilisateur, mais sur le contenu même des objets recommandés. L'idée est alors de chercher à décrire un objet selon certaines caractéristiques, et à faire de même pour les préférences de l'utilisateur. On va ensuite lui conseiller les objets qui semblent être le plus proche des attentes de l'utilisateur en se basant sur ces critères de préférences. 
 
\paragraph{}
Pour Vincent Castaignet, fondateur et directeur de la publication de Musicovery, au-delà de la recommandation éditoriale, il y a différentes manières de proposer des artistes/titres par similarité~:
\begin{enumerate}
	\item d’après les formes musicales sur lesquelles le goût des auditeurs est fondé.
	\item d’après les repères mentaux utilisés par les auditeurs (genres, sous-genres, style).
	\item social  : si tu es membre de cette tribu et aimes cet artiste, alors tu vas aimer ces artistes.
	\item contextuel  : ceux qui écoutent ce titre dans ce contexte écoutent aussi ces titres.
\end{enumerate}
Les passionnés de musique qui cherchent activement préféreront de la similarité type (2), le grand public plus passif de la similarité type (4). Un moteur de similarité intelligent devrait pouvoir combiner ces différentes formes et s’adapter en fonction du profil de chacun.
Pandora est principalement construit sur (1), last.fm est (2) et (3), Musicovery (4). Deezer avec ces 30 millions de playlists a un actif considérable à exploiter en (3) et (4).

 		\subsubsection*{Les différents types de recommandation}
Plusieurs techniques de recommandation ont été proposées : basées sur le contenu, sur des connaissances ou encore des techniques dîtes collaboratives ou sociales. Pour de meilleurs résultats, certaines de ces techniques peuvent être utilisées conjointement dans des systèmes de recommandation hybrides.

	\paragraph{\emph{Propriétés des systèmes de recommandation} \\ \quad}
Les systèmes de recommandation possèdent :
\begin{itemize}
	\item des données de base : données que le système possède avant même de commencer la recommandation
	\item des données d’entrée : données que l’utilisateur fournit au système dans le but que ce dernier lui fournissent des recommandations.
	\item un algorithme qui utilise ces données de base et d’entrée pour générer les résultats.
\end{itemize}

	\paragraph{}
\begin{figure}[hbtp]
	\centering
	On peut distinguer 5 types de systèmes de recommandation
	\includegraphics[width=1\linewidth]{images/types_recommandation.png}
	\caption{Techniques de recommandations [R. Burke, 2002] \cite{Burk02} }
	\label{types_recommandation}
\end{figure}    
avec :
\begin{itemize}
	\item I : ensemble d’objets sur lequel sont faites les recommandations
	\item U : l’ensemble d’utilisateurs dont les préférences sont connues
	\item u : l’utilisateur pour lequel les recommandations doivent être générées
	\item i : objets pour lesquels on souhaiterait prédire une préférence de la part de u
\end{itemize} 

		\paragraph{\emph{Collaborative} \\ \quad}
Méthode la plus mature et répandue. Le système agrège les notes ou recommandations des objets, relève les similarités entre les appréciations des utilisateurs et en déduit de nouvelles recommandations pour les utilisateurs. Certains systèmes prennent le temps en paramètre dans leur évaluation afin de prendre en compte l’évolution de l’intérêt des utilisateurs au fil du temps (effet de mode, etc). L’évaluation peut être simplement binaire ou plus complexe en utilisant une échelle de graduation. Les systèmes peuvent être soit basés sur une mémoire, comparant les utilisateurs par corrélation ou autre, soit basés sur un modèle :  celui-ci est dérivé à partir de l'historique des données d'évaluation et utilisé pour faire les prédictions.
La plus grande force de ces techniques est qu’elles sont complètement indépendantes de la représentation informatique des objets recommandés.

		\paragraph{\emph{Démographique} \\ \quad}
Ces systèmes de recommandation ont pour but de catégoriser l’utilisateur à partir de ses caractéristiques propres et de faire des recommandations en fonction de son appartenance à l’une des classes démographiques prédéfinies. L’avantage d’une approche démographique est qu’elle ne requiert pas un historique des évaluations des utilisateurs à l’inverse des méthodes collaboratives et basées sur le contenu.

		\paragraph{\emph{Basée sur le contenu} \\ \quad}
 La recommandation basée sur le contenu est une excroissance et la poursuite de la recherche d'information de filtrage. Dans ces systèmes, les objets sont définis en fonction de leurs caractéristiques associées. Le système apprend à connaître le profil de l’utilisateur en se basant sur les caractéristiques des objets évalués par l’utilisateur. C’est une corrélation objet-à-objet. Le profil dérivé dépend évidemment du type d’apprentissage employé : arbre de décisions, réseaux de neurones et représentations par vecteurs sont utilisés.

		\paragraph{\emph{Fondée sur l’utilité} \\ \quad}
Ces systèmes font des suggestions en se basant sur une estimation de l’utilité de chaque objet pour l’utilisateur. Le problème central étant comment créer cette fonction d’utilité pour chaque utilisateur. D’abord évaluer les objets (différentes méthodes), puis le profil utilisateur, avant de calculer la correspondance entre les deux. L’avantage de la technique est qu’elle peut prendre en compte des attributs non directement propres aux objets évalués (fiabilité du vendeur, disponibilité du produit, etc.) pour proposer des recommandations plus pertinentes (besoin immédiat ou meilleur prix par ex.).

		\paragraph{\emph{Basée sur le savoir/connaissance} \\ \quad}
Tenter de recommander des objets en inférant les besoins et préférences de l’utilisateur. Ces systèmes se distinguent en ce qu’ils ont un savoir fonctionnel : ils ont connaissance que tel objet répond à tel besoin et peuvent alors abstraire la relation entre le besoin et une possible recommandation.

	\paragraph{}
Les systèmes de recommandations basés sur l’utilité et sur le savoir n’essaient pas de construire des généralisations à long terme à propos de leurs utilisateurs, mais préfèrent baser leurs conseils sur une évaluation de la correspondance entre les besoins d’un utilisateur et un ensemble d’options disponibles.

\subsection{Outils et méthodologie}
	\subsubsection{Méthodologie}
Afin de mener à bien ses projets, l’équipe de NaturalPad emploie une méthode Agile de gestion de projet : SCRUM.
Celle-ci définit 3 rôles :
	\begin{itemize}
		\item Le Product Owner
		\item Le Scrum Master
		\item Le Développeur
	\end {itemize}
Le Product Owner est le représentant des clients et des utilisateurs. Son objectif est de maximiser la valeur du produit développé. Il a pour rôle de rédiger des User Stories (comparables à des cas d'utilisation) et de valider le travail des développeurs. 
\\Le ScrumMaster est le responsable de la méthode. Il doit s’assurer qu’elle est correctement mise en application et comprise par les développeurs. Il organise le «Daily Scrum» (voir définition plus bas).
\\Enfin, le Développeur, représente en fait une équipe pluridisciplinaire et auto-organisée : toutes les décisions sont prises ensemble, sans hiérarchie externe ni interne.
 
		\paragraph{Daily Scrum :}
Il s’agit d’une réunion quotidienne ayant pour but de faire un point sur la coordination entre les tâches et les difficultés rencontrées.  Trois questions sont posées aux développeurs : 
	\begin{itemize}
		\item Qu’as-tu fait hier ?
		\item Qu’est-ce que tu vas faire aujourd’hui ?
		\item Est-ce que tu as rencontré des difficultés ?
	\end {itemize}
	
\paragraph{}Le travail est organisé sous forme de sprint. Il s’agit d’une courte période (au maximum un mois) au bout de laquelle l’équipe doit fournir une version améliorée du produit. Chaque sprint possède un but (ex : «on doit pouvoir envoyer des paramètres au jeu») et une liste de tâches (ex : «déterminer la méthode de communication, etc...»). Dès la fin d’un sprint, un nouveau est lancé.

\paragraph{}Enfin, une réunion a lieu en fin de sprint pour faire le point sur le travail accompli, les erreurs rencontrées et comment ne pas les éviter à l'avenir, ainsi que lancer le sprint suivant. Cette méthode est très intéressante car elle permet vraiment de garder une cohésion dans l’équipe de développement et d’avancer de manière visible. 

	\subsubsection{Outils}
		\paragraph{Gestion de projet\\}
Lors de mon arrivée dans l'entreprise, l’équipe utilisait Redmine, une application web de gestion de projets. Nous avons cependant changé deux fois d'outils de gestion de projet pendant la période de ce stage. Le premier est intervenu car les mises à jour des tâches dans Redmine étaient longues et l'outil finalement peu approprié à une méthodologie AGILE, ce qui freinait son utilisation. Nous avons donc mis en place une méthode Kanban qui consiste à écrire chaque tâche sur un post-it, et de déplacer ce post-it dans des colonnes «A faire», «En cours», «Terminé» ou «Validé» selon son avancement par exemple. De cette manière, l’avancement global était bien plus visible mais cette solution était finalement gourmande en post-it et en place. C'est pourquoi, nous utilisons désormais \href{www.trello.com}{Trello}, un outil de gestion de projet en ligne, se basant sur la méthode Kanban. Il s’agit d’un tableau virtuel dans lequel nous pouvons facilement déplacer les tâches, ajouter des commentaires ou des contraintes de temps notamment.

	\begin{figure}[!h]
		\centering
		\includegraphics[height=48px]{images/redmine.jpg}
		\includegraphics[height=48px]{images/trello.jpg}
		\caption{Logos de Redmine et Trello}
		\label{Logos de Redmine et Trello}
	\end{figure}

		\paragraph{Développement}
		\subparagraph{} \emph{Unity3D\\}
La majeure partie technique de mon travail a été réalisée pour Hammer \& Planks, qui est développé avec le moteur de jeu Unity3D. Au fil de mon stage, nous sommes passés de la version 3.9 à la version 4.1. Unity permet de facilement intégrer les modèles 3D des objets réalisés dans les logiciels de modélisation 3D tels que Photoshop, Gimp ou Maya. Il propose aussi des options permettant d'utiliser un gestionnaire de versions pour les fichiers du projet.
	\begin{figure}[!h]
		\centering
		\includegraphics[height=48px]{images/unity.jpg}
		\caption{Logo d'Unity3d}
		\label{Logo d'Unity3d}
	\end{figure}

		\subparagraph{} \emph{Git\\}
Que ce soit pour Hammer \& Planks ou nos autres projets en cours, l'utilisation d'un gestionnaire de versions se révèle vite indispensable. Travaillant en équipe allant jusqu'à cinq développeurs et une graphiste, il est nécessaire de pouvoir mutualiser le travail. De plus, l'expérimentation et le développement de nouveaux éléments se prêtent très bien à l'utilisation de plusieurs branches de développement, chose que Git permet de gérer facilement.
	\begin{figure}[!h]
		\centering
		\includegraphics[height=48px]{images/git.png}
		\caption{Logo de Git}
		\label{Logo de Git}
	\end{figure}

		\subparagraph{}	\emph{BitBucket et GitHub}
Pour héberger ses projets, NaturalPad avait l'habitude d'utiliser GitHub. Avec l'arrivée de nouveaux stagiaires, il nous a fallu trouver une solution permettant un accès privé au dépôt pour un plus grand nombre de personnes, ce que permet BitBucket.
	\begin{figure}[!h]
		\centering
		\includegraphics[height=48px]{images/bitbucket.jpg}
		\includegraphics[height=48px]{images/github.jpg}
		\caption{Logos de BitBucket et Github}
		\label{Logos de BitBucket et Github}
	\end{figure}

	\subsubsection{Veille}
Le Jeu Vidéo et plus généralement l'Informatique est un domaine en constante évolution dans lequel il est nécessaire de se tenir à jour pour connaître les dernières technologies et actualités. Pour cela, j'ai observé durant l'intégralité de ma période de stage une veille technologique et stratégique. Nouveautés technologiques, logiques ou matérielles, communications d'entreprises ou de salons nationaux et internationaux ou bien encore annonces de sociétés dont le secteur d'activité est compatible avec NaturalPad ont donc été au coeur de mon étude quotidienne.
\paragraph{}Pour faciliter ce travail de veille, par ailleurs inclu dans mon planning, j'utilise un agrégateur de flux RSS, outil indispensable pour gérer aisément un contenu important sur un grand nombre de sources différentes. Il s'agit ensuite de mettre à jour et d'étendre régulièrement les sources en fonction de l'utilité observée de chacune d'entre elle ou des manques ressentis.
\paragraph{Jeux Vidéo\\ \quad}
Étant étudiant en Informatique, option Image Game and Intelligent Agents, et ayant orienté ma formation vers une spécialité Jeux Vidéo, il m'a semblé important de me tenir à jour en terme d'actualité vidéoludique. J'ai pour cela étendu ma veille aux domaines des jeux vidéo, indépendants ou blockbusters, afin d'en étudier différents aspects tels le business model, le gameplay, les technologies employées ou les mécanismes de jeu innovants par exemple. J'ai ainsi pu testé des technologies récentes comme la console Ouya ou le système de contrôle Leap Motion, qui permet d'interagir en utilisant ses mains et ses doigts. Pour plus d'informations sur le Leap Motion, vous pouvez retrouver mon billet sur le blog de \href{naturalpad.fr/category/naturalblog}{NaturalPad}.
\begin{figure}
	\centering
	\includegraphics{images/leap_motion.jpg}
	\caption{Utilisation d'un Leap Motion}Une application retranscrit à l'écran la "vision" qu'elle a des mains de l'utilisateur.
	\label{leapmotion}
\end{figure}