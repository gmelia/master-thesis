NaturalPad est une société innovante dont le domaine d'activité est encore à ses débuts. Si les jeux vidéo sérieux commencent à être connus du grand public et leur impact reconnu, ceux pour la santé ne sont pas encore assez nombreux ni assez largement acceptés. Mon travail de stage s'inscrit en réponse à cette constatation : il s'agit de proposer une méthode de conception de jeux vidéo pour la santé qui permettrait une simplification ou une amélioration de la conception de tels jeux. Proposer plus de jeux  thérapeutiques et/ou des jeux avec un 'impact' santé de plus grande qualité contribuerait à améliorer leur diffusion et leur reconnaissance. Pour ces raisons, il était ainsi primordial de parfaire ma connaissance des différents domaines concernés.

Cette partie présente le résultat de mes recherches, des techniques et outils existants et comporte des notions importantes à connaître pour la conception de serious games pour la santé.

	\subsection{Jeux Vidéo}
Durant mon stage, j'ai ainsi voulu comprendre pourquoi et comment un jeu vidéo est bon, quels en sont les mécanismes ou bien encore ce qu'est la difficulté, pourquoi et comment l'adapter.\\

		\subsubsection{Théories cognitives}
		
		\subsubsection{La difficulté}
Dans son livre \emph{La cigale : jeux, vie et utopie}, le philosophe Bernard SUITS indiquait : \quote{“Jouer consiste à tenter volontairement de surmonter des obstacles inutiles”}.
		
			\paragraph{Adaptation de la difficulté}
				\subparagraph{personnes âgées}
pas/peu d'expérience de jeu, voir même des nouvelles technologies, capacités physiques restreintes, background social à prendre en compte (jeux de cartes préférés aux jeux de guerre) \cite{Csik75}
				\subparagraph{hémiplégiques}			
			\paragraph{Système de recommandation (faire de l'adaptation via ce genre de système)}		
La proposition est ici de s'inspirer du monde la musique (ou libres, films, ventes en ligne) et de son système de recommandation. Il existe en fait deux types de recommandations. La recommandation sociale consiste par exemple à conseiller à un utilisateur des musiques qu'apprécient des personnes de son réseau, notament si elles écoutent généralement des musiques identiques. Un autre exemple est sur un site de vente en ligne, de proposer à un utilisateur venant d'acheter un objet, une liste d'objets ayant aussi été achetés par d'autres utilisateurs en même temps que le premier objet.
Le second type de recommandation se base pas non pas sur l'environnement social de l'utilisateur, mais sur le contenu même des objets recommandés. L'idée est alors de chercher à décrire un objet selon certaines caractéristiques, et à faire de même pour les préférences de l'utilisateur. On va ensuite lui conseiller les objets qui semblent être le plus proche des attentes de l'utilisateur en se basant sur ces critères de préférences. 
 \paragraph{}
 Dans notre problématique, ce système de recommandation pourrait servir à sélectionner les paramètres de jeux, voir dans de futurs travaux le jeu lui même, qui correspondraient le mieux aux besoins du joueur. Rappelons que ces besoins peuvent être soit explicites, notament à travers les recommandations et exigences du thérapeute, soit plus inconscients. Ces besoins inconscients représentent pas exemple les préférences du joueur-patient en terme de gameplay. Un jeu plus distrayant et motivant pour le patient renforcera son implication dans le programme de réhabilitation, et donc son rétablissement. Pour cela il faut donc à la fois connaître les préférences du patient, explicites ou `découvertes'  grâce à un système d'apprentissage par exemple, mais aussi s'appuyer sur un certain nombre de théories et connaissances que l'on sait efficaces pour renforcer cette immersion. 		
	\subsection{Les serious games}
	
		\subsubsection{Serious games pour la santé}
Existence de jeux sérieux à but thérapeutique ayant pour but de faciliter la réhabilitation en maintenant la motivation du patient. Cependant, ces jeux sont encore rares et ne remplissent pas encore parfaitement leur rôle à cause du problème de l'ajustement de la difficulté en fonction du joueur.
Or comme on le verra plus tard, la difficulté joue un rôle important dans la satisfaction et la motivation du joueur. La question se pose donc de savoir comment ajuster de manière dynamique la difficulté d'un jeu afin qu'elle sied au mieux à chaque joueur, à chacune de ses sessions, dans le but final de renforcer la récupération motrice du joueur-patient. On va par ailleurs chercher à fournir le meilleur environnement virtuel possible pour chaque situation, avec ici comme objectif de contexte à terme un couple patient-thérapeute avec considérations des objectifs thérapeutiques. 		
	\subsection{La réhabilitation}
Un autre aspect est la dimension médicale de la rééducation. Connaître les enjeux, les contraintes, le contexte médico-social d'une réhabilitation ainsi que les techniques existantes ou les jeux thérapeutiques existants s'avéraient donc nécessaire.
	
		\subsubsection{Adaptation de la difficulté dans une rééduc fonctionnelle aujourd'hui (exercices de récup sensorielle, de mouvements, etc)}
		
		
	\subsection{Interfaces naturelles (NUI)}
		\subsubsection{Existant}
		
		\subsubsection{Jeux avec NUI (types de jeux)}
		
		
	\subsection{Méthodes de conception}
		\subsubsection{Participative design : conception participative}
		
		\subsubsection{Innovation games}
		
		\subsubsection{Impact mapping}