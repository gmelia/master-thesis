	-dire ce que je propose et pourquoi (pourquoi et pour quoi/qui, comment ça sera censé être utilisé, dans quel but) :
	\paragraph{partie théorique : difficulté émotionnelle et agencement des théories-attributs du jeu vidéo}
	\paragraph{mes propositions dans Hammer and planks pour ajuster la difficulté ou, alors dans réalisation??}
-> proposition d'une méthode de conception participative de JV thérapeutiques. Préciser les points importants de la conception, comme l'adaptation de la difficulté.
* propositions (théoriques) d'adaptation de de gameplay pour des NUI (m'inspirant de l'existant et des méthodes de rééduc)
	- parler ici de me la phase de veille / test, en testant le leap motion, des jeux sur Wii et Kinect PSMove, me permettant de faire des propositions plus pertinentes sur les controles.
* Cas pratique défini : travail sur l'équilibre (et lombalgie) et devices (kinect et wii)
*définition de la difficulté dans le jeu vidéo thérapeutique (ma position est que c'est différent des jeux classiques)