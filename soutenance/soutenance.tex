\documentclass{beamer}
\usepackage[utf8]{inputenc}
\usetheme{Warsaw}%{Berkeley} Madrid
\usecolortheme{Orchid} %dolphin, lily, orchid, structure sidebartab wolverine crane
\title[Conception et adaptation des Serious Games]{Serious games pour la santé :\\ Méthodologie de conception et adaptation de la difficulté}
\author{MÉLIA Geoffrey - Master 2 Informatique}
\institute{Université Montpellier II - NaturalPad}
\date{05 septembre 2013}

\AtBeginSection[]
{
\begin{frame}
\tableofcontents[currentsection,hideallsubsections]
\end{frame}
}

\begin{document}


\begin{frame}
	\titlepage
\end{frame}

%Slide d'intro avant le plan
\begin{frame}{Avant-propos}
	\begin{block}{NaturalPad}
		L'informatique au service de la santé.
	\end{block}		
	\begin{block}{Contexte}
		Apporter une solution pour adapter la difficulté dans un jeu pour la santé.
	\end{block}		
\end{frame}

%Plan
\begin{frame}{Plan}
	\tableofcontents
\end{frame}


\section{Introduction}
	\begin{frame}{Introduction}
		Soutenance de mon stage
	\end{frame}
	
	\begin{frame}{Avant-propos}
	\begin{block}{NaturalPad}
		
	\end{block}		
\end{frame}

\section{Problématique}
	\begin{frame}{Prob}
		Problématique de mon stage
	\end{frame}

	\begin{frame}
		\begin{itemize}
			\item Langage utilisé par Beamer: L\uncover<2->{A}TEX
			\item Langage utilisé par Beamer: L\only<2->{A}TEX
		\end{itemize}
	\end{frame}
	
\section{Proposition et réalisations}	
	\begin{frame}
	
	\end{frame}

\section{Perspectives}
	\begin{frame}
	
	\end{frame}

\section{Conclusion}
	\begin{frame}
	
	\end{frame}

\end{document}